Every 3--manifold is the boundary of some 4--manifold.
The earliest proofs of this fact are due to Ren\'e Thom in the early 1950's \cite{Thom}.
Since then the problem has been approached from multiple avenues.
A series of papers in the 1960's by Hirsch~\cite{Hirsch61}, Rokhlin~\cite{Rokhlin65}, and Wall~\cite{Wall65} built up the theory of immersing and then embedding 3--manifolds in $\R^5$.
The embedded 3--manifold then bounds a ``Seifert 4--manifold'' in the same vein as knots bounding  Seifert surfaces in $\R^3$.
Later results by Rourke~\cite{Rourke85}

Historical arguments prove 3--manifolds bound 4--manifolds, but we normally describe our 3-- and 4--manifolds using triangulations.
We would like a process that takes us from a triangulated 3--manifold to a triangulated 4--manifold.
The work done by Turaev in ~\cite{Turaev91} paved the way for an argument in ~\cite{CostThur08} that includes a process, initially used to estimate a bound on the number of 4--dimensional simplices needed to triangulate a 4--manifold whose boundary is an input 3--manifold, which is made algorithmic in this thesis.

Chapter 2 is entirely Geometric Topology fundamentals.  Most of the material is used only to support the main topics of chapter 2, which are handle attachment and definition of triangulation.

Chapter 3 is a build up to an argument that 3--manifolds bound 4--manifolds in the smooth case.  The argument is made so that it can be turned into an algorithm in the piecewise--linear case.  The main Section is 3.3.

Chapter 4 is the collection of algorithms.  The flow is laid out at the beginning.  Everything in this chapter is essential.