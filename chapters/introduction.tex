Every closed 3--manifold is the boundary of some 4--manifold.
The earliest proofs of this fact are due to Ren\'e Thom in the early 1950's \cite{Thom}.
Since then the problem has been approached from multiple avenues.
A series of papers in the 1960's by Hirsch~\cite{Hirsch61}, Rokhlin~\cite{Rokhlin65}, and Wall~\cite{Wall65} built up the theory of immersing and then embedding 3--manifolds in $\R^5$.
The embedded 3--manifold then bounds a ``Seifert 4--manifold'' as with knots bounding Seifert surfaces in $\R^3$.
Also in the 60's, Lickorish showed that all orientable 3--manifolds may be represented as surgery on a 3--sphere, and thus bound 4--manifolds.
A later result by Rourke~\cite{Rourke85} demonstrates something similar, this time by investigating Heegaard diagrams of the given 3--manifolds.
In the 90's, Turaev~\cite{Turaev91} discussed shadow surfaces as a method of studying quantum invariants.
He also showed that a 3--manifold as well as a 4--manifold whose boundary is that 3--manifold may both be reconstructed from the same shadow surface.

Expressing a manifold in the form of a triangulation allows for automated computations of manifold invariants, so a proof that takes us from a triangulated 3--manifold to a triangulated 4--manifold whose boundary is that 3--manifold is desirable.
The work done by Turaev in ~\cite{Turaev91} paved the way for Costantino and Thurston's ``3--manifolds Efficiently Bound 4--manifolds.''
The focus of the paper is a process used to estimate a bound on the number of 4--dimensional simplices needed to triangulate a 4--manifold whose boundary is a given 3--manifold.
This process follows the structure of a proof that 3--manifolds bound 4--manifolds in the smooth case, but that proof is only provided as a sketch.

The focus of this thesis is to provide a detailed proof of that closed, orientable 3--manifolds bound 4--manifolds.
Given a closed, orientable 3--manifold $M$ and a smooth generic function $f:M\to\RR$, we find an explicit construction of a 4--manifold $W$ whose boundary is exactly $M$.
Additionally, we provide a handle decomposition of $W$ encoded as a 2--complex $S$ plus a function from the faces of $S$ to the integers.
This 2--complex is the Stein factorization of $f$, and the pair $(S,f)$ would be a \emph{gleamed shadow} of $W$ in the work of Turaev.

%What follows is a brief overview of what to expect, only slightly more descriptive than our table of contents.
%\begin{itemize}
%	\item Chapter \ref{cha:manifolds} consists entirely of geometric topology fundamentals.
%	Most of the material is used only to support the main topics of chapter 2, which are handle attachment and the definition of triangulations.
%	A reader that is well read in those topics is encouraged to skim Chapter \ref{cha:manifolds} in order to familiarize themselves with the notation used.
%	\item Chapter \ref{cha:cobordisms} is a build up to an argument that 3--manifolds bound 4--manifolds in the smooth case.
%	We make this argument in a constructive way so that it can be turned into an algorithm in the piecewise--linear case.
%	The main focus is Section \ref{sec:3bound4}, and the first two sections cover definitions and the techniques of proof in the case of orientable 2--manifolds bounding 3--manifolds that we draw from in Section \ref{sec:3bound4}.
%	\item Chapter \ref{cha:algorithm} is our collection of algorithms.
%	We initialize some key definitions and present some minor results before diving directly into the algorithms.
%\end{itemize}