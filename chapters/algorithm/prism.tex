The basic building block that we used in our 4--manifold reconstruction is the triangulated $n$--prism with identical walls.
Take $\sigma^n$ to be the standard $n$--simplex.
By $n$--prism, we mean the $n$--disc seen as $\sigma^{n-1}\times \II$ in Figure \ref{fig:nprism}.
The prism has canonical ``top'' and ``bottom'' faces which are $\sigma^{n-1}\times \{1\}$ and $\sigma^{n-1}\times \{-1\}$ respectively.
It also has $n$ identical ``walls'' which are $F^i(\sigma^{n-1})\times \II$ for each $i=0,\dots,n-1$.
The triangulation we construct for the $n$--prism is designed so that the walls have identical triangulations.
We do this so that we are able to perform an $n$--thickening of an $(n-1)$--dimensional triangulation.

\begin{figure}
	\centering
	\captionsetup{justification=centering}
	\caption{Sample 1--, 2--, and 3--prisms}
	\includegraphics[height=3cm]{figures/dummy.jpg}
	\label{fig:nprism}
\end{figure}

If a pair of $\sigma^{n-1}$ are glued over a face, then we take those simplices to be the bottom faces of a pair of prisms.
We glue these prisms together over their corresponding walls.
This provides the two $\sigma^{n-1}$ with an $n$--dimensional thickening.
If we can do this to any pair of simplices, then we can do this to an entire triangulation, as in Algorithm \ref{alg:nthickening}

\begin{algorithm}[h]
	\caption{Thickening a triangulation}
	\label{alg:nthickening}
	\KwData{an $n$--dimensional triangulation $T$}
	\KwResult{an $(n+1)$--dimensional triangulation $H(T)$ that is the $(n+1)$--thickening of $T$}
	\Begin{
		$H(T)\longleftarrow \emptyset$\;
		\ForEach{$n$--simplex $\sigma$ of $T$}{
			add a triangulated $(n+1)$--prism $H(\sigma)$ to $T'$\;
		}
		\ForEach{face $F^i(\sigma)=F^j(\tau)$ over which a pair $\sigma$, $\tau$ of $n$--simplices are glued in $T$}{
			glue the $i\nth$ wall of $H(\sigma)$ to the $j\nth$ wall of $H(\tau)$\;			
		}
	}
\end{algorithm}

Considering $F^i(\sigma^{n-1})$ is an $(n-2)$--simplex, it is reasonable to build our prisms recursively.
The intuition is that we build an $(n-1)$--sphere whose walls are $(n-2)$--prisms, themselves with identical walls.
Then, we cone the $(n-1)$--sphere to make a triangulated $n$--prism.
The details are in Algorithm \ref{alg:nprism}, assuming the base case of the 1--prism being an edge.

\begin{algorithm}[h]
	\caption{Building a standard $n$--prism}
	\label{alg:nprism}
	\KwData{a positive integer $n$}
	\KwResult{$P^n$, a triangulated $n$--prism with identical walls}
	\Begin{
		\If{$n=1$}{
			$P^1=\II$\;
			
		} \Else {
			$\sigma_{-1}^n,\sigma_1^n\longleftarrow$ standard $n$--simplices\;
			$P^n\longleftarrow \sigma_{-1}^n\sqcup\sigma_1^n$\;
			$P_0^{n-1},P_1^{n-1},\dots,P_n^{n-1}\longleftarrow$ result of Algorithm \ref{alg:nprism} with input $n-1$\;
			
			\ForEach{$(n-1)$ prism $P_i^{n-1}$}{
				attach $\sigma_{-1}^{n-1}$ of $P_i^{n-1}$ to $F^i(\sigma_{-1}^n)$ of $\sigma_{-1}^n$ in $P^n$\;
				attach $\sigma_1^{n-1}$ of $P_i^{n-1}$ to $F^i(\sigma_1^n)$	of $\sigma_1^n$ in $P^n$\;
			}
			
			\ForEach{$n-2$ simplex $F^k(F^i(\sigma_{1}^n))=F^l(F^j(\sigma_{1}^n))$ of $\sigma_{1}^n$}{
				attach the wall of $P_i^{n-1}$ containing $F^k(F^i(\sigma_{1}^n))$ to the wall of $P_j^{n-1}$ containing $F^l(F^j(\sigma_{1}^n))$\;
			}
			$P^n\longleftarrow CP^n$, the cone on $P^n$\;
		}
	}
\end{algorithm}