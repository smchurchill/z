The basic building block that we used in our 4--manifold reconstruction is the triangulated $n$--prism with identical walls.
By $n$--prism, we mean the $n$--disc seen as $\Delta^{n-1}\times [-1,1]$ in Figure \ref{fig:nprism}.
The prism has canonical ``top'' and ``bottom'' faces which are $\Delta^{n-1}\times \{1\}$ and $\Delta^{n-1}\times \{-1\}$ respectively.
It also has $n$ identical ``walls'' which are $F^i(\Delta^{n-1})\times [-1,1]$ for each $i=0,\dots,n-1$.
The triangulation we construct for the $n$--prism is designed so that the walls have identical triangulations.
We do this so that we are able to perform an $n$--thickening of an $(n-1)$--dimensional triangulation.

\begin{figure}
	\centering
	\captionsetup{justification=centering}
	\caption{Sample 1--, 2--, and 3--prisms}
	\includegraphics[height=3cm]{figures/dummy.jpg}
	\label{fig:nprism}
\end{figure}

If a pair of $\Delta^{n-1}$ are glued over a face, then we take those simplices to be the bottom faces of a pair of prisms.
We glue these prisms together over their corresponding walls.
This provides the two $\Delta^{n-1}$ with an $n$--dimensional thickening.
If we can do this to any pair of simplices, then we can do this to an entire triangulation, as in Algorithm \ref{alg:nthickening}

\begin{algorithm}[h]
	\caption{Finding edges for the Stein complex}
	\label{alg:steinedges}
	\KwData{an edge $e=(u,v)$ of $X'$}
	\KwResult{the set $N(e)$ of pairs $(U,V)$ with $U\in N(u)$ and $V\in N(v)$ such that $U$ and $V$ are adjacent over an edge of $S$}
	\Begin{
		$N(u), N(v)\longleftarrow$ the results of Algorithm \ref{alg:steinvertices} with inputs $u$, $v$\;
		$N(e)\longleftarrow\emptyset$\;
		\ForEach{element $U$ of $N(u)$}{
			\ForEach{element $V$ of $N(v)$}{
				\If{$U\cap V\neq\emptyset$}{
					add $(U,V)$ to $N(e)$\;
				}
			}	
		}
	}
\end{algorithm}

\begin{algorithm}[h]
	\caption{$n$--thickening of an $(n-1)$--dimensional triangulation}
	\label{alg:stein2cells}
	\KwData{a 2--cell $c$ of $X'$ with $\{v_0,\dots,v_k\}=\pd c$}
	\KwResult{the set $N(c)$ of cycles $\{V_0,\dots,V_k\}$ where $V_i$ is in $N(v_i)$ such that a 2--cell is attached over $\{V_0,\dots,V_k\}$ in $S$}
	\Begin{
		$C\longleftarrow$ the result of Algorithm \ref{alg:regularcircles} with input $c$\;
		$N(v_i)\longleftarrow$ the result of Algorithm \ref{alg:steinvertices} with input $v_i$\;
		$N(c)\longleftarrow\emptyset$\;
		\ForEach{simplicial circle $C_j$ of $C$}{
			$N_j\longleftarrow\emptyset$\;
			\ForEach{integer $i$ in $[0,k]$}{
				\ForEach{element $V_i$ of $N(v_i)$}{
					\If{$C_j\in V_i$}{
						add $V_i$ to $N_j$\;
					}	
				}			
			}
			add $N_j$ to $N(c)$\;
		}
	}	
\end{algorithm}


Seeing that $F^i(\Delta^{n-1})$ is an $(n-2)$--simplex, it is reasonable to build our prisms iteratively.
The intuition is that we build an $(n-1)$--sphere whose walls are $(n-2)$--prisms, themselves with identical walls.
Then, we cone the $(n-1)$--sphere to make a triangulated $n$--prism.

The 1--prism is just an edge.
This is the base case, so no coning is needed.

To make the 2--prism, we take two copies of the 1--simplex: $e\times\{1\}$ and $e\times\{-1\}$.
The two face maps of $e$ give us the two vertices of $e$.
Then $F^0(e)=v_1$ and $F^1(e)=v_0$, so we now have the 1--sphere triangulated by the top face $e\times\{1\}$, bottom face $e\times\{-1\}$, and walls $v_1\times I$ and $v_0\times I$.
Coning gives us a triangulation of $\DD$ whose walls are single edges and whose top and bottom faces are 1--simplices.

To make the 3--prism, we take two copies of the 2--simplex: $t\times\{1\}$ and $t\times\{-1\}$.
The three face maps of $t$ give us the three edges of $t$.
Then $F^0(t)=[v_1,v_2]$, $F^1(t)=[v_0,v_2]$ and $F^2(t)=[v_1,v_2]$, so we now have the 2--sphere triangulated by the top face $t\times\{1\}$, bottom face $t\times\{-1\}$, and walls $[v_1,v_2]\times I$, $[v_0,v_2]\times I$ and $[v_1,v_2]\times I$.
The walls are 2--prisms with the previous triangulation.
The walls glue to the top and bottom triangles in the obvious way, and are glued to each other over their identical $1$--dimensional walls.
Coning gives us a triangulation of $\D^3$ whose walls are the 2--prism found above and whose top and bottom faces are 2--simplices.

The 4--prism construction is nearly identical to the 3--prism construction, but it's not a bad idea to write it down.
To make the 4--prism, we take two copies of the 3--simplex $T$: $T\times\{1\}$ and $T\times\{-1\}$.
The four face maps of $T$ give us the four triangles of $T$.
Then $F^0(T)=[v_1,v_2,v_3]$, $F^1(T)=[v_0,v_2,v_3]$, $F^2(T)=[v_0,v_1,v_3]$ and $F^3(T)=[v_0,v_1,v_2]$.
We now have a triangulated 3--sphere with top face $T\times\{1\}$, bottom face $T\times\{-1\}$, and walls $[v_1,v_2,v_3]\times I$, $[v_0,v_2,v_3]\times I$, $[v_0,v_1,v_3]\times I$ and $[v_0,v_1,v_2]\times I$.
The walls are 3--prisms with the previous triangulation.
The walls glue to the top and bottom faces in the obvious way, and are glued to each other over their identical $2$--dimensional walls.
This triangulation is seen to be $\sthr$ because it is constructed by gluing together three basic pieces.
The three pieces are the top face (a 3--ball), which is glued to the union of the walls (an $\stwo\times I$), which is glued to the bottom face (a 3--ball).
Coning gives us a triangulation of $\D^4$ whose walls are each the 3--prism found above and whose top and bottom faces are 3--simplices.

\begin{rmk} 
  Before moving further with construction, we should revisit the idea of an ``$n$--dimensional thickening.''
  The 3--thickening of the M\"obius band under the definition above gives the solid Klein bottle and, in general, the thickening as described produces the trivial interval bundle over the base triangulation.
  Our goal in later sections is to produce an interval bundle over a possibly nonorientable 3--handlebody $H$ whose total space is orientable.
  For a smooth bundle, this would amount to identifying all 1--handles whose attachments cause $H$ to be nonorientable, and defining the fiber bundle so that a transition function on each such 1--handle is the reflection diffeomorphism in the fiber direction.
  Our prisms are symmetric through switching of ``top'' and ``bottom,'' so such a strategy can be employed here as well.
\end{rmk}