We establish some properties of piecewise--linear maps from polyhedra to $\RR$, and develop an algorithm to produce an associated 2--dimensional CW--complex.
This CW--complex is used as a data type to keep track of the piecewise linear version of the generic proper smooth maps of Section \ref{sec:3bound4}.
In Section \ref{sec:stein}, we use the 2--complex produced to form a Stein complex.

\begin{defn}
	\label{def:stdproj}
	Let $\Delta$ be a tetrahedron with the four vertices $u$, $v$, $w$, $x$, six edges $uv$, $uw$, $ux$, $vw$, $vx$, $wx$, and faces $\hat{u}$, $\hat{v}$, $\hat{w}$, $\hat{x}$, where a face is named by the vertex of $\Delta$ it does not contain.
	Define a projection $\pi: \Delta \to\DD$ by first choosing a map from the vertices of $\Delta$ to distinct points in $\sone$.
	Each point $p$ of $T$ is decribed by the convex combination
	\[
		p = t_u u + t_v v + t_w w + t_x x
	\]
	with the $t_*$ nonnegative and summing to 1.
	We can define $\pi$ at $p$ by
	\begin{eqnarray}
		\label{affine_extension}
		\pi(p)
		&=&
		\pi(t_u u + t_v v + t_w w + t_x x) \nonumber \\
		&=&
		t_u \pi(u) + t_v \pi(v) + t_w \pi(w) + t_x \pi(x)
	\end{eqnarray}
	  
	Without loss of generality, we assume that the points $\pi(u),\pi(v),\pi(w),\pi(x)$ are ordered in a clockwise orientation about $\sone$.
	We call $\pi:T\to\DD$ a \emph{linear tetrahedral projection}.
\end{defn}

\begin{defn}
	\label{def:projpttypes}
	A point of $\DD$ in the image of $\pi$ is one of five types:
	\begin{enumerate}
		\item The four points of type 1 are the images of the vertices of $T$ under $\pi$.
		\item The single point of type 2 is the intersection $\pi(uw)\cap\pi(vx)$.
		\item All points in $\pi(uv)\cup\pi(vw)\cup\pi(wx)\cup\pi(xu)$, excluding the points of type 1, are of type 3.
		\item All points in $\pi(uw)\cup\pi(vx)$ excluding the points of type 1 or 2 are of type 4.
		\item The points of type 5 are the points outside of the image of any vertex or edge of $T$.
	\end{enumerate}
\end{defn}

By definition, the preimage of a point of type 1 is a vertex of $T$.
The preimage of the single point of type 2 is a line segment between the edges $uw$ and $vx$ interior to $T$.
A point of type 3 is the image of exactly one point in an edge of $T$.
A point of type 4 is in the image of exactly one edge and one face, so the preimage of one of these points is a line segment interior to $T$ between those two facets.
Finally, points of type 5 are in the image of exactly two faces of $T$, and pull back to line segments interior to $T$ between those two faces.

We naturally associate the 
The image of the $1$--skeleton of $T$ forms a planar graph $G$ inside of $\DD$ whose vertex set consists of the points of types 1 and 2 and whose edges consist of points of type 3 and 4.
The planar graph embedding cuts the plane into five connected regions: four inner regions that contain all points of type 5, and one outer region.
This graph has four outer edges, consisting of the points of type 3, and four inner edges, consisting of the points of type 4.

\begin{algorithm}
	\label{alg:lintetTOplanar}
	\KwIn{An ordering of the vertices of a tetrahedron linear tetrahedral projection $\pi$}
	\KwOut{A planar graph $(G,p)$}
	\KwResult{output}
	\caption{Planar graph corresponding to a linear tetrahedral projection}
\end{algorithm}


