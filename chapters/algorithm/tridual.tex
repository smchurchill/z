An edge distinct triangulation $T$ of a closed, connected, orientable 3--manifold is a simplicial complex that can be run through the algorithms of the previous sections.
We also consider the dual polyhedral complex to $T^*$ and observe how it changes as $T$ is changed.

To begin, let $\pi:T\to\C$ be a crossing distinct projection, and let $X$ be the 2--complex produced by Algorithm \ref{alg:crossdis2complex}.
We want $\pi$ to be the piecewise--linear analogue to the generic, proper, smooth map of Section \ref{sec:3bound4}.
Because $\pi$ is linear on each tetrahedron of $T$, the only analogues to critical values would occur on the images of edges of $T$.
This leaves the 2--cells of $X$ as our analogue to regions of regular values.
Under the classification in from Definition \ref{def:projpttypes}, these are points of type 5.
Restricting our focus to a single tetrahedron $\sigma$, a point of type 5 would pull back through $\pi$ to a line segment $s$ in $\sigma$ that intersects the faces $F^i(\sigma)$ and $F^j(\sigma)$.
Because $p$ is continuous and $T$ is closed, there are tetrahedra $\sigma_i$ and $\sigma_j$ that are incident to $\sigma$ over $F^i(\sigma)$ and $F^j(\sigma)$.
Then $s$ meets the endpoints of two more line segments that pass through $\sigma_i$ and $\sigma_j$.
This continues until the segments join up into a simplicial circle interior to $T$, just as in the smooth regular value theorem.

\begin{algorithm}[h]
	\caption{Building a collection of circles that map over a region}
	\label{alg:regularcircles}
	\KwData{a 2--cell $c$ of $X$, $T$, $T^*$, $\pi:T\to\C$}
	\KwResult{a collection $C=\{C_i\}$ of simplicial circles in $T^*$ representing the circles in $T$ that project through $\pi$ over the region in the plane that correspond to $c$}
	\Begin{
		$c_\pi\longleftarrow$ the region of type 5 points in the plane corresponding to $c$\;
		$C\longleftarrow\emptyset$\;
		\ForEach{edge $o^*$ in $T^*$}{
			\If{the dual 2--cell $o$ in $T$ has $c_\pi\subset\pi(o)$}{
				add $o^*$, including its boundary vertices, to $C$\;
			}			
		}
	}	
\end{algorithm}

The circles we find this way unambiguously define cycles in the 1--skeleton of $T^*$, the dual complex to $T$.
We use the dual complex as a central data structure alongside the 2--complex $X$ to store information about $\pi$.
The first example of this is in Algorithm \ref{alg:regularcircles}, where we find all of the simplicial circles in $T^*$ that map over a region of $X$.
We produce $C$, a collection of disjoint simplicial circles in $T^*$ that project over $c_\pi$, the region of type 5 points in the plane corresponding to $c$, and we decorate the associated 2--cell $c$ of $X$ by a positive integer: the number of connected components of $C$.

In Section \ref{sub:truncttet}, we deleted balls about each vertex of $T$ to find $T'$.
The best we can do to reflect this modification in $T^*$ is to delete the interior of each 3--cell.
This modification yields a 2--complex $T'^*$ and preserves the simplicial circles found during Algorithm \ref{alg:regularcircles}.

In Section \ref{sub:edgeblowup} we blew up edges into 2--cells.
By duality of dimension, the appropriate action to consider would be to split a 2--cell by an edge.
An edge $o^*$ of $T'^*$ connects vertices that are dual to 3--cells.
These 3--cells are adjacent over the 2--cell $o$ that is dual to $o^*$.
We can therefore consider an edge in $T'^*$ to indicate adjacency between a pair of 3--cells.

Let $E$ be the edge and $\sigma$, $\tau$ be the 3--cells that we blow up.
The modification of $T'^*$ appropriate to a blowup over $E$ then begins by introducing a 1--cell $\hat{E}^*$ into $T^*$ over the vertices $\hat{\sigma}^*$ and $\hat{\tau}^*$.
We then remove $E^*$ and introduce a pair of 2--cells corresponding to the new edges of $\hat{E}$ over the newly created chordless cycles containing $\hat{E}^*$.
We can see the modified structure in Figure \ref{fig:dualblowup}.

\begin{figure}
	\centering
	\captionsetup{justification=centering}
	\caption{Dual edge blowup}
	\includegraphics[height=3cm]{figures/dummy.jpg}
	\label{fig:dualblowup}
\end{figure}

It is worth noting here that we need only build directly the collection of circles that map over a single region.
Once that is built, the circles mapping over neighbouring regions are determined from our first collection.
This is especially useful once edges are blown up, as tracking the specifics of a projection $\pi$ through such modifications is tedious.

Let $X$ be a 2--complex generated as described by any of our algorithms from a polyhedral gluing $T$ and let $T^*$ be the appropriate dual complex to $T$.
Let $c$ be a 2--cell of $X$, let $c'$ be a neighbouring 2--cell over the edge $e$ in $X$, let $C$ be the collection of circles mapping over $c$, and $C'$ the collection over $c'$.
There is an edge $E$ of $T$ and dual 2--cell $E^*$ that correspond to $e$.
By Algorithm \ref{alg:regularcircles} the only set of edges in $T^*$ on which $C$ and $C'$ differ is $\pd E^*$.
We can then obtain $C'$ from $C$ by taking the symmetric difference of $C$ with $\pd E^*$.