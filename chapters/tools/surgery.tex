\label{sec:surgery}

%%%%%%%%%%%%%%%%%%%%%%%%%%%%%%%%%%%%%%%%%%%%%%%%%%%%%%%%%%%%%%%%%%%%
%	We need to let a manifold with corners (after handle attachment)
%	exist in order to describe the events of chapter 3 with any kind
%	of clarity.
%	
%	This can be done by refusing to smooth after attachment.	
%
%	We must also describe surgery on the boundary induced by attachment.
%
%	After that, we want to add more handles.  We can't limit ourselves to maps from smooth things into the boundary.
%\begin{rmk}
%	\label{rmk:corners}
%	There is more than one way to define handle attachment, and we choose to do so in a way that feels combinatorial in nature.
%	The main concern with this approach is that the object resulting from handle attachment is a manifold with corners rather than a smooth manifold.
%	There are arguments that the corners can be smoothed away in such a way that a manifold obtained via handle attachment is smooth and unique up to diffeomorphism.
%	Delving into such an argument at this point would be a distraction.	
%	A construction that does not require the smoothing of corners can be found in \cite{Kosi93}, but the machinery makes explicit handle attachment unnecessarily complicated.
%\end{rmk}
%%%%%%%%%%%%%%%%%%%%%%%%%%%%%%%%%%%%%%%%%%%%%%%%%%%%%%%%%%%%%%%%%%%%

We construct manifolds in later chapters mainly by gluing simpler manifolds together.
These operations include handle attachment and connected sum, special cases of the general construction of joining two manifolds along a submanifold.
Special attention is given to the definitions and results needed to meaningfully attach 1-- and 2--handles to a 4--manifold.

There is a common tool in topology called the \emph{attaching map} that we use in this section and the next to build our machinery.
We take this opportunity to lay out the definition and notation.

\begin{defn}
  Let $X$ and $Y$ be topological spaces, $A\subset X$ a subspace, and $f:A\to Y$ a continuous map.
  We define an equivalence relation generated by the relations $f(x)\sim x$ for every $x$ in $A$.
  Denote the quotient space $X\sqcup Y/\sim$ by $X\cup_f Y$.
  We call the map $f$ the \emph{attaching map}.  
  We say that $X$ is \emph{attached} or \emph{glued} to $Y$ over $A$.
  A space obtained through attachment is called an \emph{adjunction space} or \emph{attachment space}.
  
  Alternatively, we let $A$ be a topological space and let $i_X:A\to X$, $i_Y:A\to Y$ be inclusions.
  Here, the adjunction is formed by taking $i_X(a)\sim i_Y(a)$ for every $a\in A$ and we denote the adjunction space by $X\cup_A Y$.
\end{defn}

Our first application of adjunction is in the connected sum operation.

\begin{defn}
	Let $M$ and $X$ be closed, oriented $n$--manifolds.
	The manifold obtained by identifying the boundary $(n-1)$--spheres of $M\setminus\B^n$ and $X\setminus\B^n$ in an orientation reversing way is called the \emph{connected sum} of $M$ and $X$ and is denoted $M\# X$.
	Symbolically,
	\[
		M\# X = (M\setminus B^n)\cup_f (X\setminus B^n)
	\]
	where $f$ is any diffeomorphism $S^n\to S^n$.
\end{defn}

\begin{prop}[VI.1.1 of \cite{Kosi93}]
	Connected sum is a well defined operation up to diffeomorphism on the space of closed, oriented $n$--manifolds with identity element $S^n$.
\end{prop}

We present handle attachment in its standard form before delving into the cases involving manifolds with faces.
When discussing handles we have $n=\lambda+\mu$, $M$ a smooth $n$--manifold with nonempty $\pd_1 M$ and $\pd_j M=\emptyset$ for $j>1$, and $H^\lambda = \D^\lambda\times\D^\mu$.
Handle attachment is the process of joining $M$ and $H^\lambda$ along an embedding of $\pd\D^\lambda\times\D^\mu$ into $\pd M$.

\begin{defn}[Handle]
	\label{def:handle}
	Take $n=\lambda+\mu$ and $M$ a smooth $n$--manifold with nonempty $\pd_1 M$.
	%	 and $\pd_j M=\emptyset$ for $j>1$
	Let $\varphi:\pd\D^\lambda\times\D^\mu\to\pd_1 M$ be an embedding and an attaching map between $M$ and $H^\lambda$.
	The attached space $H^\lambda$ is an \emph{$n$--dimensional $\lambda$--handle}, and $M\cup_\varphi H^\lambda$ is the result of an $n$--dimensional \emph{$\lambda$--handle attachment}.
\end{defn}

\begin{prop}
	\label{prop:handle}
	Let $M'=M\cup_\varphi H^\lambda$ be the result of an $n$--dimensional $\lambda$--handle attachment.
	Then $M'$ is a smooth $n$--manifold with faces, the 1--boundary of $M'$ consists of $\pd M\setminus\varphi(\pd\D^\lambda\times\D^\mu)$ and $\D^\lambda\times\pd\D^\mu$, the 2--boundary is exactly $\varphi(\pd\D^\lambda\times\pd\D^\mu)$, and $\pd_j$ is empty for $j>2$.
\end{prop}

Our results regarding attaching manifolds over faces are adapted from those found in \cite{JSP}, Section 3.1.1.
They tell us that manifolds can be attached over faces, the attachment space is a manifold with faces, and the faces of the attachment space are well defined.

\begin{prop}
	\label{prop:addface}
	Let $X$ and $X'$ be smooth $n$--manifolds with faces, let $F$ be an $(n-1)$--manifold with faces, and let $\iota:F\to X$ and $\iota':F\to X'$ be inclusion maps such that $\iota(F)$ is a face of $X$ and $\iota'(F)$ is a face of $X'$.
	Then the adjunction $X\cup_F X'$ is a smooth $n$--manifold with faces.
	
	For each $j$, the $j$--boundary of $X\cup_F X'$ is decomposed into the $j$--boundary of $X$ minus the image of $F$ in $X$, the $j$--boundary of $X'$ minus the image of $F$ in $X'$, and the $j$--boundary of $F$ seen as a subset of $X\cup_F X'$.
	Symbolically, this is $$\pd_j( X \cup_F X') = (\pd_j X\setminus \iota(F))\cup(\pd_j X'\setminus \iota'(F))\cup(\pd_j F).$$
\end{prop}

The following proof is an adaptation from \cite{davis}
%\{ASIDE FOR EDITING:  The following proof is based on some informal notes by Jim Davis found \href{http://www.indiana.edu/~jfdavis/notes/m623/smooth_structures_and_handles.pdf}{here}. Is this even citable?\}

\begin{figure}[H]
	\centering
	\caption{Face gluing near a corner.}
	\includegraphics[height=3in]{figures/dummy.jpg}
	\label{fig:addface}
\end{figure}

\begin{proof}
	To prove this proposition we put a smooth structure on $X\cup_F X'$ so that if $x\in F$ is a $j$--corner then $\iota(x)\sim \iota'(F)\in X\cup_F X'$ is a $j$--corner as well.
	We first decompose $X$ into the triad $X=(X;K,M)$ where
	\begin{enumerate}
		\item $K=\iota(F)$,
		\item $M\cup K=\pd X$, and
		\item $M\cap K=\pd K = \pd M =\iota(\pd F)$.
	\end{enumerate}
	We do the same, respectively, to $X'=(X';K',M')$.
	
	Let $\psi:K\times\HH\to X$ and $\psi':K'\times\HH \to X'$ be collars of $K$ and $K'$ in $X$ and $X'$.
	Also take
	\[
		\begin{array}{lrcl}
			\varphi_K:&
			\pd F\times\HH&
			\to&
			K\\
			\varphi_M:&
			\pd F\times\HH&
			\to&
			M\\
			\varphi_K':&
			\pd F\times\HH&
			\to&
			K'\\
			\varphi_M':&
			\pd F\times\HH&
			\to&
			M'\\		
		\end{array}
	\]
	to be collars of $\pd K$ and $\pd K'$ in their respective codomains, ensuring that 
	\[
		\iota\inv\comp \varphi_K = (\iota')\inv\comp \varphi_K'.
	\]
	Defining
	\[
		\begin{array}{rcl}
			X\setminus K	&	\into	& 	X\cup_F X', \textrm{ and}	\\
			X'\setminus K'	&	\into	&	X\cup_F X'		
		\end{array}
	\]
	by inclusion, and
	\[
		\begin{array}{crcl}
			\Psi:&(F\setminus\pd F)\times\R&\into&X\cup_F X'\\
			\Phi:&(\pd F)\times\R\times\HH&\into&X\cup_F X'		
		\end{array}
	\]
	by
	\[
		\Psi(x,t)=
		\begin{cases}
			\psi(\iota(x),t),		& 	\mbox{if } t\geq 0 \\
			\psi'(\iota'(x),-t),	&	\mbox{if } t\leq 0
		\end{cases}	
	\]
	and
	\[
		\Phi(x,r\cos\theta,r\sin\theta)=
		\begin{cases}
			\psi(\varphi_M(x,r\cos(2\theta)),r\sin(2\theta))
				&	\mbox{if } \theta\in[\,0,\pi/4\,]		\\
			\psi(\varphi_K(x,-r\cos(2\theta)),r\sin(2\theta))
				&	\mbox{if } \theta\in[\,\pi/4,2\pi/4\,]	\\
			\psi'(\varphi'_K(x,-r\cos(2\theta)),-r\sin(2\theta))				
				&	\mbox{if } \theta\in[\,2\pi/4,3\pi/4\,]	\\
			\psi'(\varphi'_M(x,r\cos(2\theta)),-r\sin(2\theta))				
				&	\mbox{if } \theta\in[\,3\pi/4,4\pi/4\,]
		\end{cases}	
	\]
	gives $X\cup_F X'$ a smooth structure induced by the open cover
	\[
		X\cup_F X'=(X\setminus K)\cup(X'\setminus K')\cup\Psi((F\setminus\pd F)\times\R)\cup	\Phi((\pd F)\times\R\times\HH).
	\]
	In this open cover, the $j$--corners of $\iota(F)=\iota'(F)$, $j\geq 1$, appear only as $j$--corners of $\Phi((\pd F)\times\R\times\HH)$, hence they are $j$--corners in $X\cup_F X'$, thus the proposition is proved.
\end{proof}

\begin{cor}
	\label{cor:addfaces}
	Let $X$, $X'$ be manifolds with faces and let $F_1,\dots, F_k$ be faces of $X$ such that $F_i\cap F_j$ is a face of $F_i$ and $F_j$ for every $i\neq j$.
	Let ${\bf F}=\cup_i F_i$ and let $\varphi:{\bf F}\to \pd X'$ be an embedding such that $\varphi(F_i)$ is a face of $X'$ for each $i$.
	Then the adjunction $X \cup_{\bf F} X'$ is a smooth manifold with faces.
		
	If $p$ is a $j$--corner of $X$, $j\geq 0$, and $k$ is the number of faces in ${\bf F}$ that contain $p$, then $p$ is a $(j-k)$--corner of $X\cup_{\bf F} X'$.
	Similarly, if $p$ is a $j$--corner of $X'$, $j\geq 0$, and $k$ is the number of faces in $\varphi({\bf F})$ that contain $p$, then $p$ is a $(j-k)$--corner of $X\cup_{\bf F} X'$.
\end{cor}

These two results are our main tools for attaching manifolds with faces together over their faces.
The following is used to produce manifolds with faces.

\begin{figure}[H]
	\centering
	\caption{Attaching two copies of $\D^2\times\I$ together, disc in disc.}
	\includegraphics[height=3in]{figures/dummy.jpg}
	\label{fig:cyloncyl}
\end{figure}

\begin{prop}
	\label{prop:attachingManifoldsFaceUInterior}
	Let $X$, $X'$ be $n$--manifolds with, let $F$ be an $(n-1)$--manifold with faces and let $\iota:F\into X$ and $\iota':F\into X'$ be inclusion maps such that $\iota(F)$ is a face of $X$ and such that $\iota'(F)$ is contained in the interior of a face of $X'$.
	Then the adjunction $X\cup_F X'$ is a smooth $n$--manifold with faces.
	
	Moreover, for each $j$, the $j$--boundary of $X\cup_F X'$ may be decomposed as follows.
	For $j=1$, the 1--boundary of $X\cup_F X'$ is decomposed into the $1$--boundary of $X$ minus the image of $F$ in $X$ and the 1--boundary of $X'$ minus the image of $F$ in $X'$.
	For $j>1$, the $j$--boundary is  decomposed into the $j$--boundary of $X$ and the $j$--boundary of $X'$.
	
	An example of this type of attachment is illustrated in Figure~\ref{fig:cyloncyl}.
\end{prop}

Note that Proposition~\ref{prop:handle} is a special case of Proposition\ref{prop:attachingManifoldsFaceUInterior}.

\begin{figure}[H]
	\centering
	\caption{Attaching two copies of $\D^2\times\I$ together, annulus to annulus.}
	\includegraphics[height=3in]{figures/dummy.jpg}
	\label{fig:DxIuDxI}
\end{figure}

The following examples illustrate the changes that occur in the boundary of a 4--manifold with faces when we perform the cornered version of a handle attachment.
Such operations are used extensively in Section~\ref{sec:3bound4}.

\begin{ex}
	\label{ex:DxIuDxI}
	Let $X$ and $Y$ each be copies of $\DD\times\I$.
	Include the face $F=S^1\times\I$ into $X$ as
	\[
	x:F= S^1\times\I\into S^1\times\I=\pd \DD \times\I\subset \pd X,
	\]
	with a similar map $y$ including $F$ into $Y$.
	We immediately deduce that the adjunction $X\cup_F Y$ is exactly $S^2\times\I$.
	See Figure~\ref{fig:DxIuDxI}
\end{ex}

\begin{ex}
	\label{ex:AxIuDxI}
	Let $X=S^1\times\I\times\I$ and $Y=\DD\times\I$.
	Include the face $F=S^1\times\I$ into $X$ as
	\[
		x:F= S^1\times\I \into S^1\times\I=S^1 \times\I\times\{1\}\subset \pd X,
	\]
	and into $Y$ as
	\[
		y:F= S^1\times\I\into S^1\times\I=\pd \DD \times\I\subset \pd Y.
	\]
	The adjunction $X\cup_F Y$ forms $\DD\times\I$.
	See Figure~\ref{fig:AxIuDxI}
\end{ex}

\begin{figure}[H]
	\centering
	\caption{Attaching $S^1\times\I\times\I$ to $\D^2\times\I$.}
	\includegraphics[height=3in]{figures/dummy.jpg}
	\label{fig:AxIuDxI}
\end{figure}

\begin{ex}
	\label{ex:PxIuDxI}
	Let $P$ be a pants surface and take $X=P\times\I$ with faces $P\times\{\pm 1\}$ and $\pd P\times\I$ (three copies of $S^1\times \I$).
	Let $Y=\DD\times\I$.
	
	Include the face $F=S^1\times\I$ into $X$ over one of its annular boundary components and into $Y$ over its only annular boundary component.
	Then the adjunction $X\cup_F Y$ is $ S^1\times\I\times\I $.
	See Figure~\ref{fig:PxIuDxI}
\end{ex}

\begin{figure}[H]
	\centering
	\caption{Attaching a pair of pants crossed with an interval to $\D^2\times\I$.}
	\includegraphics[height=3in]{figures/dummy.jpg}
	\label{fig:PxIuDxI}
\end{figure}

The pattern established continues when the cylindrical blocks fill in 3--manifolds over annular boundary components.
We examine two more specific cases of this that are relevant in Section~\ref{sec:3bound4}.

\begin{ex}
	\label{ex:interactivefiber1uDxI}
	Let $H$ be the 3--manifold with faces illustrated in Figure~\ref{fig:interactivefiber1uDxI}.
	Take 8 copies of the cylindrical block $\DD\times\I$, one for each highlighted annulus in Figure~\ref{fig:interactivefiber1uDxI}, and include a copy of the annular face $S^1\times\I$ into the boundary of each cylindrical block and into the corresponding annular boundary component of $H$.
	The resulting adjunction has boundary consisting of 6 copies of $S^2$.
\end{ex}

\begin{figure}[H]
	\centering
	\caption{Attaching cylindrical blocks over annular boundary components.
	The 3--manifold with faces is illustrated by its boundary which is embedded in $S^3$, realized as the one point compactification of $ \R^3 $.
	Each face is either an annulus or a pair of pants.
	The 3--manifold is the solid containing the point at infinity, and cylindrical blocks are attached over the highlighted annuli.}
	\includegraphics[height=3in]{figures/dummy.jpg}
	\label{fig:interactivefiber1uDxI}
\end{figure}

\begin{ex}
	\label{ex:interactivefiber2uDxI}
	Let $H$ be the 3--manifold with faces illustrated in Figure~\ref{fig:interactivefiber2uDxI}.
	Take 6 copies of the cylindrical block $\DD\times\I$, one for each highlighted annulus in Figure~\ref{fig:interactivefiber2uDxI}, and include a copy of the annular face $S^1\times\I$ into the boundary of each cylindrical block and into the corresponding annular boundary component of $H$.
	The resulting adjunction has boundary consisting of 4 copies of $S^2$.
\end{ex}

\begin{figure}[H]
	\centering
	\caption{Attaching cylindrical blocks over annular boundary components.
		The 3--manifold with faces is illustrated by its boundary which is embedded in $S^3$, realized as the one point compactification of $ \R^3 $.
		Each face is either an annulus or a pair of pants.
		The 3--manifold is the solid containing the point at infinity, and cylindrical blocks are attached over the highlighted annuli.}
	\includegraphics[height=3in]{figures/dummy.jpg}
	\label{fig:interactivefiber2uDxI}
\end{figure}

\begin{ex}
	Let $W$ be a 4--manifold with boundary $M$, itself a 3--manifold  (i.e.\ $ \pd_j W =\emptyset$ for $j>1$).
	Suppose $M$ has a decomposition $M=\cup_i H_i$ into a collection of submanifolds $H_i$ with faces such that if $H_i\cap H_j\neq\emptyset$, $i\neq j$, then $H_i\cap H_j$ is a face of each.
	The $H_i$ are \emph{decomposing blocks} of $M$.
	
	Suppose a decomposing block $H_i$ of $M$ is diffeomorphic to a regular $n$--gon $G_n$ crossed with $S^1$, and suppose $H_j$ is a decomposing block such that $H_i\cap H_j$ is an annular face of each.
	Let $D=G_n\times\DD$ be a 4--manifold with faces, and consider the inclusion maps
	\[
		\iota_W:G_n\times S^1\into H_i\subset \pd W 
	\]
	and
	\[
		\iota_D:G_n\times S^1\into G_n\times S^1\subset \pd D.
	\]
	Then $\iota_D(G_n\times S^1)$ is a face of $D$ and $\iota_W(G_n\times S^1)$ is interior to a face of $W$, so we form the adjunction $W'=W\cup_{G_n\times S^1} D$.
	By Proposition~\ref{prop:attachingManifoldsFaceUInterior}, the result is a manifold with faces, and the $j$--boundaries of $W'$ is well--defined for each $j$.
	
	For $j=1$, the $j$--boundary of $W'$ is decomposed into $M\setminus H_i$ and $g_i\times \DD$ for each $i$, where the $g_i$ are the faces of $G_n$.
	i.e.\ the $1$--boundary of $W'$ is $M\setminus H_i$ along with a collection of cylinders, one per face of $G_n$.
\end{ex}
