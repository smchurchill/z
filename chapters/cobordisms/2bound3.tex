\label{sec:2bound3}

%%%%%%%%%%%%%%%%%%%%%%%%%%%%%%%%%%%%%%%%%%%%%%%%%%%%%%%%%%%%%%%%%%%%
%	This section should be split into some lemmas followed by the theorem.
%	The structure of the lemmas will say
%		1.  The surface can be decomposed
%		2.  We can add 1--handles that give a certain structure
%		3.  We can add 2--handles that give a certain structure
%		
%%%%%%%%%%%%%%%%%%%%%%%%%%%%%%%%%%%%%%%%%%%%%%%%%%%%%%%%%%%%%%%%%%%%


We now have the necessary framework to demonstrate that a closed, oriented 2--manifold bounds a 3--manifold through an explicit construction.
This is useful as a jumping off point for the case one dimension higher which draws on the same framework and methods.

Let $\Sigma$ be a closed, oriented 2--manifold.
Our construction of a 3--manifold $M$ for which $\pd M=\Sigma$ begins by letting $f:\Sigma\to\II$ be a Morse function.
We take as our base for a 3--manifold $M=\Sigma\times\II$.
Then $M$ has two boundary components $\Sigma^-$ and $\Sigma^+$ corresponding to $\Sigma\times\{-1\}$ and $\Sigma\times\{+1\}$ respectively.
Our goal is to attach handles to $M$ over $\Sigma^+$ until that boundary component is reduced to $\emptyset$.
The construction follows the general structure of
\begin{enumerate}
	\item Decompose $\Sigma^+$ via our Morse function.
	\item Some of the pieces into which $\Sigma^+$ is decomposed serve as attaching regions in $M$ for 2--handles.
	\item The attachment of 2--handles alters $\Sigma^+$ in a controlled way, and yields attaching regions for 3--handles.  Attaching 3--handles over these regions empties $\Sigma^+$.
\end{enumerate}

Before detailing our decomposition, we define the pieces into which our surface will decomposed.

\begin{defn}
	\label{def:annulus}
	An \emph{annulus} is any surface homeomorphic to $S^1\times\I$.
	An annulus's boundary consists of two components that are homeomorphic to $S^1$.
	Another description of an annulus is as a 2--disc minus an open ball.
\end{defn}

\begin{defn}
	\label{def:pants}
	A \emph{pair of pants} or \emph{pants surface} is a surface that is homeomorphic to a 2--disc minus two disjoint open balls.
	The boundary of a pair of pants consists of three components that are homeomorphic to $S^1$.
\end{defn}

\begin{lem}
	\label{lem:sigmadecomp}
	Let $\Sigma$ be a closed, oriented 2--manifold and let $f:\Sigma\to\II$ be a Morse function with distinct critical values.
	Then $f$ induces a decomposition of $\Sigma$ as
	\[
		\Sigma=\MA\cup \MD\cup \MP
	\]
	where $\MA$ is a collection of annuli, $\MD$ is a collection of 2--discs, and $\MP$ is a collection of pants.
	
	Furthermore, if $X$ and $Y$ are components of this decomposition and $X\cap Y\neq\emptyset$, then $X\cap Y$ is a boundary circle of each of $X$, $Y$, and at least one of $X$, $Y$ is an annulus.
\end{lem}

\begin{proof}
	Let $x_1<\dots<x_n$ be the critical values of $f$ and let $t_i^+$, $t_i^-$ be real numbers such that
	\[
		\begin{array}{ccc}
			t_i^-<x_i<t_i^+ & \textrm{ and } & t_i^+<t_{i+1}^-. 		
		\end{array}
	\]
	The $t_i^\pm$ decompose the range of $f$ into the intervals $X_i=[t_i^-,t_i^+]$ for $i\in\{1,\dots,n\}$ and $T_i=[t_i^+,t_{i+1}^-]$ for $i\in\{1\dots, n-1\}$.
	For each $i$, the interval $X_i$ contains the critical value $x_i$ and $T_i$ consists entirely of regular values.
	We decompose of $\Sigma$ into a collection whose elements are the connected components of $f\inv(X_i)$ and $f\inv(T_i)$ (See Figure~\ref{fig:sigmadecomp}).
	
	\begin{figure}
		\centering
		\caption{Decomposing a surface by a Morse function}
		\includegraphics[height=3in]{figures/dummy.jpg}
		\label{fig:sigmadecomp}
	\end{figure}
		
	First, we investigate the $T_i$.
	No $T_i$ contains a critical value of $f$, so $f\inv(a)$ for any $a$ in $T_i$ is a 1--dimensional submanifold of $\Sigma$, i.e.\ a disjoint collection of copies of $S^1$.
	Let $a\in\inter{T_i}$.
	Then $f\inv (a)$ is a closed submanifold of the closed manifold $\Sigma$, hence $f\inv(a)$ has a closed tubular neighbourhood in $\Sigma$ diffeomorphic to $f\inv (a)\times \I$, i.e.\ a disjoint collection of annuli.
	Because $f\inv(T_i)$ is diffeomorphic to, and deformation retracts onto, $f\inv[a-\varepsilon,a+\varepsilon]$ for any $\varepsilon>0$ by Theorem~\ref{thm:morseretract}, we conclude that $f\inv(T_i)$ is diffeomorphic to a disjoint collection of annuli -- one component for each circle in $f\inv(a)$.
	
	Next, each $X_i$ contains exactly one critical value at $x_i$ corresponding to the critical point $p_i$ in $\Sigma$.
	Call the $m$ connected components of $f\inv(X_i)$ that do not contain $p_i$ by $\Sigma_{i,k}$ for $k\in\{1\dots m\}$ and call the connected component of $f\inv(X_i)$ that contains $p_i$ by $\Sigma_{i,p}$.
	
	Consider $f$ restricted to $\Sigma\setminus\Sigma_{i,p}$.
	This restriction has only regular values in $X_i$, so each of $\Sigma_{i,k}$ is an annulus by the previous argument.
	
	For the components $\Sigma_{i,p}$, we check the index $\lambda_i$ of $p_i$.
	When $\lambda_i$ is 0 or 2, Lemma~\ref{lem:morselemma} tells us that $\Sigma_{i,p}$ is diffeomorphic to a 2--disc.
	When $\lambda_i=1$, we use Lemma~\ref{lem:morselemma} and the orientability of $\Sigma$ to say that $\Sigma_p$ diffeomorphic to a pair of pants.
	
%	Use Saeki p.14 as reference if necessary
	
	We now have $\Sigma$ decomposed into the connected components of $f\inv(T_i)$ and $f\inv(X_i)$, and each decomposing piece is either a 2--disc, an annulus, or a pair of pants.
	By design, a pair of decomposing pieces will either intersect along a circle in some $f\inv(t_i^\pm)$ or not at all.
	Further, if a decomposing piece is a 2--disc or pair of pants, then it is a connected component of $f\inv(X_i)$ for some $i$.
	Because the $X_i$ are separated by $T_i$, so are the $f\inv(X_i)$ by $f\inv(T_i)$.	
\end{proof}

With $\Sigma$ decomposed, we start attaching handles.
In the notation used in the of the proof of Lemma~\ref{lem:sigmadecomp}, we attach 2--handles over the $f\inv(T_i)$.

\begin{lem}
	\label{lem:sigma2handles}
	Let $\Sigma$ be a closed, oriented 2--manifold and let $f:\Sigma\to\II$ be a Morse function with distinct critical values.
	Let $M=\Sigma\times\I$ be a 3--manifold with boundary and consider the boundary component $\Sigma\times\{1\}=\Sigma^+$.
	Then it is possible to attach 2--handles to $M$ over submanifolds of $\Sigma^+$	such that $M\cup\{\textrm{2--handles}\}$ is a 3--manifold with faces whose boundary consists of $\Sigma\times\{-1\}$ and a disjoint collection of boundary components that are homeomorphic to 2--spheres.
\end{lem}

\begin{proof}
	By Lemma~\ref{lem:sigmadecomp}, there is a decomposition of $\Sigma^+$ into a collection of 2--discs, annuli, and pants.
	This decomposition was obtained by first decomposing $f(\Sigma)$ into intervals $T_i$ and $X_i$, where each $X_i$ contained exactly one critical value, the $T_i$ consisted entirely of regular values, the $T_i$ separated the $X_i$, and vice versa.
	
	Consider the annuli in this decomposition that map only over regular values through $f$.
	In other words, the $f\inv(T_i)$.
	Because the $T_i$ are disjoint, none of these annuli intersect one another.
	We attach a 2--handle to $M$ over each of these annuli.
	
	A 3--dimensional 2--handle is a copy of $\D^2\times\I$ attached over the $\pd(D^2)\times\I$ annular boundary component.
	By Proposition~\ref{prop:handle}, these handle attachments alter $\Sigma^+$ by replacing the annuli $f\inv(T_i)$ with 2--discs at each circular boundary component.
	This is equivalent to attaching a 2--disc to $\Sigma^+\setminus(\cup_i f\inv(T_i))$ over each boundary component.
	
	This leaves $M\cup\{\textrm{2--handles}\}$ with boundary consisting of $\Sigma\times\{-1\}$ plus a disjoint collection of 2--discs, annuli, and pants with 2--discs attached over their boundary components.	
\end{proof}

\begin{theorem}
	\label{thm:2bound3}
	Let $\Sigma$ be a closed, oriented 2--manifold and let $f:\Sigma\to\II$ be a Morse function with distinct critical values.
	Then there exists a cobordism of the pair $(\Sigma,\emptyset)$.
\end{theorem}

\begin{proof}
	We begin the cobordism as $M=\Sigma\times\II$ with boundary components $\Sigma\times\{1\}=\Sigma^+$ and $\Sigma\times\{-1\}=\Sigma^-$.
	By Lemmas~\ref{lem:sigmadecomp} and ~\ref{lem:sigma2handles}, we attach 2--handles to $M$ on the $\Sigma^+$ boundary component to produce a 3--manifold $M'$ with faces consisting of $\Sigma^-$ plus a collection of 2--manifolds homeomorphic to $S^2$.
	
	Attach a 3--handle over each boundary component homeomorphic to $S^2$.
	The result is a 3--manifold with boundary exactly $\Sigma^-$, i.e.\ a cobordism of $(\Sigma,\emptyset)$.
\end{proof}

\begin{lem}
	\label{lem:complexcorresp}
	Let $\Sigma$ be a closed, oriented 2--manifold and let $f:\Sigma\to\II$ be a Morse function with distinct critical values and let $M$ be a cobordism of the pair $(\Sigma,\emptyset)$ induced by $f$ via Theorem~\ref{thm:2bound3}.
	
	The Stein complex of $f$ can be given the structure of a 1--dimensional CW--complex $C$.
	Furthermore, there is a one--to--one correspondence between the edges of $C$ and the 2--handles of $M$ and between the vertices of $C$ and the 3--handles of $M$.
\end{lem}

\begin{proof}
	Let $\Sigma\overset{h}{\to}S\overset{g}{\to}\II$ be the Stein factorization of $f$ and let $x_1<\cdots<x_n$ be the critical values of $f$.

	Every element $s$ of $S$ such that $g(s)=x_i$ for some $i$ is considered a vertex of $C$.
	If $u$ and $v$ are vertices of $C$ such that $g(u)$ and $g(v)$ are consecutive critical values of $f$, then $f\inv (g(u),g(v))$ is a collection of open annuli in $Sigma$, and $h\comp f\inv (g(u),g(v))$ is a collection of arcs, copies of $(g(u),g(v))$, embedded in $S$.
	These arcs form the edges of $C$.
	
	We turn our attention to a one--to--one correspondence $c$ between the edges of $C$ and the 2--handles of $M$ and between the vertices of $C$ and the 3--handles of $M$.
	Consider the decomposition of $\Sigma$ induced by the real numbers $t_i^\pm$ in Lemma~\ref{lem:sigmadecomp}.
	Each interval $T_i$ induced the attachment of some number of 2--handles: one 2--handle per connected component of $f\inv T_i$.
	If $T_{i,k}$ is a connected component of $f\inv T_i$, then we call the 2--handle induced by $T_{i,k}$ by $H_{i,k}^2$.
	We built $C$ such that $h(T_{i,k})$ is contained entirely inside of a edge $e_{i,k}$ of $C$, so we define $c(e_{i,k})=H_{i,k}^2$.
	
	Each interval $X_i$ from Lemma~\ref{lem:sigmadecomp} induced the attachment of some number of 3--handles: one 3--handle per connected component of $f\in X_i$.
	After we attached 2--handles in Lemma~\ref{lem:sigma2handles}, the boundary of $\Sigma\times\II$ consisted of $\Sigma^-$ along with a collection of attaching regions for 3--handles.
	These attaching regions were exactly the connected components of $f\inv X_i$ with 2--discs attached,over their boundary components, 2--discs introduced by the attachment of 2--handles.
	If $X_{i,k}$ is a connected component of $f\inv X_i$, then we call the 3--handle attached over $X_{i,k}$ by $H_{i,k}^3$.
	We built $C$ such that $h(X_{i,k})$ contains exactly one vertex $v_{i,k}$ of $C$, so we define $c(v_{i,k})=H_{i,k}^3$.
	Thus, $c$ is a one--to--one correspondence between the edges of $C$ and the 2--handles of $M$ and between the vertices of $C$ and the 3--handles of $M$.
\end{proof}

\begin{figure}
	\caption{A typical Morse function $T^2\# T^2\to\R$, its Stein factorization, and corresponding dual handle decomposition.}
	\centering
	\includegraphics[height=4in]{figures/typicalmorse.jpg}
	\label{fig:typicalmorse}
\end{figure}

\begin{theorem}
	\label{thm:2stein3}
	Let $\Sigma$ be a closed, oriented 2--manifold, let $f:\Sigma\to\II$ be a Morse function with distinct critical values, and let $\Sigma\overset{h}{\to}S\overset{g}{\to}\II$ be the Stein factorization of $f$.
	
	Then the CW--complex structure $C$ of $S$ can be interpreted as a set of instructions that constructs a handle decomposition for a cobordism of the pair $(\Sigma,\emptyset)$.
	Furthermore, this handle decomposition is dual to the decomposition produced by Theorem~\ref{thm:2bound3}.
	Consequentially, this handle decomposition contains handles of index at most 1. 
\end{theorem}

\begin{proof}
	To begin, we let $M^*$ be an empty handle decomposition.
	In each step, we add some number of handles to $M^*$ and let $M^*=M^*\cup\{\textrm{attached handles}\}$.
	The instructions follow.
	\begin{enumerate}
		\item
			Let $M^*$ be an empty handle decomposition.
		
		\item
			Let $T$ be a spanning tree of $C$.
						
		\item
			For each vertex $v$ of $T$, attach a 0--handle $H_v^0$ to $M^*$.
			This is equivalent to taking the disjoint union of $M^*$ with a 3--disc.
						
		\item
			For each edge $(u,v)$ of $T$, attach a 1--handle $H_{u,v}^1$ to $M^*$ over the attaching region that consists of a disjoint pair of 2--discs, one contained in the boundary of $H_u^0$ and one in the boundary of $H_v^0$.
			
		\item
			For each remaining edge $(u,v)$ of $C$, attach a 1--handle $H_{u,v}^1$ to $M^*$ over the attaching region that consists of a disjoint pair of 2--discs, one contained in the boundary of $H_u^0$ and one in the boundary of $H_v^0$, in such a way that the resulting 3--manifold is orientable.
	\end{enumerate}
	Each vertex of $C$ induces a 0--handle of $M^*$, and each edge of $C$ induces a 1--handle of $M^*$.

	To see that this handle decomposition is dual to that produced in Theorem~\ref{thm:2bound3}, we embed $C$ in $M$ and in $M^*$ via the maps $\phi$, $\phi*$, defined as follows.
	Let $c$ be the one--to--one correspondence between the edges of $C$ and the 2--handles of $M$ and between the vertices of $C$ and the 3--handles of $M$ defined in Theorem~\ref{lem:complexcorresp}.
	
	For a vertex $v$ of $C$, define $\phi^*(v)$ to be the core of the 0--handle $H_v^0$ and define $\phi(v)$ to be the cocore of the 3--handle $c(v)$.
	For an edge $e=(u,v)$ of $C$, let $x\in e$ be parametrized by $x(t)=(1-t)u+tv$, $t\in(0,1)$.
	Then we define $\phi^*(t)$ piecewise by demanding that
	\begin{enumerate}
		\item
			$\phi^*$ takes $\sqb{1/3,2/3}$ to the core of the 1--handle $H_{u,v}$,
		\item
			$\phi^*$ takes $[\,2/3,1)$ to the line segment connecting the centre of the attaching disc contained in the boundary of  $H_v^0$ to the centre of $H_v^0$, and
		\item
			$\phi^*$ takes $(0,1/3\,]$ to the line segment connecting the centre of the attaching disc contained in the boundary of $H_u^0$ to the centre of $H_u^0$.
	\end{enumerate}
	We define $\phi$ similarly, taking the middle third of $e$ to the cocore of $c(e)$ and connecting the cocore of $c(e)$ to the cocores of $c(u)$ and $c(v)$.
	
	This shows that the cores of the $k$--handles in $M^*$ are the cocores of the $(3-k)$--handles of $M$.
	Thus, the handle decompositions describe the same 3--manifold, and $M^*$ is a cobordism of the pair $(\Sigma,\emptyset)$.		
\end{proof}



%
%\begin{proof}
%	Figure \ref{fig:typicalmorse} depicts a typical Morse function acting on the closed oriented surface of genus 2.
%	We begin by examining the preimages of points $x\in\R$.
%	Denote the space $f\inv(x)$ by $\Sigma_x$, a small interval around $x$ by $\varepsilon(x)=[\,x-\varepsilon,x+\varepsilon\,])$, and the neighbourhood in $\Sigma$ around $\Sigma_x$ by $f\inv(\varepsilon(x))=\varepsilon(\Sigma_x)$.
%	The spaces $\Sigma_x$ and $\varepsilon(\Sigma_x)$ may not be connected, so we index the connected components by superscript.
%	Let $p$ be a critical point of $f$ with critical value $f(p)=x$.
%	By our assumption that critical values are distinct, $p$ is the only critical point of $f$ in $\Sigma_x$.
%	The connected component of $\Sigma_x$ containing $p$ is called the \emph{singular fiber} at $p$ and is denoted $\Sigma_x^p$.
%	The connected component of $\varepsilon(\Sigma_x)$ containing $p$ is called the \emph{critical neighbourhood} at $p$ and is denoted $\varepsilon^p(\Sigma_x)$.
%	
%	By the regular value theorem, any regular value pulls back through $f\inv$ to a disjoint collection of circles in $\Sigma$ called \emph{regular fibers}.
%	Likewise, a neighbourhood $\varepsilon(x)$ that contains no critical values of $f$ pulls back to a disjoint collection of annuli in $\Sigma$.
%	The restriction of $f$ to the components of $\Sigma_x$ that do not contain a critical point has $x$ as a regular value, so the remaining connected components of $\Sigma_x$ are all copies of $S^1$ that we index by $\Sigma_x^i$ for $i=1,\dots,k$, and their associated neighbourhoods are the annuli $\varepsilon^i(\Sigma_x)$.
%	When referring to an arbitrary connected subspace of $\Sigma_x$ that could be the singular fiber $\Sigma_x^p$ or a regular fiber $\Sigma_x^i$, we will use the notation $\Sigma_x^*$.
%	
%	The shape of a singular fiber $\Sigma_x^p$ is determined by the index of $p$.
%	Because $f$ is a Morse function whose domain is a surface, its critical points are easily classified by the Morse lemma (Lemma \ref{lem:morselemma}).
%	Locally in $\Sigma$, a critical point of index 0 is a minimum, of index 1 a saddle, and of index 2 a maximum.
%	
%	Let $x$ be a critical value for the critical point $p$ and take $\varepsilon$ to be small enough that $\varepsilon(x)$ contains no critical values other than $x$.
%	When $p$ is of index 0 or 2, we can immediately deduce that $\varepsilon^p(\Sigma_x)$ is diffeomorphic to a disc.
%	When $p$ is of index 1, the Morse lemma tells us that $\Sigma$ looks like a standard saddle near $p$.
%	The intersection of $\Sigma_x^p$ with this saddle is a cross whose centre is $p$.
%	For $y\in\varepsilon(x)$, $y$ is a regular value whose preimage is a disjoint union of circles.
%	The circles above and below the saddle singularity are the result of smoothing out the cross into a pair of oriented arcs, done in two possible ways.
%	The orientations of these circles orient the cross, which has two incoming arms and two outgoing arms which appear in alternating order.
%	A Morse function has distinct singular fibers, so the cross we know about in $\Sigma_x^p$ must have its arms connected in $\Sigma_x^p$ through nonsingular orientation--preserving arcs.
%	We can then see that $\Sigma_x^p$ is a figure 8, and $\varepsilon^p(\Sigma_x)$ is a pair of pants in $\Sigma$.
%	
%
%	\begin{figure}
%		\centering
%		\caption{Smoothing a cross into a saddle}
%		\includegraphics[height=3in]{figures/smoothcross.jpg}
%		\label{fig:smoothcross}
%	\end{figure}
%	
%	This analysis is sufficient to form a handle decomposition for a cobordism of $(\Sigma,\emptyset)$.
%	We begin with the 3--manifold $\Sigma\times\I$ whose boundary components $\Sigma_0=\Sigma\times\{0\}$ and $\Sigma_1=\Sigma\times\{1\}$ are copies of $\Sigma$, and let $f:\Sigma_0\to\I$ be a Morse function with distinct critical values $x_i$.
%	The general idea is that we attach handles to $\Sigma_0$, altering that boundary component until it is empty.
%	The $x_i$ partition $\I$ into open intervals containing only regular values.
%	We take the associated regular annuli to be attaching regions for 3--dimensional 2--handles, and then fill in what remains with 3--dimensional 3--handles.
%	
%	For an interval $(x_i,x_{i+1})$, consider the subinterval $\varepsilon(t_i)$ where $x_i<t_i<x_{i+1}$ and $\varepsilon$ is small enough that $\varepsilon(t_i)$ contains neither $x_i$ nor $x_{i+1}$.
%	Take the associated regular annuli $\varepsilon(\Sigma_{t_i})$ to be the attaching regions for 3--dimensional 2--handles.
%	The boundary component that was once $\Sigma_0$ is now the result of removing the interiors of the annuli $\varepsilon(\Sigma_{t_i})$ from $\Sigma_0$ for each $t_i$, and then introducting discs parallel to the cores of the attached 2--handles.
%	These discs are seen as $\DD\times\{0\}$ and $\DD\times\{1\}$ inside of a 2--handle $\DD\times\D^1$.
%	The boundary circles of the discs are $\Sigma_{t_i-\varepsilon}$ and $\Sigma_{t_i+\varepsilon}$ for each $i$.
%	
%	There are now intervals about each critical point that we know, from our analysis above, pull back to discs, annuli, and pairs of pants.
%	Each of these subsurfaces have circle boundaries that correspond to the regular fibers $\Sigma_{t_i-\varepsilon}$ and $\Sigma_{t_i+\varepsilon}$ which were capped with discs in the previous step.
%	The altered boundary described before is therefore a collection of copies of $S^2$, which we take to be the attaching regions of 3--dimensional 3--handles.
%	
%	With these handles attached, we obtain $(\Sigma\times\I)\cup\{\textrm{2--handles}\}\cup\{\textrm{3--handles}\}$, which is a 3--manifold whose only boundary component is $\Sigma_1$, i.e. is a cobordism of the pair $(\Sigma,\emptyset)$.
%	
%	The corresponding dual handle decomposition is realized by turning the process upside down.
%	Where we previously had 3--handle attachments with cores $\D^3\times\{\vec{0}\}$ and cocores $\{\vec{0}\}\times\{\vec{0}\}$, we now have 0--handles with cores $\{\vec{0}\}\times\{\vec{0}\}$ and cocores $\D^3\times\{\vec{0}\}$.
%	In other words, the dual construction begins by taking the disjoint union of a collection of 3--discs that correspond to the space obtained by capping the boundary circles of the components $\varepsilon(\Sigma_{x_i})$ with 2--discs and then filling the spheres with 3--balls.
%	Where we previously had 2--handle attachments with cores $\DD\times\{\vec{0}\}$ and cocores $\{\vec{0}\}\times\D^1$, we now have 1--handle attachments with cores $\{\vec{0}\}\times\D^1$ and cocores $\DD\times\{\vec{0}\}$.
%	In other words, we connect the 0--handles together using 1--handles.
%	The old belt 0--spheres of the 2--handles in the previous construction correspond here to new attaching 0--spheres, and the attaching maps are chosen to preserve orientability cf.\ Remark~\ref{rmk:1handle}.
%	The new attaching 0--spheres bound the new cores, the old core 2--discs are now cocores.
%	This construction yields a 3--manifold whose boundary is exactly $\Sigma$, and is indeed another cobordism of the pair $(\Sigma,\emptyset)$.
%	
%	A Stein factorization $f=g\comp h$ is simple to describe.
%	Define the equivalence relation $\sim$ on the points in $\Sigma$ by putting $p\sim q$ if and only if $f(p)=f(q)=x$ and $p$ and $q$ are in the same subspace $\Sigma_x^*$.
%	Then $h$ is the quotient map $\Sigma\to \Sigma/\!\!\sim$ where the points of $\Sigma/\!\!\sim$ are the subspaces $\Sigma_x^*$, and $g$ is the map $\Sigma/\!\!\sim\, \to\R$ defined by $g(\Sigma_x^*)=x$.
%
%	The Stein complex $S=\Sigma/\!\!\sim$ can be viewed as a graph $G$.
%	A critical value $x=f(p)$ has as its preimage one associated singular fiber $\Sigma_x^p$ in $S$ plus a number of copies of $S^1$ (possibly 0) given by $\Sigma_x^i$.
%	We take the associated points $h(\Sigma_x)$ in $S$ for each critical value to be the vertex set $v(G)$.
%	For a pair of adjacent critical values $x_{j}$, $x_{j+1}$, an appropriate choice of $\varepsilon$ and $x\in(x_{j},x_{j+1})$ yields a collection of regular annuli $\varepsilon(\Sigma_x)$ in $\Sigma$ that has boundary inside of $\Sigma_{x_{j}}\cup\Sigma_{x_{j+1}}$.
%	In $S$, we find $h(\varepsilon(\Sigma_x))$ to consist of 1--dimensional strands that connect components of $\Sigma_{x_{j}}$ and $\Sigma_{x_{j+1}}$.
%	The set of pairs $(v,w)$ of connected subspaces with $v\in\Sigma_{x_{j}}$ and $w\in\Sigma_{x_{j+1}}$ such that $v$ and $w$ are connected by a strand in $S$ forms the edge set $e(G)$.
%	
%	At this point, $G$ corresponds exactly to the dual handle decomposition given earlier.
%	For each vertex of $G$ we get a 0--handle.
%	For each edge, a 1--handle.
%	We attach 1--handles with only one demand: that the resulting space continues to be orientable.
%\end{proof}
%
%The Stein complex obtained in the proof of Theorem~\ref{thm:2bound3} may have superfluous vertices.
%In particular, a vertex $v$ of $G$ that is adjacent to exactly two verticex $u$ and $w$ via the edges $(u,v)$ and $(v,w)$ may be replaced, along with its adjacent edges, by a single edge $(u,w)$.
%To see that the handle decomposition described by this Stein complex graph is equivalent, examine the attaching region of the 1--handle corresponding to $(u,v)$ as it sits inside the boundary of the 0--handle corresponding to $v$.
%This region may be isotoped over the 1--handle corresponding to $(v,w)$, where it ends up in the boundary of the 0--handle corresponding to $w$.
%We are left with a 0--handle $v$ and 1--handle $(v,w)$ that are contractible into $w$.