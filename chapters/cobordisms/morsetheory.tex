\label{sec:morsetheory}

Our primary use of Morse theory is to define a pair of related constructions.
The first is the \emph{handle decomposition} of a manifold, which comes with a dual decomposition.
The second is the \emph{Stein complex} of a Morse function on a manifold.
The main results from this chapter connect a Stein complex with a handle decomposition, and this construction is extended to triangulations in the remaining chapters.

\begin{defn}
  A generic smooth function $f:M\to\R$ where $M$ is an $n$--manifold is called a \emph{Morse function}.
  For a critical point $p\in M$ of $f$, the \emph{index} of $p$ is the dimension of the largest subspace of $T_p M$ such that $d^2f_p$ is negative definite.
  Less formally, this corresponds to the number of independent directions in $M$ along which $f$ decreases.
  We denote by $M^a$ the subspace $f\inv(-\infty,a\,]$ where $f$ is a fixed Morse function on $M$.
\end{defn}

A key question of Morse theory regards how the topology of $M^a$ changes as $a$ passes the critical values of $f$.
This question is answered by the following two theorems.

\begin{theorem}
  \label{thm:morseretract}
  Let $f:M\to\R$ be a Morse function with no critical values in $(a,b\,]$, $a<b$.
  If $f\inv[\,a,b\,]$ is compact, then $M^a$ and $M^b$ are diffeomorphic, and $M^b$ deformation retracts onto $M^a$.
\end{theorem}

\begin{theorem}
  \label{thm:morsehandle}
  Let $f:M\to\R$ be a Morse function.
  Let $p$ be a critical point of $f$ of index $\lambda$ with associated critical value $f(p)=q$.
  If $f\inv[\,q-\varepsilon, q+\varepsilon\,]$ is compact and contains no critical points other than $p$, then $M^{q+\varepsilon}$ is diffeomorphic to $M^{q-\varepsilon}\cup_\varphi H^\lambda$ for some attaching map $\varphi:S^{\lambda-1}\times\D^\mu\to \pd(M^{q-\varepsilon})$.
\end{theorem}

\begin{defn}
  Let $f:M\to\R$ be a Morse function on a closed $n$--manifold $M$ with critical points $\{p_1,\dots,p_k\}$ of indices $\{\lambda_1,\dots,\lambda_k\}$ and a set $\{t_0,\dots,t_k\}$ of numbers in $\R$ such that
  \[
	0\leq t_0 < f(p_1) < t_1 < f(p_2) < \cdots < t_{k-1} < f(p_k) < t_k \leq 1,
  \]
  and
  \[
	  \lambda_1 \geq \lambda_2 \geq \cdots \geq \lambda_k.
  \]
  The previous theorems guarantee that, for each $i\geq 1$, $f\inv[\,t_{i-1},t_i\,]$ is diffeomorphic to $(f\inv(t_{i-1})\times\I)\cup H^{\lambda_i}$, giving us a realization of $M$ as
  \[
	  \emptyset = M_0 \subset M_1 \subset M_2 \subset \cdots \subset M_{k-1} \subset M_k = M
  \]
  where $M_i$ is obtained from $M_{i-1}$ by attaching a $\lambda_i$--handle.
  Such a realization is called a \emph{handle decomposition} of $M$.
\end{defn}

A Morse function $f$ satisfying the requirements to define a handle decomposition actually defines two handle decompositions --- one from $f$ and one from $1-f$.
The handle decompositions from $f$ and $1-f$ are dual in the sense that every $\lambda$--handle attached in the making of the decomposition from $f$ has a corresponding $\mu$--handle in the decomposition from $1-f$.
This handle is realized by reversing the roles of the core and co--core in a given $\lambda$--handle $H^\lambda$ to obtain a $\mu$--handle $H^\mu$.

Similar results can be found for manifolds with boundary.
Let $W$ be an $n$--manifold with boundary $\pd W = M_0\sqcup M_1$, where both $M_0$ and $M_1$ are compact.
The Morse functions on $W$ that extend Theorems \ref{thm:morseretract} and \ref{thm:morsehandle} are those that fix $f(M_0)=0$, $f(M_1)=1$.
Now, the handle decomposition of $W$ is built on top of the component $M_1\times\{1\}$ of $M_0\times\I$, with $M_0\times\{0\}$ corresponding to the boundary component $M_0$ in the finished product.

\begin{defn}
  \label{def:morse2function}
  A generic smooth function $f:M\to\RR$ where $M$ is an $n$--manifold is called a \emph{Morse 2--function}.
\end{defn}

\begin{defn}
  \label{def:steinfactorization}
\end{defn}

\begin{defn}
  \label{def:steincomplex}
\end{defn}
