
In the case that $G$ is of genus 1, 

$G$ contains no critical points of $f$, and we have already seen how to give $G$ the structure of a circle bundle over a disc whose fibers are the preimages through $f$ of points in $\nbhd{x}$.
Here, $h(G)$ is a 2--disc in a region of $S$, and fills a hole 
The handlebody of genus 1 collapses through $h$ to fill a hole in a regular level, and in this way we cap off all $S^1$ boundary regions of the Stein complex with 2--discs.

and is of genus 0 or 2 when the folds over $x$ are not interactive.


\begin{figure}
	\centering
	\captionsetup{justification=centering}
	\caption{Stein complex of uninteractive and interactive definite folds projecting near codimension 2 critical values}
	\includegraphics[height=3cm]{figures/dummy.jpg}
	\label{fig:codim2steindef}
\end{figure}

The genus 0 handlebodies occur in the case of definite folds, both interactive and uninteractive.
In the uninteractive case, there is a section within an edge of the trivalent graph that is not labeled definite.
We extend that region to its termination as in the previous case, labeling the terminating points as definite edges.
When the definite fold interacts with an indefinite fold, the internal edge terminates in a vertex of the trivalent graph that is not labeled definite.
As before, the regions are extended out their terminus and the resulting edges are labeled definite.
In this case there is a vertex corresponding to the termination of a saddle singularity that we label definite.
Both cases are considered in Figure \ref{fig:codim2steindef}

\begin{figure}
	\centering
	\captionsetup{justification=centering}
	\caption{Stein complex of the uninteractive indefinite fold projecting near codimension 2 critical values}
	\includegraphics[height=3cm]{figures/dummy.jpg}
	\label{fig:codim2steinindefun}
\end{figure}

Genus 2 handlebodies are dealt with exactly as in the case of the pair of pants crossed with the interval.
The hole is filled with the local polyhedral model corresponding to the edge, as seen in Figure \ref{fig:codim2steinindefun}.

\begin{figure}
	\centering
	\captionsetup{justification=centering}
	\caption{Crossing saddle singular fiber}
	\includegraphics[height=3cm]{figures/dummy.jpg}
	\label{fig:crossingsaddles}
\end{figure}

When the handlebody is of genus 3, we are in the situations detailed in Section 4.4 of \cite{CostThur08}.
The hole corresponding to $K_4$ is filled in with the polyhedral model corresponding to a vertex.
This Stein complex is simple enough to justify, as it is simply what happens when a pair of saddles cross as in Figure \ref{fig:crossingsaddles}.

\begin{figure}
	\centering
	\captionsetup{justification=centering}
	\caption{Rotating torus singular fiber}
	\includegraphics[height=3cm]{figures/dummy.jpg}
	\label{fig:doublecone}
\end{figure}

The final hole can be filled in with the nonstandard vertex--like compact topological space in \ref{fig:doublecone}.
This justification is more difficult to see, but Figure \ref{fig:doublecone} attempts to demonstrate.
We consider a torus and a Morse function given by a simple height map.
By rotating the torus over a horizontal axis halfway between the punctures, we cause the singular fibers to eventually reside in the same critical level, and then separate again.






	
	They are handlebodies of 
	A sleeve around a codimension 2 critical value $x$ pulls back to a disjoint collection of handlebodies in $M$.
	
	If we were to remove these portions from $S$, we would be left with 
	At this point the boundary of the Stein complex has several connected components.
	The components that intersect the definite edges of the boundary 
	is the disjoint union of a trivalent graph whose edges all intersect definite edges of the boundary along with a some copies of $S^1$ and some graphs of the shapes found in Figure \ref{fig:holegraphs}.


	
	\begin{figure}
		\centering
		\captionsetup{justification=centering}
		\caption{Boundary graphs in the Stein complex}
		\includegraphics[height=3cm]{figures/dummy.jpg}
		\label{fig:holegraphs}
	\end{figure}

The Stein complex formed has the structure of a simple polyhedron almost everywhere.


It is possible to replace this shape with an integer decorated simple polyhedron containing two internal vertices and a trivalent boundary graph matching the hole we need to fill, but we need to discuss integer decoration first.

We now have a Stein complex that almost describes a handle decomposition of a cobordism of $(M,\emptyset)$.
For each vertex we attach a 0--handle, for each edge we attach a 1--handle, and for each region we attach a 2--handle.
The only piece of information missing is a framing coefficient that defines how to attach our 2--handles.
Each region of the complex is be decorated with the framing coefficient for that handle determined during the proof of Theorem \ref{thm:3bound4}.

\begin{figure}
	\centering
	\captionsetup{justification=centering}
	\caption{The integer decorated simple polyhedron that replaces the double cone vertex}
	\includegraphics[height=3cm]{figures/dummy.jpg}
	\label{fig:simplepoly}
\end{figure}

Now that we know what integer decoration means, we do away with the double cone vertex.
We instead fill the final holes of the Stein complex with the simple polyhedron in Figure \ref{fig:simplepoly}.
The boundary regions of the polyhedron are equipped with relative integers that modify the decorations associated to the regions they are joined to in the existing Stein complex.
The internal region has an absolute integer to describe its 2--handle attachment.

We are left with an integer decorated Stein complex of the map $f:M\to\BB$ that describes fully the dual handle decomposition recovered in Theorem \ref{thm:3bound4} for a cobordism of the pair $(M,\emptyset)$.

































Before we delve into the specifics of obtaining the Stein complex and how it provides a set of instructions used to build the dual decomposition, we define the 2--dimensional analogue to a graph that we use to describe the Stein complex combinatorially.

\begin{defn}
	\begin{figure}
		\centering
		\caption{The three local polyhedral models}
		\includegraphics[height=3cm]{figures/dummy.jpg}
		\label{fig:localpoly}
	\end{figure}
	The three \emph{standard local polyhedral models} are the following 3 compact topological spaces, as they sit inside of $\R^3$, under the subspace topology.
	\begin{enumerate}
		\item the closed 2--disc $\D^2$.
		\item the product of the interval $\I$ with the $Y$--shaped graph $K_{1,3}$.
		\item the cone on the complete graph $K_4$.
	\end{enumerate}
	The models each have boundary.
	Being a surface, the 2--disc has a well defined boundary.
	In the local model $K_{1,3}\times I$, the boundary consists of the union of the sets $(K_{1,3}\times \{0\})$, $(K_{1,3}\times \{1\})$, and $(v(K_{1,3})\times I)$, where $v(K_{1,3})$ denotes the vertex set of the graph $K_{1,3}$.
	The cone on $K_4$ is defined as $K_4\times I /\sim$ where $(x,1)\sim (y,1)$ for every $x$, $y$ in $K_4$.
	When we define the cone this way the boundary is $K_4\times \{0\}$.
\end{defn}

\begin{defn}
	Let $P$ be a compact topological space.	
	If every point $p$ of $P$ has a neighbourhood homeomorphic to an open set in one of the standard local polyhedral models, then $P$ is a \emph{simple polyhedron}.
	The set of points which do not have a neighbourhood homeomorphic to an open set in $D^2$ form a 4--valent graph which we call the \emph{singular set} of $P$ and denote by $\Sing (P)$.
	The vertices and edges of $\Sing (P)$ are called the vertices and edges of $P$.
	We call the connected components of $P\setminus\Sing (P)$ the \emph{regions} of $P$.
	A simple polyhedron whose regions are homeomorphic to discs is \emph{standard}.
	
	If a point $p$ of $P$ has a neighbourhood homeomorphic to an open set containing a point in the boundary of one of our three local models, then $p$ is a \emph{boundary point} of $P$.
	The set of all boundary points of $P$ is the boundary of $P$ and is denoted by $\pd P$.
	A region of $P$ is \emph{internal} if its closure is disjoint from $\pd P$.
	If $\pd P$ is empty then $P$ is \emph{closed}.
\end{defn}

\begin{theorem}
	Let $M$ be a closed, orientable 3--manifold and $f:M\to\BB$ a generic proper smooth map.
	Then the Stein complex of $f$ has a canonical embedding in  is a simple polyhedron everywhere except for at most finitely many points.
\end{theorem}



\begin{lem}[Lemma 4.4 from \cite{CostThur08}]
	The simple polyhedron that is the Stein complex of $f$ subject to the conditions in the proof of Theorem \ref{thm:3bound4} is standard.
\end{lem}


\begin{theorem}	
	Let $W$ be a cobordism of the pair $(M,\emptyset)$ constructed in Theorem \ref{thm:3bound4} and $S$ the Stein complex of $f$.
	Then $S$ embeds in $W$ in such a way that the regions of $S$ are locally flat in $W$ and the map $h:M\to S$ of the Stein factorization of $f$ can be extended to a map $h:W\to S$.
\end{theorem}

\begin{rmk}
	The justification that the double cone can be replaced by the polyhedron in Figure \ref{fig:simplepoly} necessitates delving into the central object of study in \cite{CostThur08} and \cite{Turaev91}: the shadow surface.
	The Stein complex constructed in Theorem \ref{thm:3stein4} is a \emph{shadow} of both $M$ and the cobordism $W$ of $(M,\emptyset)$ whose dual decomposition Theorem \ref{thm:3bound4} describes.
	Moreover, the dual decomposition described follows a specific case of the Turaev Reconstruction Theorem that can be found in either \cite{CostThur08} or \cite{Turaev91}.
\end{rmk}







	To satisfy our requirement that the handle decomposition contains handles of index at most 2, we turn the decomposition obtained on its head and take its dual.
	This method begins with an empty space and attaches 4--dimensional handles to it until the boundary of the handle decomposition is exactly $M$.
	
	Let $W_0$ be an empty decomposition.
	The cocores of our 4--handles are single points, so each 4--handle yields a 0--handle in the dual decomposition.
	Add a disjoint 4--disc (i.e.\ a 0--handle) to $W_0$ for each 4--handle added in the final step of building the original decomposition.
	We have after this $W_1=\bigsqcup \D^4$ whose boundary is $\pd W_1 = \bigsqcup S^3$.
	This is the same boundary that we had before adding any 4--handles to the original decomposition.
	
	The cocores of our 3--handles are intervals that connect 4--handles.
	The belt sphere of a 3--handle is the boundary of the cocore, so it is a copy of $S^0$ intersecting exactly two 4--handles.
	We connect the associated 0--handles by 3--disc bundles over copies of the interval whose endpoints are embedded in our 0--handles (i.e.\ attach 1--handles) in a way that preserves orientability.
	We get $W_2$ as the 4--dimensional thickening of a graph whose vertices correspond to 0--handles and whose edges correspond to 1--handles.
	The boundary of this is the collection of $S^3$'s from $\pd W_1$ that have been connected together by copies of $S^2\times\I$.
	
	Attaching 0--handles and 1--handles in a way that preserves orientability is done uniquely.
	This is not the case with 4--dimensional 2--handles (see Remark \ref{rmk:2handle}).
	The 2--handles must be attached in a way that is dual to the attachment described in the original handle decomposition, else the boundary of the resulting 4--manifold need not be $M$.
	
	An original 2--handle is attached over a solid torus $V$ with $f(V)=D$, and the desired action on the boundary $M_0$ was the replacement of $V$ by $V^*$, a solid torus that could be filled in by a 3--handle.
	The core of the handle attached over $V$ is the disc bounded by the zero section $z(V)$, and the cocore of the handle is the disc bounded by the zero section $z(V^*)$.
	The core and cocore intersect at $\vec{0}$, and nowhere else.
	
	The attachment of 3--and 4--handles filled in $V^*$, so adding 0-- and 1--handles to our dual decomposition recovers $V^*$, the 4--manifold constructed contains $V^*$, running through $S^3$'s and along $S^2\times\I$'s, but not $V$.
	We give $V^*$ the structure of a trivial 2--disc bundle over the belt sphere of the original 2--handle with trivialization $\psi:V^*\to S^1\times\DD$.
	In the boundary of $V^*$ we find the curves $J$, $K$ and $L$ as defined before, displayed in Figure \ref{fig:VV*}.
	Just as it was when we were attaching the original 2--handle, Figure \ref{fig:VV*} is slightly misleading.
	What it really shows us is how the curves sit in $\psi(V^*)$.
	
	In $S^1\times\DD$, $\psi(J)$ is a meridian, $\psi(L)$ is a curve that wraps once around $\pd(S^1\times\DD)$ in the meridinal direction and $\kappa$ times around $\pd(S^1\times\DD)$ in the longitudinal direction, with oriented intersection count with $\psi(J)$ of $\kappa$, and $\psi(K)$ is exactly $S^1\times\{1\}$ in $\pd(S^1\times\DD)$.
	
	For a 2--handle attached over $V^*$, take $\pd (\DD\times\DD)$ to be the genus 1 Heegaard splitting of $S^3$ as before.
	One of the solid tori will be identified with $V^*$ and the other will be $V^{**}$.
	We need an explicit identification $G^*:V^*\to S^1\times\DD$ such that $G^*(K)$ is a meridian of $V^{**}$
	
	At this point it becomes convenient to view $T^2$ as the space $[0,1]\times[0,1]/\sim$, where $(\theta,0)\sim(\theta,1)$ and $(0,\phi)\sim(1,\phi)$.
	Using the notation from Theorem \ref{thm:mpgV}, a meridian is of the form $(\theta,0)$, and $\psi(J)$ is then isotopic to a line with slope $1/0$, and $\psi(K)$ is isotopic to the longitude $(0,\phi)$, which is a line with slope $0/1$.
	We find that $\psi(L)$ is isotopic to a line of slope $-1/\kappa$, or, alternatively, is a line of the form $-\kappa\phi=\theta$.
	
	
	
	
	
	\begin{defn}
		Let $S$ be a compact topological space that is generically a 2--manifold.
		We call the set of points in $S$ at which $S$ fails to be a 2--manifold the \emph{singular set} of $S$, and denote it by $\Sing(S)$.
		In other words, we take $\Sing(S)$ to be the minimal subset of $S$ such that $S\setminus\Sing(S)$ is a (possibly disconnected) closed 2--manifold.
		The connected components of $\Sigma$ are called the \emph{regions} of $S$.
		Note that when $S$ is a 2--manifold with boundary, $\Sing(S)=\pd S$.
		The singular set of $S$ is generically a 1--manifold.
	\end{defn}
	
	
	
	
	
	
	We simultaneously construct the Stein complex of $f$ and the map $\psi$.
	A disc $D$ of regular values of $f$ pulls back to a collection of solid tori.
	To obtain a handle decomposition for $W$, we first attached 2--handles over these tori.
	
	
	
	These circle bundles share a boundary 2--torus $T=\pd V=\pd V^*$.
	Let $\mathcal{J}$ be the isotopy class of curves in $T$ that are the preimages of points in the boundary of $D$.
	Recall that a representative of $\mathcal{J}$ is a longitude of $V$ and a meridian of $V^*$ by construction.
	This means that some representative $J$ of $\mathcal{J}$ bounds a disc in $V^*$, and is a section over the boundary of the circle bundle over the core of $H$.
	We first examine $H$ as a 2--disc bundle over its cocore $\{\vec{0}\}\times\DD$, appearing as the zero section of the bundle.
	By 
	The cocore of $H_{i,k}^2$ is a disc embedded in $W$, .cocore of these 2--handles are 
	has $f\inv(D_i)$ a circle bundle over a disjoint collection of 2--discs, and the components of this collection collapse through $h$ to connected regions of regular levels in the Stein complex.
	
	
	A sleeve around an arc of codimension 1 critical values (Figure \ref{fig:arcsleeve}) pulls back to a disjoint collection of surfaces crossed with the interval.
	In the case of the annulus crossed with an interval, $h$ sees $A\times\I$ as a bundle of preimage circles over $\I\times\I$ and collapses that bundle onto $\I\times\I$.
	Two ends of this strip already map through $\psi$ to the boundaries of a pair of cocores of 2--handles.
	In $S$, we get a copy of $\I\times\I$ bridging a pair of 2--discs from the previous step.
	We define $\psi$ on this copy of $\I\times\I$ to bridge the associated cocores in $W$.
	
	When the surface is a disc, there is an interval of boundary circles that map through $f$ to regular values, so $\psi$ is already defined on this circle.
	As we move inwards along concentric circles towards the central interval $\{0\}\times\I$ of $\DD\times\I$, $h$ collapses these circles to extend an existing region of the Stein complex.
	When we get to the core of the solid cylinder, $h$ takes $\{0\}\times\I$ to a boundary edge of the extended region.
	We label the portion of the boundary edge formed this way by $d$ for \emph{definite}, and this edge becomes part of $\Sing(S)$.
	The central interval corresponds to the cocore of the 3--handle we attached over $D^2\times\I\subset S^2\times\D^1$.
	We define $\psi$ on the definite boundary edge to go to the cocore of this 3--handle, and extend over the rest of the region in a smooth way.
	
	When the surface is a pair of pants, the boundary circles of the pants map through $\psi$ over the boundaries of 2--handle cocores, and those cocores correspond to regions of $S$.
	The interval of critical points in $P\times\I$ corresponds to the cocore of a 3--handle, as in the case of the 2--disc.
	The three regions of $S$ corresponding to the cocores discussed above extend to meet along an internal edge in $\Sing(S)$.
	The neighbourhood of an internal edge in $S$ looks like a ``three--page book,'' or a $Y$--shaped graph crossed with an interval.
	We take this internal edge to the cocore of the 3--handle attached over $P\times\I\subset S^2\times\D^1$ through $\psi$, and extend $\psi$ over the regions meeting this edge in a smooth way.
	
	\begin{figure}
		\centering
		\captionsetup{justification=centering}
		\caption{Stein complex of genus 1 handlebody of regular points projecting near codimension 2 critical values}
		\includegraphics[height=3cm]{figures/dummy.jpg}
		\label{fig:codim2steinregular}
	\end{figure}
	
	The only parts of the Stein complex we haven't investigated yet are the portions corresponding to the codimension 2 critical values of $f$.
	Let $x$ be a codimension 2 critical value and $\nbhd{x}$ its neighbourhood (Figure \ref{fig:isolatesleeve}).
	We've already classified the subspaces of $M$ that project over $\nbhd{x}$: they are handlebodies.
	Such a handlebody completes into $S^3$ in every case encountered during our construction of $W$, and over these 3--spheres we attached 4--handles.
	Let $G$ be a handlebody that projects over $\nbhd{x}$.
	
	
	\newpage
	{
		section incomplete
	}
	\newpage
	
	
	
	
	
	
	