\label{sec:3bound4}

To extend the results of the previous section to the case of 3--manifolds bounding 4--manifolds, we consider generic proper smooth maps from closed orientable 3--manifolds into the 2--ball $B^2$.
Such a map is Morse--like in the sense that its singular set is well behaved, it can be studied via the same techniques as in Section \ref{sec:2bound3}, and we may recover a Stein factorization and Stein complex that define a handle decomposition for a bounding 4--manifold.

\begin{theorem}
	\label{thm:3bound4}
	Let $M$ be a closed orientable 3--manifold.
	Then a proper generic smooth map $f:M\to \BB$ determines a handle decomposition for a cobordism of the pair $(M,\emptyset)$ that contains handles of index at most 2.
\end{theorem}

\begin{proof}
	By Sard's theorem, the image of the singular set $f(s_f)$ in $\BB$ has Lebesgue measure 0.
	By genericity, and because the maps constructed in Chapter \ref{cha:alg1} satisfy this requirement, we take $s_f$ to consist of a set of arcs in the plane that intersect each other only pairwise and transversally, in such a way that $\BB\setminus s_f$ consists of a collection of regions $R_i$ homeomorphic to copies of $\BB$ along with a single annular region $R_\infty$ that does not intersect $f(M)$, and in such a way that $f(s_f)$ has a natural structure as a simple undirected 4--valent planar graph.
	We classify the critical values of $f$ as codimension 1 inside of arcs away from crossings, and codimension 2 at arc crossings.
	This classification makes the simple undirected planar graph structure of $f(s_f)$ explicit.
	The vertex set is given by the set of codimension 2 critical values and the edge set is the collection of codimension 1 critical values.
	
	We obtain a 4--manifold with boundary $M$ by attaching 2--handles corresponding to the regular regions $R_i$, 3--handles corresponding to codimension 1 singularities, and 4--handles corresponding to codimension 2 singularities.
	
	We will first focus on a region $R$ in $\BB$ that is not $R_\infty$.
	Begin by ``shrinking'' $R$ away from $f(s_f)$ into a space $\D$ that is homeomorphic to $\DD$
	Every point of $\D$ has preimage a disjoint collection of circles by the regular value theorem, so we centre our analysis on an arbitrary regular value $x$ and its regular fibers, where $M_{x,k}$ denotes the $k\nth$ regular fiber mapped over $x$ by $f$.
	An appropriate closed tubular neighbourhood of $M_{x,k}$ will consist entirely of regular fibers that map into $\D$ and has the structure of a closed 2--disc bundle over the circle, hence is a solid torus.
	The neighbourhood can be extended to enclose the entirety of $f\inv(\D)$.
	
	In summary, an arbitrary point $x_i$ is chosen from a closed 2--disc $D_i$ which itself is taken as a shrinking of an open region $R_i$.
	For each $k$, the $k\nth$ regular fiber $M_{i,k}$ over $x_i$ is taken as the zero section of a 2--disc bundles that map over $D_i$.
	Such a 2--disc bundles is an open solid tori, and the $k\nth$ bundle is denoted $V_{i,k}$.
	Of particular importance in $V_{i,k}$ is the zero section, which is a regular fiber that maps over a point interior to $\D_i$, and the isotopy class for a longitude found as the $k\nth$ regular fiber over a point in the boundary of $\D_i$.
	This pair is used to completely determine an attaching map for 4--dimensional 2--handle attached over $V_{i,k}$.
	
	\begin{figure}
		\centering
		\caption{A closed sleeve around an arc of codimension 1 critical values}
		\includegraphics[height=3cm]{figures/dummy.jpg}
		\label{fig:arcsleeve}
	\end{figure}
	
	Consider an arc $\alpha$ of codimension 1 critical values that separates a pair of regions $R_i$ and $R_j$ where either $i$ or $j$ may be $\infty$.
	As when we considered regions, $\alpha$ has been shrunk away from the codimension 2 critical values it connects.
	Let $\nbhd{\alpha}$ be a closed ``sleeve'' around $\alpha$ whose boundary intersects the boundaries of $D_i$ and $D_j$ as in Figure \ref{fig:arcsleeve}.
	The restriction of $f$ to the preimage of a linear transversal to $\alpha$ that connects $D_i$ and $D_j$ is, itself, a Morse function.
	The intersection with $\alpha$ is a critical point of this restriction, and the same techniques from the proof of Theorem \ref{thm:2bound3} may be applied to $f\inv(\nbhd{\alpha})$.
	A strand of singular fibers over $\alpha$ consists of an interval in the case of an extremum for the associated Morse function or an interval crossed with the figure 8 when the associated Morse function yields a saddle singularity.
	The critical neighbourhood $f\inv(\nbhd{\alpha})$ contains a connected component that is either $\DD\times\I$ when we have cross sectional extremum, or $P\times\I$, where $P$ is the pair of pants surface, i.e.\ the twice-punctured 2--disc, when the cross section yields a saddle.
	All remaining connected components are $A\times\I$ where $A$ is an annulus, i.e.\ the once-punctured 2--disc.
	
	Whenever one of the regions that $\alpha$ separates is $R_\infty$, the cross section necessarily gives us an extremum, and the critical neighbourhood over $\nbhd{\alpha}$ is $\DD\times\I$.
	We call $\alpha$ a \emph{definite fold} when the critical neighbourhood is $\DD\times\I$, and an \emph{indefinite fold} when the critical neighbourhood is $P\times\I$.
	
	\begin{figure}
		\centering
		\caption{A closed sleeve around a codimension 2 critical value}
		\includegraphics[height=3cm]{figures/dummy.jpg}
		\label{fig:isolatesleeve}
	\end{figure}
	
	Let $x$ be a codimension 2 critical value and $\nbhd{x}$ the closed neighbourhoods whose boundary agrees with the shrunken regions and arc sleeves it is near as in Figure \ref{fig:isolatesleeve}.
	The possible singular fibers over $x$ are cataloged in \cite{Saeki}, and Figure \ref{fig:saekising} displays them.
	When the singular fibers are disconnected, $f\inv(\nbhd{x})$ is categorized by the codimension 1 analysis.
	Otherwise, we can find a full analysis in Section 4.4 of \cite{CostThur08}.
	In each case, $\nbhd{x}$ pulls back to a genus 3 handlebody whose boundary is $f\inv(\pd\,\nbhd{x})$.
	
	\begin{figure}
		\centering
		\captionsetup{justification=centering}
		\caption{Possible singular fibers of a proper generic smooth map from an orientable 3--manifold to a surface}
		\includegraphics[height=3cm]{figures/dummy.jpg}
		\label{fig:saekising}
	\end{figure}
	
	We may now begin describing a handle decomposition for a cobordism $W$ of the pair $(M,\emptyset)$.
	Begin with $M\times\I$, whose boundary is $M_0\sqcup M_1$ with $M_0=M\times\{0\}$ and $M_1 = M\times\{1\}$, and a proper generic smooth map $f:M_0\to\BB$.

	For each $i<\infty$ indexing the connected regions of $\BB\setminus f(s_f)$, we index over $k$ the connected components of $f\inv(R_i)$.
	Consider $D_i\subset R_i$, a 2--disc of regular values, and $V_{i,k}$, the $k\nth$ regular solid torus that maps over $D_i$.
	The torus $V_{i,k}$ has disc bundle structure over the regular fiber $M_{i,k}$ with $f(M_{i,k})=x$ in the interior of $\D_i$.
	To attach a 2--handle over $V_{i,k}$, we need
	\begin{enumerate}
		\item The isotopy class of an embedding $\varphi_0:S^1\to M_0$, and
		\item the isotopy class of an identification of $S^1\times\DD$ with $\nbhd{\varphi_0(S^1)}$.
	\end{enumerate}
	The embedding $S^1\to M_0$ is easy enough to define as it takes $S^1$ to $M_{i,k}$.
	For an identification of $S^1\times\DD$ with $\nbhd{\varphi_0(S^1)}$, consider the isotopy class of $f\inv(y)$ in $\pd V_{i,k}$ for an arbitrary point $y$ in the boundary of $D_i$.
	Becuase $f\inv(y)$ is a longitude, its isotopy class can be represented by a curve in the boundary of $V_{i,k}$ of the form $h_{M}^n(S^1\times\{1\})$ for some $n$, cf.\ Remark \ref{rmk:2handle}.
	Map the circle $S^1\times\{1\}$ in $S^1\times\DD$ to $h_{M}^n(S^1\times\{1\})$, and extend over the boundary torus by rotation.
	Note that the definition of $h_M$ gives an orientation to the rotations corresponding to positive or negative integers $n$.
	 and wh
	This extends to a map $H_M^n$ as in Remark \ref{rmk:2handle}.

	Attachment of a 2--handle modifies the boundary $M_0$ in a controlled way.
	Colloquially, we delete a solid torus that is ``mean'' in some way and replace it with a solid torus that is ``nice,'' where ``nice'' means that the newly introduced solid torus can be deleted by attaching a 3--handle.
	More precisely, we remove $V_{i,k}$ and glue 2--discs to the longitudes in $\pd V_{i,k}$ that map over single points in the boundary of $D_i$.
		
	With 2--handles attached, we move onto the preimages of arc sleeves.
	Let $\nbhd{\alpha}$ be an arc sleeve, and consider the connected components of $f\inv(\nbhd{\alpha})$.
	Figure \ref{fig:arcsleevepre} displays the possible connected components of arc sleeve preimages, each of which has the form of a surface crossed with the interval.
	The boundary circles of these surfaces at a cross section $\Sigma\times\{t\}$ project through $f$ over the boundaries of shrunken regions, so they are filled with discs that sit inside of attached 2--handles from the previous step.
	In each case, we obtain a copy of $S^2\times\D^1$ over which we attach a 3--handle.
	The further modification to $M_0$ that takes place when we attach 3--handles can be thought of as the deletion of a copy of $S^2\times D^1$ followed by gluing 3--discs over the newly created 2--sphere boundary components.
	
	\begin{figure}
		\centering
		\captionsetup{justification=centering}
		\caption{Possible connected components of arc sleeve preimages}
		\includegraphics[height=3cm]{figures/dummy.jpg}
		\label{fig:arcsleevepre}
	\end{figure}
	
	Finally, let $x$ be a codimension 2 critical value and let $\nbhd{x}$ be its sleeve.
	We find $x$ at the crossing of a pair of strands of codimension 1 critical values, and those strands are classified as definite or indefinite folds.
	The analysis of the codimension 2 critical value is then broken down into cases:
	\begin{enumerate}[label=(\alph*)]
		\item \label{case:defdef} The crossing of definite folds.
		\item \label{case:defindisco} The crossing of a definite fold with an indefinite fold with disconnected singular fiber.
		\item \label{case:definconn} The crossing of a definite fold with an indefinite fold with connected singular fiber.
		\item \label{case:inindisc} The crossing of indefinite folds with disconnected singular fiber.
		\item \label{case:ininconn} The crossing of indefinite folds with connected singular fiber.
	\end{enumerate}
	Cases \ref{case:defdef}, \ref{case:defindisco}, and \ref{case:inindisc} are simply extensions of the codimension 1 situation, except we are at the point where we may attach 4--handles to close off these boundary components completely.
	The folds and singular fibers in cases \ref{case:ininconn} and \ref{case:definconn} are said to be \emph{interactive}, rather than the \emph{uninteractive} cases \ref{case:defdef}, \ref{case:defindisco}, and \ref{case:inindisc}.
	
	\begin{figure}
		\centering
		\captionsetup{justification=centering}
		\caption{Regular level surfaces projecting near a codimension 2 critical value}
		\includegraphics[height=3cm]{figures/dummy.jpg}
		\label{fig:regprojcodim2}
	\end{figure}
	
	First, however, we look at the connected components of $f\inv(\nbhd{x})$ that do not contain a singular fiber over $x$.
	These connected components are made from regular fibers, i.e.\ circles, that project over $\nbhd{x}$, i.e. a 2--disc.
	As in the case of regions of regular values, they are solid tori $V_{x,k}$, using the same naming convention that has been established for solid tori that map over the regions $D_i$.
	Seeing $V_{x,k}$ as a circle bundles over $\nbhd{x}$, we can see that longitudes of $V_{x,k}$ map over the boundary $\pd\nbhd{x}$.
	This means the longitudinal circles of $V_{x,k}$ were filled with discs when we attached 2-- and 3--handles.
	Taking the union of these discs to be meridinal in another solid torus $V_{x,k}^*$, we now have a Heegaard splitting $V_{x,k}\cup_\varphi V_{x,k}^*$ where $\varphi$ takes a longitude of $V_{x,k}$ to a meridian of $V_{x,k}^*$.
	This is the standard genus 1 Heegaard splitting of $S^3$, so we have a copy of $S^3\times\D^0$ over which we may attach a 4--handle.

	\begin{figure}
		\centering
		\captionsetup{justification=centering}
		\caption{Definite fold surfaces projecting near a codimension 2 critical value}
		\includegraphics[height=3cm]{figures/dummy.jpg}
		\label{fig:deffoldprojcodim2}
	\end{figure}

	The component that comes from an uninteractive definite fold is of the form $\DD\times\I$ as in the case of codimension 1 critical values.
	The 2--sphere boundary of this shape is filled with 2--discs from our 2-- and 3--handles, so this shape is the adjunction of a pair of 3--discs glued over their boundary.
	This is the genus 0 Heegaard splitting of $S^3$, so we have a copy of $S^3\times\D^0$ over which we may attach a 4--handle.
	
	\begin{figure}
		\centering
		\captionsetup{justification=centering}
		\caption{Indefinite uninteractive fold surfaces projecting near a codimension 2 critical value}
		\includegraphics[height=3cm]{figures/dummy.jpg}
		\label{fig:indeffoldprojcodim2}
	\end{figure}
	
	\begin{figure}
		\centering
		\caption{Destabilizing pairs for the pair of pants}
		\includegraphics[height=3cm]{figures/dummy.jpg}
		\label{fig:destabpants}
	\end{figure}
		
	An uninteractive singular fiber that comes from an indefinite fold is the figure 8, and the preimage $f\inv(\nbhd{x})$ is a closed tubular neighbourhood of the figure 8 graph, which is homeomorphic to the pair of pants crossed with an interval.
	Here, we find a Heegaard splitting of genus 2 with two destabilizing pairs, showing that the splitting is a splitting of $S^3$, and we can attach a 4--handle to it by Theorem \ref{thm:lilwald}.
	An appropriate choice of meridians in $f\inv(\nbhd{x})$ would be a pair of curves that run from cuff to waist along the inside of each of the legs in the pair of pants, crest the waist, then down the outside of the legs back to the cuff.
	Meridians for the solid filled by our 2-- and 3--handles would run around the cuffs of each leg.
	Figure \ref{fig:destabpants} illustrates the idea.
	
	\begin{figure}
		\centering
		\captionsetup{justification=centering}
		\caption{The surfaces resulting from interaction between definite and indefinite folds as they project near a codimension 2 critical value}
		\includegraphics[height=3cm]{figures/dummy.jpg}
		\label{fig:defindefconn}
	\end{figure}
	
	When an indefinite and definite fold interact, we get the same shapes found in the case of an uninteractive definite fold.
	From a 1--dimensional Morse function viewpoint, we witness a homotopy of a pair of peaks separated by a saddle come together into a single peak, eliminating the saddle.
	See Figure \ref{fig:defindefconn} for a demonstration.

	\begin{figure}
		\centering
		\caption{Indefinite interactive fold surfaces projecting near a codimension 2 critical value}
		\includegraphics[height=3cm]{figures/dummy.jpg}
		\label{fig:interactivefoldprojcodim2}		
	\end{figure}
	
	This leaves us with the case of interacting indefinite folds.
	The full analysis is covered in Section 4.4 of \cite{CostThur08}.
	They show that filling in $f\inv(\pd\,\nbhd{x})$ with 2--discs and gluing the two genus 3 handlebodies together over $f\inv(\pd\,\nbhd{x})$ results in a Heegard splitting of $S^3$ by explicitly finding 3 destabilizing pairs.
	We again attach 4--handles over these 3--spheres.
	Attaching a 4--handle over a 3--sphere fills in the 3--sphere with a 4--disc, so these final modifications completely eliminate what is left of $M_0$.
	We are left with a handle decomposition for a cobordism of the pair $(M,\emptyset)$ built on top of $M_0$ in $M\times\I$.

	To satisfy our requirement that the handle decomposition contains handles of index at most 2, we turn the decomposition obtained on its head and take its dual.
	We begin with an empty decomposition.
	At first we add a disjoint 4--disc for each 4--handle added in the final step of building the original decomposition.
	The cocores of our 4--handles are single points, so each 4--handle yields a 0--handle in the dual decomposition.
	The cocores of our 3--handles are intervals that connect 4--handles.
	The belt sphere of a 3--handle is the boundary of the cocore, so it is a copy of $S^0$ intersecting exactly two 4--handles.
	We connect the associated 0--handles by a 1--handle in a way that preserves orientability.

	Attachment of 0--handles and 1--handles that preserve orientability are done uniquely.
	This is not the case with 4--dimensional 2--handles, as we may recall from remark \ref{rmk:2handle}.
	The 2--handles must be attached in a way that is dual to the attachment described in the original handle decomposition, else the boundary of the resulting 4--manifold need not be $M$.
	Original 2--handles were attached over solid tori in $M_0$ that were seen as the connected components of preimages of a solid discs of regular points in the image of $f$.
	Let $V_{i,k}$ be the $k\nth$ solid torus projecting over $D_i$ as before.
	To dissect the anatomy of the 2--handle attached over $V_{i,k}$, we begin with the attaching sphere.
	We chose an arbitrary point $x$ in the interior of $D_i$ and took $f\inv(x)$ to be the attaching sphere, with $V_{i,k}$ the tubular neighbourhood to $f\inv(x)$.
	


		
\end{proof}

\begin{rmk}
	\label{rmk:definitehandles}
	Before touching on the Stein complex, we discuss attaching 2--handles over 1--handles corresponding to definite folds.
\end{rmk}


\begin{theorem}
	\label{thm:3stein4}
	Let $M$ be a closed orientable 3--manifold.
	The dual handle decomposition recovered from a proper generic smooth map $f:M\to \BB$ in Theorem \ref{thm:3bound4} is described fully by an integer decorated Stein complex of $f$.
\end{theorem}

To examine the Stein complex of a proper generic smooth map such as $f$, we first define the simple polyhedron.

\begin{defn}
	\begin{figure}
		\centering
		\caption{The three local polyhedral models}
		\includegraphics[height=3cm]{figures/dummy.jpg}
		\label{fig:localpoly}
	\end{figure}
	The three \emph{standard local polyhedral models} are the following 3 compact topological spaces, as they sit inside of $\R^3$, under the subspace topology.
	\begin{enumerate}
		\item the closed 2--disc $\D^2$.
		\item the product of the interval $\I$ with the $Y$--shaped graph $K_{1,3}$.
		\item the cone on the complete graph $K_4$.
	\end{enumerate}
	The models each have boundary.
	Being a surface, the 2--disc has a well defined boundary.
	In the local model $K_{1,3}\times I$, the boundary consists of the union of the sets $(K_{1,3}\times \{0\})$, $(K_{1,3}\times \{1\})$, and $(v(K_{1,3})\times I)$, where $v(K_{1,3})$ denotes the vertex set of the graph $K_{1,3}$.
	The cone on $K_4$ is defined as $K_4\times I /\sim$ where $(x,1)\sim (y,1)$ for every $x$, $y$ in $K_4$.
	When we define the cone this way the boundary is $K_4\times \{0\}$.
\end{defn}

\begin{defn}
	Let $P$ to be a compact topological space.	
	If every point $p$ of $P$ has a neighbourhood homeomorphic to an open set in one of the standard local polyhedral models, then $P$ is a \emph{simple polyhedron}.
	The set of points which do not have a neighbourhood homeomorphic to an open set in $D^2$ form a 4--valent graph which we call the \emph{singular set} of $P$ and denote by $\Sing (P)$.
	The vertices and edges of $\Sing (P)$ are called the vertices and edges of $P$.
	We call the connected components of $P\setminus\Sing (P)$ the \emph{regions} of $P$.
	
	
	If a point $p$ of $P$ has a neighbourhood homeomorphic to an open set containing a point in the boundary of one of our three local models, then $p$ is a \emph{boundary point} of $P$.
	The set of all boundary points of $P$ is the boundary of $P$ and is denoted by $\pd P$.
	A region of $P$ is \emph{internal} if its closure is disjoint from $\pd P$.
	If $\pd P$ is empty then $P$ is \emph{closed}.
\end{defn}

\begin{proof}[Proof of Theorem \ref{thm:3stein4}]
	The Stein factorization and complex of $f$ are defined as before.
	Define an equivalence relation $\sim$ on $M$ by $p\sim q$ if and only if $f(p)=f(q)=x$ and $p$, $q$ are in the same connected component of $f\inv(x)$.
	Then $f=g\comp h$ where $h$ is the quotient map $M\to M/\!\!\sim$ and $g$ takes a point in $f\inv(x)$ to $x$.
	
	\begin{figure}
		\centering
		\captionsetup{justification=centering}
		\caption{Stein complex over regular values}
		\includegraphics[height=3cm]{figures/dummy.jpg}
		\label{fig:regstein}
	\end{figure}
	
	We get simple polyhedral structure in almost all cases, so we will use the language of simple polyhedra to describe the Stein complex.
	A disc $D_i$ of regular values has $f\inv(D_i)$ a circle bundle over a disjoint collection of 2--discs, and the components of this collection collapse through $h$ to connected regions of regular levels in the Stein complex.
	
	\begin{figure}
		\centering
		\captionsetup{justification=centering}
		\caption{Stein complex over codimension 1 critical values}
		\includegraphics[height=3cm]{figures/dummy.jpg}
		\label{fig:codim1stein}
	\end{figure}	
	
	A sleeve around an arc of codimension 1 critical values pulls back to a disjoint collection of surfaces crossed with the interval.
	In the case of the annulus crossed with an interval, $h$ sees $A\times\I$ as a circle bundle over $\I\times\I$ and collapses that bundle into a bridge between a pair of regions of regular levels.
	The pair of regular levels join into a single region of regular levels.
	When the surface is a disc, that disc has boundary circle that maps through $f$ to a regular value, so $h$ already takes that circle to a regular level.
	As we move inwards along internal circles towards the core of $\DD\times\I$, $h$ continues to collapse these circles to regular levels, extending an existing region of the Stein complex.
	When we get to the core of the solid cylinder, $h$ takes $\{0\}\times\I$ to a boundary edge of the extended region.
	We label the portion of the boundary edge formed this way by $e$ for \emph{definite}.
	When the surface is a pair of pants, the boundary circles of the pants all lie in distinct regions of regular levels.
	The circles join together at the singular fibers mapping over the arc of critical values, and $h$ maps the strand of singular fibers to an internal edge of the Stein complex.
	One may also think of this case as a copy of the local polyhedral model corresponding to an internal edge connecting the three regions.

	\begin{figure}
		\centering
		\captionsetup{justification=centering}
		\caption{Boundary graphs in the Stein complex}
		\includegraphics[height=3cm]{figures/dummy.jpg}
		\label{fig:holegraphs}
	\end{figure}
	
	At this point the boundary of the Stein complex is the disjoint union of a trivalent graph whose edges all intersect definite edges of the boundary along with a some copies of $S^1$ and some graphs of the shapes found in Figure \ref{fig:holegraphs}.
	A sleeve around a codimension 2 critical value $x$ pulls back to a disjoint collection of handlebodies in $M$.
	
	\begin{figure}
		\centering
		\captionsetup{justification=centering}
		\caption{Stein complex of genus 1 handlebody of regular points projecting near codimension 2 critical values}
		\includegraphics[height=3cm]{figures/dummy.jpg}
		\label{fig:codim2steinregular}
	\end{figure}
	
	Such a handlebody is of genus 1 when it contains no critical points and is of genus 0 or 2 when the folds over $x$ are not interactive.
	The handlebody of genus 1 collapses through $h$ to fill a hole in a regular level, and in this way we cap off all $S^1$ boundary regions of the Stein complex with 2--discs.

	\begin{figure}
		\centering
		\captionsetup{justification=centering}
		\caption{Stein complex of uninteractive and interactive definite folds projecting near codimension 2 critical values}
		\includegraphics[height=3cm]{figures/dummy.jpg}
		\label{fig:codim2steindef}
	\end{figure}
	
	The genus 0 handlebodies occur in the case of definite folds, both interactive and uninteractive.
	In the uninteractive case, there is a section within an edge of the trivalent graph that is not labeled definite.
	We extend that region to its termination as in the previous case, labeling the terminating points as definite edges.
	When the definite fold interacts with an indefinite fold, the internal edge terminates in a vertex of the trivalent graph that is not labeled definite.
	As before, the regions are extended out their terminus and the resulting edges are labeled definite.
	In this case there is a vertex corresponding to the termination of a saddle singularity that we label definite.
	Both cases are considered in Figure \ref{fig:codim2steindef}

	\begin{figure}
		\centering
		\captionsetup{justification=centering}
		\caption{Stein complex of the uninteractive indefinite fold projecting near codimension 2 critical values}
		\includegraphics[height=3cm]{figures/dummy.jpg}
		\label{fig:codim2steinindefun}
	\end{figure}
	
	Genus 2 handlebodies are dealt with exactly as in the case of the pair of pants crossed with the interval.
	The hole is filled with the local polyhedral model corresponding to the edge, as seen in Figure \ref{fig:codim2steinindefun}.
	
	\begin{figure}
		\centering
		\captionsetup{justification=centering}
		\caption{Crossing saddle singular fiber}
		\includegraphics[height=3cm]{figures/dummy.jpg}
		\label{fig:crossingsaddles}
	\end{figure}
	
	When the handlebody is of genus 3, we are in the situations detailed in Section 4.4 of \cite{CostThur08}.
	The hole corresponding to $K_4$ is filled in with the polyhedral model corresponding to a vertex.
	This Stein complex is simple enough to justify, as it is simply what happens when a pair of saddles cross as in Figure \ref{fig:crossingsaddles}.

	\begin{figure}
		\centering
		\captionsetup{justification=centering}
		\caption{Rotating torus singular fiber}
		\includegraphics[height=3cm]{figures/dummy.jpg}
		\label{fig:doublecone}
	\end{figure}
	
	The final hole can be filled in with the nonstandard vertex--like compact topological space in \ref{fig:doublecone}.
	This justification is more difficult to see, but Figure \ref{fig:doublecone} attempts to demonstrate.
	We consider a twice punctured torus and a Morse function given by a simple height map.
	By rotating the torus over a horizontal axis halfway between the punctures, we cause the singular fibers to eventually reside in the same critical level, and then separate again.
	It is possible to replace this shape with an integer decorated simple polyhedron containing two internal vertices and a trivalent boundary graph matching the hole we need to fill, but we need to discuss integer decoration first.
	
	We now have a Stein complex that almost describes a handle decomposition of a cobordism of $(M,\emptyset)$.
	For each vertex we attach a 0--handle, for each edge we attach a 1--handle, and for each region we attach a 2--handle.
	The only piece of information missing is a framing constant that defines how to attach our 2--handles.
	Each region of the complex that does not contain a definite edge, cf. Remark \ref{rmk:definitehandles}, will be decorated with the framing constant for that handle determined during the proof of Theorem \ref{thm:3bound4}.
	
	\begin{figure}
		\centering
		\captionsetup{justification=centering}
		\caption{The integer decorated simple polyhedron that replaces the double cone vertex}
		\includegraphics[height=3cm]{figures/dummy.jpg}
		\label{fig:simplepoly}
	\end{figure}
	
	Now that we know what integer decoration means, we do away with the double cone vertex.
	We instead fill the final holes of the Stein complex with the simple polyhedron in Figure \ref{fig:simplepoly}.
	The regions decorated by relative integers modify the regions they are joined to in the existing Stein complex, whereas the absolute integral region stands alone.	  
	
	We are left with an integer decorated Stein complex of the map $f:M\to\BB$ that describes fully the dual handle decomposition recovered in Theorem \ref{thm:3bound4} for a cobordism of the pair $(M,\emptyset)$.
\end{proof}

\begin{rmk}
	The justification that the double cone can be replaced by the polyhedron in Figure \ref{fig:simplepoly} necessitates delving into the central object of study in \cite{CostThur08} and \cite{Turaev91}: the shadow surface.
	The Stein complex constructed in Theorem \ref{thm:3stein4} is a \emph{shadow} of both $M$ and the cobordism $W$ of $(M,\emptyset)$ whose dual decomposition Theorem \ref{thm:3bound4} describes.
	Moreover, the dual decomposition described follows a specific case of the Turaev Reconstruction Theorem that can be found in either \cite{CostThur08} or \cite{Turaev91}.
\end{rmk}
