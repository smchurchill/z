\label{sec:3bound4}

To extend the results of the previous section to the case of 3--manifolds bounding 4--manifolds, we consider generic proper smooth maps from closed orientable 3--manifolds into the 2--ball $B^2$.
Such a map is Morse--like in the sense that its singular set is well behaved, it can be studied via the same techniques as in Section \ref{sec:2bound3}, and we may recover a Stein factorization and Stein complex that define a handle decomposition for a bounding 4--manifold.

\begin{theorem}
	\label{thm:3bound4}
	Let $M$ be a closed orientable 3--manifold.
	Then a proper generic smooth map $f:M\to \BB$ determines a handle decomposition for a cobordism of the pair $(M,\emptyset)$.
\end{theorem}

\begin{proof}
	By Sard's theorem, the image of the singular set $f(s_f)$ in $\BB$ has Lebesgue measure 0.
	By genericity, and because the maps constructed in Chapter \ref{cha:alg1} satisfy this requirement, we take $s_f$ to consist of a set of arcs in the plane that intersect each other only pairwise and transversally, in such a way that $\BB\setminus s_f$ consists of a collection of regions $R_i$ homeomorphic to copies of $\BB$ along with a single annular region $R_\infty$ that does not intersect $f(M)$, and in such a way that $f(s_f)$ has a natural structure as a simple undirected 4--valent planar graph.
	We classify the critical values of $f$ as codimension 1 inside of arcs away from crossings, and codimension 2 at arc crossings.
	This classification makes the simple undirected planar graph structure of $f(s_f)$ explicit.
	The vertex set is given by the set of codimension 2 critical values and the edge set is the collection of codimension 1 critical values.
	
	We obtain a 4--manifold with boundary $M$ by attaching 2--handles corresponding to the regular regions $R_i$, 3--handles corresponding to codimension 1 singularities, and 4--handles corresponding to codimension 2 singularities.
	
	We will first focus on a region $R$ in $\BB$ that is not $R_\infty$.
	Begin by ``shrinking'' $R$ away from $f(s_f)$ into a space $\D$ that is homeomorphic to $\DD$
	Every point of $\D$ has preimage a disjoint collection of circles by the regular value theorem, so we centre our analysis on an arbitrary regular value $x$ and its regular fibers, where $M_{x,k}$ denotes the $k\nth$ regular fiber mapped over $x$ by $f$.
	An appropriate closed tubular neighbourhood of $M_{x,k}$ will consist entirely of regular fibers that map into $\D$ and has the structure of a closed 2--disc bundle over the circle, hence is a solid torus.
	The neighbourhood can be extended to enclose the entirety of $f\inv(\D)$.
	
	In summary, an arbitrary point $x_i$ is chosen from a closed 2--disc $D_i$ which itself is taken as a shrinking of an open region $R_i$.
	For each $k$, the $k\nth$ regular fiber $M_{i,k}$ over $x_i$ is taken as the zero section of a 2--disc bundle that maps over $D_i$.
	Such a bundle is an open solid torus, and the $k\nth$ bundle is denoted $V_{i,k}$.
	Of particular importance in $V_{i,k}$ are the zero section and a certain isotopy class of longitudes.
	The zero section $z(V_{i,k})$ is a regular fiber that maps over a point interior to $\D_i$, and the longitudinal isotopy class is the one that contains regular fibers that map to single boundary points of $D_i$.
	This pair determines an attaching map for 4--dimensional 2--handle attached over $V_{i,k}$.
	
	\begin{figure}
		\centering
		\captionsetup{justification=centering}
		\caption{A closed sleeve around an arc of codimension 1 critical values}
		\includegraphics[height=3cm]{figures/dummy.jpg}
		\label{fig:arcsleeve}
	\end{figure}
	
	Consider an arc $\alpha$ of codimension 1 critical values that separates a pair of regions $R_i$ and $R_j$ where either $i$ or $j$ may be $\infty$.
	As when we considered regions, $\alpha$ has been shrunk away from the codimension 2 critical values it connects.
	Let $\nbhd{\alpha}$ be a closed ``sleeve'' around $\alpha$ whose boundary intersects the boundaries of $D_i$ and $D_j$ as in Figure \ref{fig:arcsleeve}.
	The restriction of $f$ to the preimage of a linear transversal to $\alpha$ that connects $D_i$ and $D_j$ is, itself, a Morse function.
	The intersection with $\alpha$ is a critical point of this restriction, and the same techniques from the proof of Theorem \ref{thm:2bound3} may be applied to $f\inv(\nbhd{\alpha})$.
	A strand of singular fibers over $\alpha$ consists of an interval in the case of an extremum for the associated Morse function or an interval crossed with the figure 8 when the associated Morse function yields a saddle singularity.
	The critical neighbourhood $f\inv(\nbhd{\alpha})$ contains a connected component that is either $\DD\times\I$ when we have cross sectional extremum, or $P\times\I$, where $P$ is the pair of pants surface, i.e.\ the 2--sphere with three interior 2--balls removed, when the cross section yields a saddle.
	All remaining connected components are intervals crossed with regular annuli.
	We use $A=S^1\times\D^1$, alternatively the 2--sphere with two interior 2--balls removed, to denote an annulus, and find $A\times\I$ to be the form of all remaining connected components.
	
	Whenever one of the regions that $\alpha$ separates is $R_\infty$, the cross section necessarily gives us an extremum, and the critical neighbourhood over $\nbhd{\alpha}$ is $\DD\times\I$.
	We call $\alpha$ a \emph{definite fold} when the critical neighbourhood is $\DD\times\I$, and an \emph{indefinite fold} when the critical neighbourhood is $P\times\I$.
	
	\begin{figure}
		\centering
		\captionsetup{justification=centering}
		\caption{A closed sleeve around a codimension 2 critical value}
		\includegraphics[height=3cm]{figures/dummy.jpg}
		\label{fig:isolatesleeve}
	\end{figure}
	
	Let $x$ be a codimension 2 critical value and $\nbhd{x}$ the closed neighbourhoods whose boundary agrees with the shrunken regions and arc sleeves it is near as in Figure \ref{fig:isolatesleeve}.
	The possible singular fibers over $x$ are cataloged in \cite{Saeki}, and Figure \ref{fig:saekising} displays them.
	The singular fibers over $x$ may be disconnected.
	When that is the case, the fibers have the form seen in our codimension 1 analysis.
	Otherwise, the singular fiber has the shape of a 4--valent directed graph.
	The regular fibers are still copies of $S^1$.
	In each case $f\inv(\nbhd{x})$ is a 3--dimensional thickening of the corresponding fibers, hence is a disjoint collection of handlebodies of genus 0,1,2, or 3.
	The neighbourhood of a regular fiber is still a solid torus, so all but at most two of the connected components of $f\inv(\nbhd{x})$ will be genus 1 handlebodies.
	
	\begin{figure}
		\centering
		\captionsetup{justification=centering}
		\caption{Possible singular fibers of a proper generic smooth map from an orientable 3--manifold to a surface}
		\includegraphics[height=3cm]{figures/dummy.jpg}
		\label{fig:saekising}
	\end{figure}
	
	We may now begin describing a handle decomposition for a cobordism $W$ of the pair $(M,\emptyset)$.
	Begin with $M\times\I$, whose boundary is $M_0\sqcup M_1$ with $M_0=M\times\{0\}$ and $M_1 = M\times\{1\}$, and a proper generic smooth map $f:M_0\to\BB$.

	For each $i<\infty$ indexing the connected regions of $\BB\setminus f(s_f)$, we index over $k$ the connected components of $f\inv(R_i)$.
	Consider $D_i\subset R_i$, a 2--disc of regular values, and $V_{i,k}$, the $k\nth$ regular solid torus that maps over $D_i$.
	The solid torus $V_{i,k}$ has a trivial disc bundle structure over the regular fiber $M_{i,k}=z(V_{i,k})$, and $f(M_{i,k})=x$ in the interior of $\D_i$.
	Because we consider a single solid torus for the rest of this argument, we abbreviate $V_{i,k}$ to $V$, $M_{i,k}$ to $z(V)$ and $\D_i$ to $\D$.

	Let's remember what we're doing here --- we're attaching handles in such a way that, once all handles have been attached, the boundary component $M_0$ of $M\times\I$ has been filled in.
	In this step we attach 2--handles over $V$.
	Our goal is to do so in such a way that we remove the ``bad'' solid torus $V$ and replace it with a solid torus that is ``nice,'' where ``nice'' means that the newly introduced solid torus can be filled in by attaching 3-- and 4--handles.
	This happens by deleting $V$ and gluing 2--discs to the longitudes in $\pd V$ that are regular fibers of points in the boundary of $D$.
	To attach a 2--handle over $V$, we need
	\begin{enumerate}
		\item The isotopy class of an embedding $g:S^1\to M_0$, and
		\item the isotopy class of an identification $G:S^1\times\DD\to\nbhd{g(S^1)}$.
	\end{enumerate}
	The embedding $S^1\to M_0$ is easy enough to define as we will just be identifying $S^1$ with $z(V)$.
	This $S^1$ lives as the attaching sphere $S^1\times\{0\}$ inside of the 2--handle $\DD\times\DD$ that we plan to attach.
	Taking $\pd (\DD\times\DD)=(S^1\times\DD)\cup(\DD\times S^1)$ to be the genus 1 Heegard splitting of $S^3$, we will be defining a map $G:V\to S^1\times\DD$, where $S^1\times\DD$ is the first torus component.
	We call the other torus $V^*$.
	In order to satisfy our desired criteria, we need to have $G$ be in the isotopy class of a map that takes a regular fiber $J$ over a point in the boundary of $\D$ to $S^1\times\{1\}$.
	When this is true, we may form the adjunction $(M\times\I)\cup_{G\inv}(\DD\times\DD)$ and $J$ bounds a disc in $V^*$.
	Every fiber in the interior of $V$ now bounds a disc, and the new boundary is exactly $V^*$, whose meridians are the fibers that project over the boundary of $\D$.
	This is the property we wanted --- we've replaced the ``bad'' solid torus $V$ with the ``nice'' solid torus $V^*$.
	
	\begin{figure}
		\centering
		\captionsetup{justification=centering}
		\caption{The solid tori $V$ and $V^*$ with boundary curves $J$, $K$, and $L$.}
		\includegraphics[height=3cm]{figures/dummy.jpg}
		\label{fig:VV*}
	\end{figure}
	
	A class $[G]$ satisfying the above is unique and it exists.
	To see why, let $\varphi:V\to S^1\times\DD$ be any trivialization of $V$, take $K=\varphi\inv(\{1\}\times S^1)$ to represent the meridinal isotopy class of $V$, and take $L$ be the longitude in $\pd V$ defined by $L=\varphi\inv(S^1\times\{1\})$.
	Let $J=f\inv(q)$ in $\pd V$ represent the isotopy class of regular fibers over boundary points of $\D$, where $q$ is an arbitrary point in the boundary of $D$.	
	Figure \ref{fig:VV*} shows the curves $J$, $K$, and $L$ with orientations sitting on $\varphi(V)$.
	Thinking of $S^1\times\DD$ as a subset of $\C^2$ gives it a natural orientation, which we have pulled back through $\varphi$ onto $K$ and $L$.
	We give $J$ a compatible orientation so that its intersections with $K$ and $L$ can be counted.
	Put $\kappa$ to be the oriented intersection number of $J$ and $L$.
	Then $\varphi(J)\in \pd\varphi(V)$ is in the isotopy class of $h_M^{\kappa}(\varphi(L))$, where $h_M$ is the meridinal twist defined in Theorem \ref{thm:mpgV}.
	Take $G=H_M^{-\kappa}\comp\varphi$, where $H_M$ is the extension of $h_M$ to the solid torus defined in Remark \ref{rmk:2handle}.
	In the boundary of $G(V)$, $G(J)$ is in the isotopy class of $S^1\times\{1\}$, $G(K)$ is in the isotopy class of $\{1\}\times S^1$, and $G(L)=h_M^\kappa(S^1\times\{1\})$.
	Existence of $G$ is shown, and uniqueness up to isotopy comes from the same uniqueness in trivializations of $V$ and of $H_M$.
		
	With 2--handles attached, we move onto the preimages of arc sleeves.
	Let $\nbhd{\alpha}$ be an arc sleeve, and consider the connected components of $f\inv(\nbhd{\alpha})$.
	Figure \ref{fig:arcsleevepre} displays the possible connected components of arc sleeve preimages, each of which has the form of a surface crossed with the interval.
	The boundary circles of these surfaces at a cross section $\Sigma\times\{t\}$ project through $f$ over the boundaries of shrunken regions, so they are filled with discs that sit inside of attached 2--handles from the previous step.
	In each case, we obtain a copy of $S^2\times\D^1$ over which we attach a 3--handle.
	The further modification to $M_0$ that takes place when we attach 3--handles can be thought of as the deletion of a copy of $S^2\times D^1$ followed by gluing 3--discs over the newly created 2--sphere boundary components.
	
	\begin{figure}
		\centering
		\captionsetup{justification=centering}
		\caption{Possible connected components of arc sleeve preimages}
		\includegraphics[height=3cm]{figures/dummy.jpg}
		\label{fig:arcsleevepre}
	\end{figure}
	
	Finally, let $x$ be a codimension 2 critical value and let $\nbhd{x}$ be its sleeve.
	We find $x$ at the crossing of a pair of strands of codimension 1 critical values, and those strands are classified as definite or indefinite folds.
	The analysis of the codimension 2 critical value is then broken down into five cases.
	In four cases, we are able to extend the codimension 1 situation.
	This happens when the singular fiber is disconnected, called an \emph{uninteractive} singular fiber, or when one of crossing folds is definite.
	When both folds are indefinite and the singular fiber is connected, then the singular fiber is called \emph{interactive} and we defer to the analysis in \cite{CostThur08}.
	
	\begin{figure}
		\centering
		\captionsetup{justification=centering}
		\caption{Regular level surfaces projecting near a codimension 2 critical value}
		\includegraphics[height=3cm]{figures/dummy.jpg}
		\label{fig:regprojcodim2}
	\end{figure}
	
	First we look at the connected components of $f\inv(\nbhd{x})$ that do not contain a singular fiber over $x$.
	These connected components are made from regular fibers, i.e.\ circles, that project over $\nbhd{x}$, i.e. a 2--disc.
	As in the case of regions of regular values, they are solid tori $V_{x,k}$, using the same naming convention that has been established for solid tori that map over the regions $D_i$.
	Seeing $V_{x,k}$ as a 2--disc bundle over $f\inv(x)$, we can find an isotopy class of longitudes that project over single points in the boundary $\pd\nbhd{x}$.
	These longitudes bound discs introduced during 2-- and 3--handle attachment.
	The union of all such discs yields a solid torus $V_{x,k}^*$ in the boundary.
	The boundary component consisting of $V_{x,k}$ and $V_{x,k}^*$ is described by the Heegaard splitting $V_{x,k}\cup_\varphi V_{x,k}^*$ where $\varphi$ takes a longitude of $V_{x,k}$ to a meridian of $V_{x,k}^*$.
	This is the standard genus 1 Heegaard splitting of $S^3$, so we have a copy of $S^3\times\D^0$ over which we may attach a 4--handle.

	\begin{figure}
		\centering
		\captionsetup{justification=centering}
		\caption{Definite fold surfaces projecting near a codimension 2 critical value}
		\includegraphics[height=3cm]{figures/dummy.jpg}
		\label{fig:deffoldprojcodim2}
	\end{figure}

	The component that comes from an uninteractive definite fold is of the form $\DD\times\I$ as in the case of codimension 1 critical values.
	The 2--sphere boundary of this shape is filled with 2--discs from our 2-- and 3--handles, so this shape is the adjunction of a pair of 3--discs glued over their boundary.
	This is the genus 0 Heegaard splitting of $S^3$, so we have a copy of $S^3\times\D^0$ over which we may attach a 4--handle.
	
	\begin{figure}
		\centering
		\captionsetup{justification=centering}
		\caption{Indefinite uninteractive fold surfaces projecting near a codimension 2 critical value}
		\includegraphics[height=3cm]{figures/dummy.jpg}
		\label{fig:indeffoldprojcodim2}
	\end{figure}
	
	\begin{figure}
		\centering
		\caption{Destabilizing pairs for the pair of pants}
		\includegraphics[height=3cm]{figures/dummy.jpg}
		\label{fig:destabpants}
	\end{figure}
		
	An uninteractive singular fiber that comes from an indefinite fold is the figure 8, and the preimage $f\inv(\nbhd{x})$ is a closed tubular neighbourhood of the figure 8 graph, which is homeomorphic to the pair of pants crossed with an interval.
	Here, we find a Heegaard splitting of genus 2 with two destabilizing pairs, showing that the splitting is a splitting of $S^3$, and we can attach a 4--handle to it by Theorem \ref{thm:lilwald}.
	An appropriate choice of meridians in $f\inv(\nbhd{x})$ would be a pair of curves that run from cuff to waist along the inside of each of the legs in the pair of pants, crest the waist, then down the outside of the legs back to the cuff.
	Meridians for the solid filled by our 2-- and 3--handles would run around the cuffs of each leg.
	Figure \ref{fig:destabpants} illustrates the idea.
	
	\begin{figure}
		\centering
		\captionsetup{justification=centering}
		\caption{The surfaces resulting from interaction between definite and indefinite folds as they project near a codimension 2 critical value}
		\includegraphics[height=3cm]{figures/dummy.jpg}
		\label{fig:defindefconn}
	\end{figure}
	
	When an indefinite and definite fold interact, we get the same shapes found in the case of an uninteractive definite fold.
	From a 1--dimensional Morse function viewpoint, we witness a homotopy of a pair of peaks separated by a saddle come together into a single peak, eliminating the saddle.
	See Figure \ref{fig:defindefconn} for a demonstration.
	
	\begin{figure}
		\centering
		\caption{Indefinite interactive fold surfaces projecting near a codimension 2 critical value}
		\includegraphics[height=3cm]{figures/dummy.jpg}
		\label{fig:interactivefoldprojcodim2}		
	\end{figure}
	
	This leaves us with the case of interacting indefinite folds.
	The full analysis is covered in Section 4.4 of \cite{CostThur08}.
	They show that filling in $f\inv(\pd\,\nbhd{x})$ with 2--discs and gluing the two genus 3 handlebodies together over $f\inv(\pd\,\nbhd{x})$ results in a Heegard splitting of $S^3$ by explicitly finding 3 destabilizing pairs.
	We attach 4--handles over these 3--spheres.
	Attaching a 4--handle over a 3--sphere fills in the 3--sphere with a 4--disc, so these final modifications completely eliminate what is left of $M_0$.
	We are left with a handle decomposition for a cobordism of the pair $(M,\emptyset)$ built on top of $M_0$ in $M\times\I$.
\end{proof}

Let $f:M:\to\BB$ be a proper generic smooth function and $W$ a the cobordism of the pair $(M,\emptyset)$, both as per the construction in Theorem \ref{thm:3bound4}. 
Because the handle decomposition produced for $W$ consists of handles of index 2,3, and 4, the associated dual decomposition contains handles of degree 0,1, and 2.
Our goal is to obtain a combinatorial description of this decomposition, and we find this through the Stein complex.

Recall that the Stein complex for a Morse function on a surface was generically a 1--manifold, and we described the object combinatorially via a graph.
In the case of a proper generic smooth map $M\to\BB$, the Stein complex is generically a surface.
We define the Stein factorization and complex of $f$ as usual.
Let $\sim$ be an equivalence relation on $M$ defined by $p\sim q$ if and only if $f(p)=f(q)=x$ and $p$, $q$ are in the same connected component of $f\inv(x)$.
Then $f=g\comp h$ where $h$ is the quotient map $M\to M/\!\!\sim$ and $g$ takes a point in $f\inv(x)$ to $x$.
The complex $S=h(M)$ sits inside of $W$ in a way that suggests its use as an instruction set for a handle decomposition.
We make this more precise with some definitions.

\begin{defn}
	Let $W$ be an $n$--manifold and $X$ an $m$--manifold with $m<n$ embedded in $W$.
	We say that the embedded submanifold is \emph{locally flat} or that \emph{$X$ is locally flat in $W$} if, for every point $p\in X$, there is a chart $(U,f)$ about $p$ for which $f(U\cap X)$ sits inside of $\R^m\subset \R^n$.
\end{defn}

\begin{defn}
	Let $S$ be a compact topological space that is generically a 2--manifold.
	We call the set of points in $S$ at which $S$ fails to be a 2--manifold the \emph{singular set} of $S$, and denote it by $\Sing(S)$.
	In other words, we take $\Sing(S)$ to be the minimal subset of $S$ such that $S\setminus\Sing(S)$ is a (possibly disconnected) closed 2--manifold.
	The connected components of $\Sigma$ are called the \emph{regions} of $S$.
	Note that when $S$ is a 2--manifold with boundary, $\Sing(S)=\pd S$.
	The singular set of $S$ is generically a 1--manifold.
\end{defn}

\begin{theorem}
	Let $f$, $M$, $S$, and $W$ be as we defined above.
	Let $\Sigma = S\setminus\Sing(S)$ be a boundaryless 2--manifold.
	There is a map $\psi:S\into W$ such that $\psi(\Sigma)$ is locally flat in $W$.
\end{theorem}

\begin{proof}
	We simultaneously construct the Stein complex of $f$ and the map $\psi$.
	A disc $D$ of regular values of $f$ pulls back to a collection of solid tori.
	To obtain a handle decomposition for $W$, we first attached 2--handles over these tori.
	Let $H$ be the 2--handle attached over one such solid torus $V$.
	The boundary of $H$ is a 3--sphere with genus 1 Heegaard splitting consisting of $V$ and a dual torus $V^*$.
	It is useful to think of $H$ in terms of two possible structures as a $\DD$ bundle over $\DD$, where the base space, and zero section, is either the core or cocore of the handle.
	In both cases we restrict to the 1--sphere boundaries of the 2--disc fibers to get circle bundles over the core and cocore.
	When the base space is the core, the associated circle bundle is $V^*$.
	When the base space is the cocore, the associated circle bundle is $V$ and the fibres of this bundle are the circles $p_V = f\inv(p)\cap V$ for $p$ in $D$.
	Then $h$ is constant on a fibre $p_V$, and this immediately implies that $h(V)$ is a 2--disc subset of a region of $S$.
	The bundle structure of $V$ over the cocore of $H$ comes with a projection $\pi$, and we define $\psi$ to map $h(V)$ to the cocore of $H$ by defining $\psi(h(p_V))=\pi(p_V)$.
		
	\begin{figure}
		\centering
		\captionsetup{justification=centering}
		\caption{Stein complex over regular values}
		\includegraphics[height=3cm]{figures/dummy.jpg}
		\label{fig:regstein}
	\end{figure}
										
									These circle bundles share a boundary 2--torus $T=\pd V=\pd V^*$.
									Let $\mathcal{J}$ be the isotopy class of curves in $T$ that are the preimages of points in the boundary of $D$.
									Recall that a representative of $\mathcal{J}$ is a longitude of $V$ and a meridian of $V^*$ by construction.
									This means that some representative $J$ of $\mathcal{J}$ bounds a disc in $V^*$, and is a section over the boundary of the circle bundle over the core of $H$.
									We first examine $H$ as a 2--disc bundle over its cocore $\{\vec{0}\}\times\DD$, appearing as the zero section of the bundle.
									By 
									The cocore of $H_{i,k}^2$ is a disc embedded in $W$, .cocore of these 2--handles are 
									has $f\inv(D_i)$ a circle bundle over a disjoint collection of 2--discs, and the components of this collection collapse through $h$ to connected regions of regular levels in the Stein complex.
		
	\begin{figure}
		\centering
		\captionsetup{justification=centering}
		\caption{Stein complex over codimension 1 critical values}
		\includegraphics[height=3cm]{figures/dummy.jpg}
		\label{fig:codim1stein}
	\end{figure}	
	
	A sleeve around an arc of codimension 1 critical values (Figure \ref{fig:arcsleeve}) pulls back to a disjoint collection of surfaces crossed with the interval.
	In the case of the annulus crossed with an interval, $h$ sees $A\times\I$ as a bundle of preimage circles over $\I\times\I$ and collapses that bundle onto $\I\times\I$.
	Two ends of this strip already map through $\psi$ to the boundaries of a pair of cocores of 2--handles.
	In $S$, we get a copy of $\I\times\I$ bridging a pair of 2--discs from the previous step.
	We define $\psi$ on this copy of $\I\times\I$ to bridge the associated cocores in $W$.
	
	When the surface is a disc, there is an interval of boundary circles that map through $f$ to regular values, so $\psi$ is already defined on this circle.
	As we move inwards along concentric circles towards the central interval $\{0\}\times\I$ of $\DD\times\I$, $h$ collapses these circles to extend an existing region of the Stein complex.
	When we get to the core of the solid cylinder, $h$ takes $\{0\}\times\I$ to a boundary edge of the extended region.
	We label the portion of the boundary edge formed this way by $d$ for \emph{definite}, and this edge becomes part of $\Sing(S)$.
	The central interval corresponds to the cocore of the 3--handle we attached over $D^2\times\I\subset S^2\times\D^1$.
	We define $\psi$ on the definite boundary edge to go to the cocore of this 3--handle, and extend over the rest of the region in a smooth way.
	
	When the surface is a pair of pants, the boundary circles of the pants map through $\psi$ over the boundaries of 2--handle cocores, and those cocores correspond to regions of $S$.
	The interval of critical points in $P\times\I$ corresponds to the cocore of a 3--handle, as in the case of the 2--disc.
	The three regions of $S$ corresponding to the cocores discussed above extend to meet along an internal edge in $\Sing(S)$.
	The neighbourhood of an internal edge in $S$ looks like a ``three--page book,'' or a $Y$--shaped graph crossed with an interval.
	We take this internal edge to the cocore of the 3--handle attached over $P\times\I\subset S^2\times\D^1$ through $\psi$, and extend $\psi$ over the regions meeting this edge in a smooth way.

	\begin{figure}
		\centering
		\captionsetup{justification=centering}
		\caption{Stein complex of genus 1 handlebody of regular points projecting near codimension 2 critical values}
		\includegraphics[height=3cm]{figures/dummy.jpg}
		\label{fig:codim2steinregular}
	\end{figure}
	
	The only parts of the Stein complex we haven't investigated yet are the portions corresponding to the codimension 2 critical values of $f$.
	Let $x$ be a codimension 2 critical value and $\nbhd{x}$ its neighbourhood (Figure \ref{fig:isolatesleeve}).
	We've already classified the subspaces of $M$ that project over $\nbhd{x}$: they are handlebodies.
	Such a handlebody completes into $S^3$ in every case encountered during our construction of $W$, and over these 3--spheres we attached 4--handles.
	Let $G$ be a handlebody that projects over $\nbhd{x}$.
												
												
												\newpage
												{
													HERE BE DRAGONS
												}
												\newpage
												
												
													In the case that $G$ is of genus 1, 
													
													$G$ contains no critical points of $f$, and we have already seen how to give $G$ the structure of a circle bundle over a disc whose fibers are the preimages through $f$ of points in $\nbhd{x}$.
													Here, $h(G)$ is a 2--disc in a region of $S$, and fills a hole 
													The handlebody of genus 1 collapses through $h$ to fill a hole in a regular level, and in this way we cap off all $S^1$ boundary regions of the Stein complex with 2--discs.
													
													 and is of genus 0 or 2 when the folds over $x$ are not interactive.
												
													
													\begin{figure}
														\centering
														\captionsetup{justification=centering}
														\caption{Stein complex of uninteractive and interactive definite folds projecting near codimension 2 critical values}
														\includegraphics[height=3cm]{figures/dummy.jpg}
														\label{fig:codim2steindef}
													\end{figure}
													
													The genus 0 handlebodies occur in the case of definite folds, both interactive and uninteractive.
													In the uninteractive case, there is a section within an edge of the trivalent graph that is not labeled definite.
													We extend that region to its termination as in the previous case, labeling the terminating points as definite edges.
													When the definite fold interacts with an indefinite fold, the internal edge terminates in a vertex of the trivalent graph that is not labeled definite.
													As before, the regions are extended out their terminus and the resulting edges are labeled definite.
													In this case there is a vertex corresponding to the termination of a saddle singularity that we label definite.
													Both cases are considered in Figure \ref{fig:codim2steindef}
													
													\begin{figure}
														\centering
														\captionsetup{justification=centering}
														\caption{Stein complex of the uninteractive indefinite fold projecting near codimension 2 critical values}
														\includegraphics[height=3cm]{figures/dummy.jpg}
														\label{fig:codim2steinindefun}
													\end{figure}
													
													Genus 2 handlebodies are dealt with exactly as in the case of the pair of pants crossed with the interval.
													The hole is filled with the local polyhedral model corresponding to the edge, as seen in Figure \ref{fig:codim2steinindefun}.
													
													\begin{figure}
														\centering
														\captionsetup{justification=centering}
														\caption{Crossing saddle singular fiber}
														\includegraphics[height=3cm]{figures/dummy.jpg}
														\label{fig:crossingsaddles}
													\end{figure}
													
													When the handlebody is of genus 3, we are in the situations detailed in Section 4.4 of \cite{CostThur08}.
													The hole corresponding to $K_4$ is filled in with the polyhedral model corresponding to a vertex.
													This Stein complex is simple enough to justify, as it is simply what happens when a pair of saddles cross as in Figure \ref{fig:crossingsaddles}.
													
													\begin{figure}
														\centering
														\captionsetup{justification=centering}
														\caption{Rotating torus singular fiber}
														\includegraphics[height=3cm]{figures/dummy.jpg}
														\label{fig:doublecone}
													\end{figure}
													
													The final hole can be filled in with the nonstandard vertex--like compact topological space in \ref{fig:doublecone}.
													This justification is more difficult to see, but Figure \ref{fig:doublecone} attempts to demonstrate.
													We consider a torus and a Morse function given by a simple height map.
													By rotating the torus over a horizontal axis halfway between the punctures, we cause the singular fibers to eventually reside in the same critical level, and then separate again.
													
													
													
													
													
													
													
													
													
													
													
													
\end{proof}
												
												
												
													
													They are handlebodies of 
													A sleeve around a codimension 2 critical value $x$ pulls back to a disjoint collection of handlebodies in $M$.
													
													If we were to remove these portions from $S$, we would be left with 
													At this point the boundary of the Stein complex has several connected components.
													The components that intersect the definite edges of the boundary 
													is the disjoint union of a trivalent graph whose edges all intersect definite edges of the boundary along with a some copies of $S^1$ and some graphs of the shapes found in Figure \ref{fig:holegraphs}.
												
												
													
													\begin{figure}
														\centering
														\captionsetup{justification=centering}
														\caption{Boundary graphs in the Stein complex}
														\includegraphics[height=3cm]{figures/dummy.jpg}
														\label{fig:holegraphs}
													\end{figure}
												
												The Stein complex formed has the structure of a simple polyhedron almost everywhere.
												
												
												It is possible to replace this shape with an integer decorated simple polyhedron containing two internal vertices and a trivalent boundary graph matching the hole we need to fill, but we need to discuss integer decoration first.
												
												We now have a Stein complex that almost describes a handle decomposition of a cobordism of $(M,\emptyset)$.
												For each vertex we attach a 0--handle, for each edge we attach a 1--handle, and for each region we attach a 2--handle.
												The only piece of information missing is a framing coefficient that defines how to attach our 2--handles.
												Each region of the complex is be decorated with the framing coefficient for that handle determined during the proof of Theorem \ref{thm:3bound4}.
												
												\begin{figure}
													\centering
													\captionsetup{justification=centering}
													\caption{The integer decorated simple polyhedron that replaces the double cone vertex}
													\includegraphics[height=3cm]{figures/dummy.jpg}
													\label{fig:simplepoly}
												\end{figure}
												
												Now that we know what integer decoration means, we do away with the double cone vertex.
												We instead fill the final holes of the Stein complex with the simple polyhedron in Figure \ref{fig:simplepoly}.
												The boundary regions of the polyhedron are equipped with relative integers that modify the decorations associated to the regions they are joined to in the existing Stein complex.
												The internal region has an absolute integer to describe its 2--handle attachment.
												
												We are left with an integer decorated Stein complex of the map $f:M\to\BB$ that describes fully the dual handle decomposition recovered in Theorem \ref{thm:3bound4} for a cobordism of the pair $(M,\emptyset)$.
												
												
												
												
												
												
												
												
												
												
												
												
												
												
												
												
												
												
												
												
												
												
												
												
												
												
												
												
												

												
												
												
												Before we delve into the specifics of obtaining the Stein complex and how it provides a set of instructions used to build the dual decomposition, we define the 2--dimensional analogue to a graph that we use to describe the Stein complex combinatorially.
												
												\begin{defn}
													\begin{figure}
														\centering
														\caption{The three local polyhedral models}
														\includegraphics[height=3cm]{figures/dummy.jpg}
														\label{fig:localpoly}
													\end{figure}
													The three \emph{standard local polyhedral models} are the following 3 compact topological spaces, as they sit inside of $\R^3$, under the subspace topology.
													\begin{enumerate}
														\item the closed 2--disc $\D^2$.
														\item the product of the interval $\I$ with the $Y$--shaped graph $K_{1,3}$.
														\item the cone on the complete graph $K_4$.
													\end{enumerate}
													The models each have boundary.
													Being a surface, the 2--disc has a well defined boundary.
													In the local model $K_{1,3}\times I$, the boundary consists of the union of the sets $(K_{1,3}\times \{0\})$, $(K_{1,3}\times \{1\})$, and $(v(K_{1,3})\times I)$, where $v(K_{1,3})$ denotes the vertex set of the graph $K_{1,3}$.
													The cone on $K_4$ is defined as $K_4\times I /\sim$ where $(x,1)\sim (y,1)$ for every $x$, $y$ in $K_4$.
													When we define the cone this way the boundary is $K_4\times \{0\}$.
												\end{defn}
												
												\begin{defn}
													Let $P$ be a compact topological space.	
													If every point $p$ of $P$ has a neighbourhood homeomorphic to an open set in one of the standard local polyhedral models, then $P$ is a \emph{simple polyhedron}.
													The set of points which do not have a neighbourhood homeomorphic to an open set in $D^2$ form a 4--valent graph which we call the \emph{singular set} of $P$ and denote by $\Sing (P)$.
													The vertices and edges of $\Sing (P)$ are called the vertices and edges of $P$.
													We call the connected components of $P\setminus\Sing (P)$ the \emph{regions} of $P$.
													A simple polyhedron whose regions are homeomorphic to discs is \emph{standard}.
													
													If a point $p$ of $P$ has a neighbourhood homeomorphic to an open set containing a point in the boundary of one of our three local models, then $p$ is a \emph{boundary point} of $P$.
													The set of all boundary points of $P$ is the boundary of $P$ and is denoted by $\pd P$.
													A region of $P$ is \emph{internal} if its closure is disjoint from $\pd P$.
													If $\pd P$ is empty then $P$ is \emph{closed}.
												\end{defn}
												
												\begin{theorem}
													Let $M$ be a closed, orientable 3--manifold and $f:M\to\BB$ a generic proper smooth map.
													Then the Stein complex of $f$ has a canonical embedding in  is a simple polyhedron everywhere except for at most finitely many points.
												\end{theorem}
												
												
												
												\begin{lem}[Lemma 4.4 from \cite{CostThur08}]
													The simple polyhedron that is the Stein complex of $f$ subject to the conditions in the proof of Theorem \ref{thm:3bound4} is standard.
												\end{lem}
												
												
												\begin{theorem}	
													Let $W$ be a cobordism of the pair $(M,\emptyset)$ constructed in Theorem \ref{thm:3bound4} and $S$ the Stein complex of $f$.
													Then $S$ embeds in $W$ in such a way that the regions of $S$ are locally flat in $W$ and the map $h:M\to S$ of the Stein factorization of $f$ can be extended to a map $h:W\to S$.
												\end{theorem}
												
												\begin{rmk}
													The justification that the double cone can be replaced by the polyhedron in Figure \ref{fig:simplepoly} necessitates delving into the central object of study in \cite{CostThur08} and \cite{Turaev91}: the shadow surface.
													The Stein complex constructed in Theorem \ref{thm:3stein4} is a \emph{shadow} of both $M$ and the cobordism $W$ of $(M,\emptyset)$ whose dual decomposition Theorem \ref{thm:3bound4} describes.
													Moreover, the dual decomposition described follows a specific case of the Turaev Reconstruction Theorem that can be found in either \cite{CostThur08} or \cite{Turaev91}.
												\end{rmk}
