\label{alg:fillholes}
There are some holes in our shadow represented by digraphs whose edges are coloured $H$.
Whenever we work with one of these graphs, it will be named $X$.
The projection of any $X$ through $\pi$ is one of the copies of the directed cycle on 4 vertices introduced during interior vertex deletion in Section~\ref{alg:finalmod}.
Each of the four vertices of $\pi(X)$ is adjacent to a single edge, and those edges are projected onto by a finite number of strands in $S$ by $g:S\to\DD$.
At most one of these projecting strands per edge corresponds to a strand of singular points through $\pi$.
These singular strands are determined by $\pi$ and extend along the entirety of the edge of $T$ in which they lie, corresponding to edges or shrinking singularites in $S$.
This means if one of our graphs $X$ is adjacent to a strand of singular points, then it is adjacent to another strand from the same edge of $T$, hence these singular strands occur in pairs.

In every case, $X$ is one of five possible graphs which we classify by their number of simple directed cycles.
The paper on which this thesis is based contains a detailed justification of the form of polyhedron we glue to $S$ in order to fill these holes.
The five possible graphs are classified by the number of singular and nonsingular strands adjacent to $X$.

\{ FIGURE: THE FIVE POSSIBLE GRAPHS \}

\{ FIGURE: THE FOUR FILLING POLYHEDRA \}

The first case occurs when $X$ is adjacent to exactly two strands of shrinking singularities and one nonsingular strand.
Here, we colour the edges of $X$ by $F$ and let $X$ join the rest of the false boundary.

The second case occurs when $X$ is adjacent to no singular strands.
We fill in the hole with a disc whose boundary has triangulation $X$.

The third case occurs when $X$ is adjacent to exactly two strands of non--shrinking singularities and three nonsingular strands.
We fill in the hole with the model shadow edge.

The fourth case occurs when $X$ is adjacent to exactly four strands of non--shrinking singularites and four nonsingular strands.
We fill this hole with the model shadow vertex.

The fifth case occurs when $X$ is adjacent to exactly four strands of non--shrinking singularites and no nonsingular strands.
We will this hole with a shadow polyhedron that has exactly two vertices and adjust the gleams of the regions adjacent to where we glue this polyhedron slightly.
The polyhedron in this case is the shadow of $X$ as it sits in $S^3$.

All boundary components of $S$ are filled this way.
At the end of this algorithm we have a shadow of $T$.
Once the internal regions of $S$ are assigned gleams, we will be able to construct a triangulated 4--manifold.



