\begin{defn}
\label{def:manifold}
A separable Hausdorff topological space $X$ is an $n$--dimensional \emph{topological manifold} if, for every point $p$ in $X$, there is a neighbourhood $U$ of $p$ and a homeomorphism $f:U\to V$ for some open subset $V$ of the $n$--dimensional real half space $$\RNP=\{(x_1,\dots,x_{n})\in\RRN : x_n\geq 0\}.$$
We call the pair $(U,f)$ a \emph{chart}, and call a collection of charts $\{(U_\alpha,f_\alpha):\alpha\in A\}$ an \emph{atlas} if our collection of neighbourhoods $U_\alpha$ cover $X$.
The map $f_\alpha\comp f_\beta$ defined on $f_\beta(U_\alpha\cap U_\beta)$ is called the \emph{transition map} between the charts $(U_\alpha,f_\alpha)$ and $(U_\beta,f_\beta)$.
If the transition maps $f_\alpha\comp f_\beta$ are smooth maps between subsets of $\RNP$, then we call $X$ a \emph{smooth} $n$--dimensional manifold.
If the transition maps are piecewise--linear, then $X$ is a \emph{piecewise--linear} manifold

A point of $X$ that is mapped to the subspace $\{(x_1,\dots,x_n): x_n = 0\}$ by a chart is called a \emph{boundary point} of $X$.
The set of all boundary points of $X$, called the \emph{boundary of $X$}, is denoted by $\pd X$ and $\pd X$ is an $(n-1)$--dimensional submanifold of $X$.
A point $x$ in $X\setminus \pd X$ is called \emph{interior}.
If $X$ is compact and $\pd X=\emptyset$, then $X$ is said the be \emph{closed}.
It is easy to see that the $(n-1)$--dimensional submanifold $\pd X$ of $X$ is closed.

\end{defn}

Brief Digression On Orientations of $\RRN$.

\begin{defn}
\label{def:orient}

here we define tangent spaces and orientability of a manifold

\end{defn}

\begin{defn}
\label{def:isotopy}

A smooth \emph{isotopy} between embeddings $\varphi_0,\varphi_1:Y\to X$ is a smooth homotopy $\varphi_t: Y \to X$ ($0\leq t\leq 1$) through embeddings.
If an isotopy exists, we say that $\varphi_0,\varphi_1$ are \emph{isotopic}.

\end{defn}

brief digression on isotopy extension

\begin{theorem}[Isotopy Extension \{CITE: Miln\}]
\label{thm:isotopyextension}

Let $M$ be a smooth compact submanifold of the smooth closed manifold $N$.
If $h_t$ is a smooth isotopy of the inclusion $M\into N$ then $h_t$ is the restriction of a smooth isotopy $h_t':N\to N$ of the identity $\ident{N}$ such that $h_t'$ is the identity on $N$ outside of a compact subset of $N$.

\end{theorem}

We would like to restate the above theorem into a more useful form.
Let $Y\subset X$ be a smooth compact submanifold in the interior of the smooth manifold $X$ and let $i:Y\into X$ be inclusion.
Any smooth isotopy $\varphi_t:Y\to X$ with $\varphi_0=i$ induces an isotopy $\Phi_t:X\to X$ through diffeomorphisms $X\to X$ with $\Phi_0=\ident{X}$ and $\varphi_t=\Phi_t\comp\varphi_0$ for each $t$.
The isotopy $\Phi_t$ is called an \emph{ambient isotopy}, and a pair of submanifolds $Y_1,Y_2$ of $X$ are \emph{ambiently isotopic} is there exists a diffeomorphism $\Phi:X\to X$ such that $\Phi$ is homotopic to $\ident{X}$ through diffeomorphisms and $\Phi$ maps $Y_1$ to $Y_2$.
