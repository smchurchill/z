At its simplest, most colloquial definition, an $n$--dimensional manifold is a space that locally looks like real $n$--dimensional space or half space $\HN$.
We use \emph{charts} and \emph{atlases} to make explicit what is meant by ``looks like.''

\begin{defn}[Coordinates]
	\label{def:coordinates}
	Let $\HN\subset\RRN$ denote the closed real half space with the subspace topology, defined as
	\[
		\HN=\{(x_1,\dots,x_{n})\in\RRN : x_n\geq 0\}.
	\]
	We use the notations $\inter{\HN}$ and $\pd \HN$ to denote the interior and boundary of $\HN$ as subsets of $\RRN$ which, when $n>0$, are
	\[
		\begin{array}{ccccc}
			\inter{\HN} & = & \{(x_1,\dots,x_{n})\in\RRN : x_n > 0\}, & &\\
			\pd \HN	& = & \{(x_1,\dots,x_{n})\in\RRN : x_n = 0\} & \approx & \R^{n-1}.
		\end{array}
	\]
	When $n=0$, $\HH^0 = \R^0 = \{0\}$, so $\inter{\HH^0}=\R^0$ and $\pd\HH^0=\emptyset$.

	Let $X$ be a second-countable Hausdorff space.
	The pair $(U,f)$ where $U$ is an open subset of $X$ and $f$ is a homeomorphism from $U$ onto an open set of either $\RRN$ or $\HN$ is called a \emph{chart} of $X$.
	The map $f$ is a \emph{coordinate system} on $U$ and its inverse $f\inv$ is a \emph{parameterization} of $U$.
	Writing $f$ as
	\[
		f(u) = (\xi_1(u),\dots,\xi_n(u)),
	\]
	the functions $\xi_i$ are \emph{coordinate functions}
\end{defn}

\begin{defn}[Atlas]
	\label{def:atlas}
	Let $\mathcal{A}=\{(U_\alpha,f_\alpha)\}$ be a collection of charts of the space $X$ parameterized by $\alpha\in A$ for some indexing set $A$.
	If $\bigcup_A U_\alpha$ contains $X$, then $\mathcal{A}$ is an \emph{atlas} for $X$.
	The homeomorphisms $f_\alpha\comp f_\beta\inv:f_\beta(U_\alpha\cap U_\beta)\to f_\alpha(U_\alpha\cap U_\beta)$ are \emph{transition maps} of $\mathcal{A}$.
	We say that $(U_\alpha,f_\alpha)$ and $(U_\beta,f_\beta)$ are \emph{smoothly compatible} if either $U_\alpha\cap U_\beta$ is empty or the transition maps $f_\alpha\comp f_\beta\inv$ and $f_\beta\comp f_\alpha\inv$ are smooth as maps $\RRN\to\RRN$, i.e.\ they have continuous partial derivatives of all orders.
	A smooth homeomorphism with smooth inverse is called a \emph{diffeomorphism}.
	If every pair of charts in an atlas is smoothly compatible then that atlas is a \emph{smooth atlas}.
	Two smooth atlases are equivalent if their union is a smooth atlas.
\end{defn}

\begin{defn}[Manifolds]
	\label{def:manifold}
	Let $X$ be a second-countable Hausdorff topological space, and $\mathcal{A}$ an atlas for $X$.
	If $\mathcal{A}$ is a smooth atlas, then the pair $(X,\mathcal{A})$ is a \emph{smooth n--manifold}, or just \emph{n--manifold}.
	We omit writing the atlas when talking about a manifold.
	If $\mathcal{A}$ is a maximal smooth atlas then we call it a \emph{smooth structure} on $X$.
	We will assume that a smooth manifold is always equipped with a smooth structure.
	If $X$ is an $n$--manifold and $Y\subset X$ satisfies the definition of an $m$--manifold, then $Y$ is an $m$--dimensional \emph{submanifold} of $X$.
\end{defn}

\begin{defn}[Boundary, Interior]	
	\label{def:boundary}
	Let $X$ be a smooth $n$--manifold.
	A chart $(U,f)$ is called an \emph{interior chart} is $f(U)$ is an open subset of $\RRN$ and is a \emph{boundary chart} if $f(U)$ is an open subset of $\HN$ with $f(U)\cap\pd \HN\neq\emptyset$.
	Let $p\in X$.
	We say that $p$ is an \emph{interior point} if it is in the domain of an interior chart, and is a \emph{boundary point} if it is in the domain of a boundary chart $(U,f)$ so that $f(p)\in\pd\HN$.
	The set of all boundary points of $X$ is called the \emph{boundary of $X$} and is denoted by $\pd X$.
	Similarly, the set of all interior points of $X$ is called the \emph{interior} of $X$ and is denoted by $\inter{X}$.
	If $X$ is compact with empty boundary then $X$ is \emph{closed} as a manifold.
\end{defn}

\begin{prop}[Boundaries are Manifolds]
	\label{prop:boundariesaremanifolds}
	Let $X$ be an $(n+1)$--manifold.
	The boundary of $X$, $\pd X$, is an $n$--dimensional closed submanifold of $X$.
\end{prop}

\begin{proof}
	Let $\pd X\subset X$ be a second-countable Hausdorff space under the subspace topology.
	Let $\mathcal{A} = \{(U_\alpha, f_\alpha)\}$ be a smooth structure on $X$.
	There is a smooth atlas $\mathcal{B}$ for $\pd X$ that can be described in term of $\mathcal{A}$.
	For each $(U_\alpha,f_\alpha)$ in $\mathcal{A}$, there is a corresponding chart $(U_\beta,f_\beta)$ in $\mathcal{B}$ where $U_\beta = U_\alpha \cap \pd X$ and $f_\beta = \restr{f_\alpha}{U_\beta}$.
	Each $f_\beta$ is a homeomorphism whose domain is a set $U_\beta$ that is open in $\pd X$.
	Each point in $U_\beta$ is a boundary point and $f_\beta$ is a homeomorphism, so $f_\beta(U_\beta)$ is an open subset of $\pd \HH^{n+1} \approx\RRN$.
	The union of $U_\alpha$'s contains $X$, so the union of $U_\beta$'s contains $\pd X$, so $(\pd X, \mathcal{B})$ is a smooth $n$--manifold.
	Any chart $(U,f)$ of $\mathcal{B}$ has $f(U)$ an open subset of $\RRN$, so $(U,f)$ is an interior chart.
	We conclude that $\pd(\pd X)$ is empty hence $\pd X$ is a closed manifold. 
\end{proof}

Another important concept in the study of manifolds is that of the tangent space.
Manifolds are defined by their local homogeneity, and a tangent space is a precise description of that homogeneity near a point.

\begin{defn}[Tangent Space]
	\label{def:tangentspace}
	Let $X$ be an $n$--manifold, $p$ an interior point of $X$, $(U,f)$ a chart containing $p$ with $f(p)=\vec{0}$, $B_r^k$ the open ball of radius $r$ centred at $\vec{0}$ in $\R^k$, and $\gamma$ a map $$\gamma:B_r^k\to \inter{X}$$
	with $\gamma(0)=p$.
	We say that $\gamma$ is \emph{smooth} in a neighbourhood of $t$ in $B_r^k$ if $f\comp\gamma$ is smooth at $t$ as a map $\R^k\supset B_r^k\to\RRN$.
	When $k=1$ and $r$ is some small $\varepsilon$, $B_r^k$ is the interval $(-\varepsilon, \varepsilon)$.
	In this case, when $\gamma$ is smooth on the whole of the interval $(-\varepsilon, \varepsilon)$, such a $\gamma$ is called a \emph{curve} in $X$ through $p$.
	Note that this definition is independent of the chart used, as all of our transition maps are smooth.
	
	Let $C_p X$ be the space of smooth curves in $X$ through $p$ and $\gamma_1, \gamma_2$ elements of $C_p X$.
	We can define an equivalence relation $\sim$ on $C_p(X)$ by saying that $\gamma_1\sim\gamma_2$ if
	\[
		\Ddt(f\comp\gamma_1)(0) = \Ddt(f\comp\gamma_2)(0).
	\]
	An equivalence class of $C_p X/\sim$ is a \emph{tangent vector} at $p$, and the space $C_p X/\sim$ is the \emph{tangent space} at $p$, denoted $T_p X$.
\end{defn}

\begin{prop}
	\label{prop:tangentspacevectorspace}
	Let $X$ be an $n$--manifold and $p$ a point in $\inter{X}$.
	Then $T_p X$ is a vector space isomorphic to $\RRN$.
\end{prop}

\begin{proof}
	Let $(U,f)$ be a chart containing $p$ with $f(p)=\vec{0}$.
	For a vector $v$ of $\RRN$, let $L_v$ be the parameterized straight line through $\vec{0}$ whose velocity is $v$.
	Precisely, $L_v(t)=tv.$
	It follows from our definition of tangent space that the map defined by
	\[
		\begin{array}{cccc}
			df : & \RRN & \to & T_p X \\
			& v & \mapsto & [f\inv\comp L_v] .
		\end{array}
	\]
	is a bijection.
	We define the operation on $T_p X$ so that $F$ is strengthened to a vector space isomorphism:
	\[
		\begin{array}{ccc}
			df(v)+df(w)=df(v+w), & \hspace{10mm} \text{and} \hspace{10mm} & tdf(v)=df(tv).
		\end{array}
	\]
	The vector space structure of $T_p X$ is independent of the choice of chart.
	To see this, let $(U_1, f_1)$ and $(U_2, f_2)$ be charts with $f_1(p)=f_2(p)=\vec{0}$ and let $\phi=f_2\comp f_1\inv$ be their transition map.
	
\end{proof}

The main result of this work is applicable to orientable 3--manifolds, so we will review what is meant by a space being orientable.
Essentially, orientability is a guarantee that when you go for a walk your right and left sides haven't switched places by the time you get home.

\begin{defn}[Orientation]
	\label{def:orientation}
	Let $\RRN$ be $n$--dimensional real space, $\mathcal{B}(\RRN)$ the set of ordered bases for $\RRN$, and $\gl{n}{\R}$ the general linear group, i.e.\ the space of $n\times n$ invertible matrices with entries in $\R$.
	Let $b_1$ and $b_2$ be any two ordered bases from $\mathcal{B}(\RRN)$.
	It is a standard result of linear algebra that there is a unique element $A$ of $\gl{n}{\R}$ that transforms $b_1$ into $b_2$.
	If the determinant $\det A$ is positive, then $b_1$ and $b_2$ are \emph{positively oriented} with respect to each other.
	If $\det A$ is negative, then $b_1$ and $b_2$ are \emph{negatively oriented}.
	We can define an equivalence relation on $\mathcal{B}(\RRN)$ by saying that $b_1$ is equivalent to $b_2$ if they are positively oriented.
	An \emph{orientation} of $\RRN$ is a choice of one of the two equivalence classes of $\mathcal{B}(\RRN)$.
\end{defn}


