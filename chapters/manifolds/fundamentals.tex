At its simplest, most colloquial definition, an $n$--dimensional manifold is a space that locally looks like an $n$--dimensional real or half space.
We use \emph{charts} and \emph{atlases} to make explicit what is meant by ``looks like.''

\begin{defn}[Coordinates]
	\label{def:coordinates}
	Let $\HH\subset\R$ denote the closed real half space under the subspace topology, defined as $\HH=\{x\in\R : x\geq 0\}$.
	We use the notations $\inter{\HH}$ and $\pd \HH$ to denote the topological interior and boundary of $\HH$ as subsets of $\R$ which are $\{0\}$ and $\{x\in\R:x>0\}$.
	Our main use for $\HH$ is to take the product of it with a real vector space:
	\[
		\HH^k\times\R^{n-k}=\{(x_0,\dots,x_{n-1})\in\RRN : x_i\geq 0, i=0,\dots,k-1\}.
	\]
	When $k=1$, we use the notations $\inter{\HH\times\R^{n-1}}$ and $\pd \HH\times\R^{n-1}$ to denote the topological interior and boundary of $\HH\times\R^{n-1}$ as subsets of $\RRN$ which, when $n>0$, are
	\[
		\begin{array}{ccccc}
			\inter{\HH\times\R^{n-1}} & = & \{(x_0,\dots,x_{n-1})\in\RRN : x_0 > 0\}, & &\\
			\pd (\HH\times\R^{n-1)}	& = & \{(x_0,\dots,x_{n-1})\in\RRN : x_0 = 0\} & \approx & \R^{n-1}.
		\end{array}
	\]
	When $n=0$, $\HH^0 = \R^0 = \{0\}$, so $\inter{\HH^0}=\R^0$ and $\pd\HH^0=\emptyset$.

	Let $M$ be a second-countable Hausdorff space.
	The pair $(U,\varphi)$ where $U$ is an open subset of $M$ and $\varphi$ is a homeomorphism from $U$ onto an open set of either $\RRN$ or $\HH\times\R^{n-1}$ is called a \emph{chart} of $M$.
	The pair $(U,\varphi)$ where $U$ is an open subset of $M$ and $\varphi$ is a homeomorphism from $U$ onto an open set of $\HH^k\times\R^{n-k}$ is called a \emph{corner chart} of $M$.
	
	The map $\varphi$ is a \emph{coordinate system} on $U$ and its inverse $\varphi\inv$ is a \emph{parameterization} of $U$.
	Writing $\varphi$ as
	\[
		\varphi(u) = (\xi_0(u),\dots,\xi_{n-1}(u)),
	\]
	the functions $\xi_i$ are \emph{coordinate functions}
\end{defn}

\begin{defn}[Atlas]
	\label{def:atlas}
	Fix a nonnegative integer $n$.
	Let $\mathcal{A}=\{(U_\alpha,\varphi_\alpha):\alpha\in A\}$ be a collection of charts of $M$ such that the codomain for each chart is of the fixed dimension $n$.
	If $\bigcup_A U_\alpha$ contains $M$, then $\mathcal{A}$ is an \emph{atlas} for $M$.
	The homeomorphisms $\varphi_\alpha\comp \varphi_\beta\inv:\varphi_\beta(U_\alpha\cap U_\beta)\to \varphi_\alpha(U_\alpha\cap U_\beta)$ are \emph{transition maps} of $\mathcal{A}$.
	We say that $(U_\alpha,\varphi_\alpha)$ and $(U_\beta,\varphi_\beta)$ are \emph{smoothly compatible} if either $U_\alpha\cap U_\beta$ is empty or the transition maps $\varphi_\alpha\comp \varphi_\beta\inv$ and $\varphi_\beta\comp \varphi_\alpha\inv$ are smooth as maps $\RRN\to\RRN$, i.e.\ they have continuous partial derivatives of all orders.
	If every pair of charts in an atlas is smoothly compatible then that atlas is a \emph{smooth atlas}.
	Two smooth atlases are equivalent if their union is a smooth atlas.
\end{defn}

\begin{defn}[Manifold]
	\label{def:manifold}
	Let $M$ be a second-countable Hausdorff topological space, $\mathcal{A}$ an atlas for $M$, and $n$ the dimension of the codomain of each chart in $\mathcal{A}$.
	If $\mathcal{A}$ is a smooth atlas, then the pair $(M,\mathcal{A})$ is a \emph{smooth n--manifold},  \emph{n--manifold}, or just \emph{manifold}.
	If $\mathcal{A}$ is a maximal smooth atlas then we call it a \emph{smooth structure} on $M$.
\end{defn}

It eases the notational burden to assume that our manifolds are always equipped with a smooth structure.
This assumption allows us to omit writing an atlas when talking about a manifold.
	
\begin{defn}[Boundary]
	Let $M$ be a smooth $n$--manifold.
	A chart $(U,\varphi)$ is called an \emph{interior chart} if $\varphi(U)$ is an open subset of $\RRN$ and is a \emph{boundary chart} if $\varphi(U)$ is an open subset of $\HH\times\R^{n-1}$ with $\varphi(U)$ nontrivially intersecting $\pd \HH\times\R^{n-1}$.
	Let $p\in M$.
	We say that $p$ is an \emph{interior point} if it is in the domain of an interior chart, and is a \emph{boundary point} if it is in the domain of a boundary chart $(U,\varphi)$ such that $\varphi(p)$ lies in $\pd\HH\times\R^{n-1}$.
	The set of all boundary points of $M$ is called the \emph{boundary of $M$} and is denoted by $\pd M$.
	Similarly, the set of all interior points of $M$ is called the \emph{interior} of $M$ and is denoted by $\inter{M}$.
	If $M$ is compact with empty boundary then $M$ is \emph{closed} as a manifold.
\end{defn}

There is a potential conflict of this definition with the concept of topological closure.
In the rare case that we want to say that a manifold is topologically closed, we will specify that the type of closure is topological.
Otherwise, a ``closed manifold'' will refer to a manifold with empty boundary.

\begin{defn}[Manifolds with Corners]
	Let $M$ be a second-countable Hausdorff topological space.
	An atlas that permits corner charts is called an \emph{atlas with corners}, and equipping $M$ with a maximal atlas with corners produces a \emph{manifold with corners}.
	If $(M,\mathcal{A})$ is an atlas with corners such that $\mathcal{A}$ is a smooth atlas on $\inter{M}$, then $M$ is a \emph{smooth manifold with corners}.
	
	A corner chart $(U,\varphi)$ is a \emph{full corner chart} if $\varphi(U)$ is an open subset of $\HH^2\times\R^{n-2}$ and $\varphi(U)$ intersects the subspace 
	\[
		(\HH^2\times\R^{n-2})_c = \{(x_0,x_1,\dots,x_{n-1})\in\HH\times\HH\times\R^{n-2}:x_0=0,x_1=0\}
	\]
	nontrivially.
	
	Let $p\in M$.
	We say that $p$ is a \emph{corner point} if it is in the domain of a full corner chart $(U,\varphi)$ so that $\varphi(p)\in(\HH^2\times\R^{n-2})_c$.
	The set of all corner points of $M$ is called the \emph{corner(s) of $M$} and is denoted by $M_c$.
\end{defn}

\begin{ex}
	The $n$--sphere, denoted $S^n$ and defined as a subset of $\R^{n+1}$, is
	\[
	S^n = \{x\in\R^{n+1}:\norm{x}=1\}.
	\]
	The $(n+1)$--dimensional topologically open ball, or $(n+1)$--ball, defined as a subset of $\R^{n+1}$, is
	\[
	B^{n+1} = \{x\in\R^{n+1}:\norm{x}< 1\}.
	\]
	We use $\norm{\,\cdot\,}$ to denote the Euclidean norm defined for $x\in\R^{n+1}$ by
	\[
	\norm{x} = \Big( \sum x_i^2 \Big)^{1/2}.
	\]
	The space $\D^{n+1}$, the topological closure of $B^{n+1}$, is the closed $(n+1)$--disc.
	Both $B^{n+1}$ and $\D^{n+1}$ are $(n+1)$--manifolds.
	The disc $\D^{n+1}$ is a manifold whose boundary the $n$--sphere, a closed $n$--manifold.
	
	It is often more efficient, in terms of notation, to consider $S^1$, $\D^2$, $S^3$, and $D^4$ as submanifolds of the complex plane $\C$ or $\C^2$.
	We do exactly this in Theorem \ref{thm:mpgV} while defining maps $S^1\times\DD\to S^1\times\DD$.
	
	A regular $n$--gon in the plane is a 2--manifold with corners that is homeomorphic to $\DD$.
\end{ex}
