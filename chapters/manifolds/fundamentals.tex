At its simplest, most colloquial definition, an $n$--dimensional manifold is a space that locally looks like an $n$--dimensional real, half, or cornered space.
We use \emph{charts} and \emph{atlases} to make explicit what is meant by ``looks like.''

%%%%%%%%%%%%%%%%%%%%%%%%%%%%%%%%%%%%%
%	we can easily rework the definition to a more standard one
%	by that, i just mean that we define a manifold without boundary
%	then a manifold with boundary then a manifold with corners.
%	it only made sense to do it as it is now when we wanted to combine
%	boundaryless and boundarywith manifolds into one definition.
%
%	we would define a "manifold" as anything matching the criteria
%	(with or without boundary) as our later chapters were written to allow
%	these things to be defined for manifolds with boundary
%
%	check on tangent space...
%%%%%%%%%%%%%%%%%%%%%%%%%%%%%%%%%%%%%

\begin{defn}[Real, Half, and Cornered Spaces]
	Let $n>0$ and $0\leq k\leq n$ be integers.
	The \emph{$n$--dimensional real space} is denoted $\R^n$ and is defined by
	\[
		\R^n=\{(x_0,x_1,\dots x_{n-1}):x_i\in\R\textrm{ for each }i\},
	\]
	and the \emph{closed half space}, denoted $\HH$, is defined by
	\[
		\HH=\{x\in\R:x\geq 0\},
	\]
	and is a topological subspace of $\R$ endowed with the subspace topology.
	We use the notations $\inter{\HH}$ and $\pd \HH$ to denote the topological interior and boundary of $\HH$ as subsets of $\R$ which are $\{0\}$ and $\{x\in\R:x>0\}$ respectively.
	
	We take products of $\HH^k$ with $\R^{n-k}$ to form \emph{$n$--dimensional real space with a $k$--corner}.
	Explicitly, this space is
	\[
		\HH^k\times\R^{n-k} = \{(x_0,\dots,x_{n-1})\in\RRN:x_i\geq\textrm{ for } i=0,\dots,k-1\},
	\]
	with the origin $\zz=(0,\dots,0)$ called the \emph{model $k$-corner point}.
	
	We use the notations $\inter{\HH^k\times\R^{n-k}}$ and $\pd (\HH^k\times\R^{n-k})$ to denote the topological interior and boundary of $\HH^k\times\R^{n-k}$ as subsets of $\RRN$ which, when $n>0$, are
	\[
	\begin{array}{rcl}
	\inter{\HH^k\times\R^{n-1}} & = & \{(x_0,\dots,x_{n-1})\in\RRN : x_i > 0\textrm{ for }i=0,\dots,k-1\},\\
	\pd (\HH^k\times\R^{n-1})	& = & \{(x_0,\dots,x_{n-1})\in\RRN : x_i = 0\textrm{ for some }i\in\{0,\dots,k-1\}\}.
	\end{array}
	\]
	When we allow $n=0$, $\HH^0 = \R^0 = \{0\}$, so $\inter{\HH^0}=\R^0$ and $\pd\HH^0=\emptyset$.
	The \emph{dimension} of $\HH^k\times\R^{n-k}$ is $n$.
\end{defn}

\begin{defn}[Coordinates]
	\label{def:coordinates}
	Let $M$ be a second-countable Hausdorff space.
	The pair $(U,\varphi)$ where $U$ is an open subset of $M$ and $\varphi$ is a homeomorphism from $U$ onto an open set of $\HH^k\times\R^{n-k}$ is called a \emph{chart}.
	If $p$ is a point in $M$ and $(U,\varphi)$ is a chart such that $p\in U$, then $(U,\varphi)$ is a \emph{chart about $p$}.
	The map $\varphi$ is a \emph{coordinate system} on $U$ and its inverse $\varphi\inv$ is a \emph{parameterization} of $U$.
	Writing $\varphi$ as
	\[
		\varphi(u) = (\xi_0(u),\dots,\xi_{n-1}(u)),
	\]
	the functions $\xi_i$ are \emph{coordinate functions}	
\end{defn}

\begin{defn}[Atlas]
	\label{def:atlas}
	Fix a nonnegative integer $n$.
	Let $\mathcal{A}=\{(U_\alpha,\varphi_\alpha):\alpha\in A\}$ be a collection of charts of $M$ such that the codomain for each chart is of the fixed dimension $n$.
	If $\bigcup_A U_\alpha$ contains $M$, then $\mathcal{A}$ is an \emph{atlas} for $M$.
	
	First , recall that a map $f:\HH^{k_\alpha}\times\R^{n-k_\alpha}\to\HH^{k_\beta}\times\R^{n-k_\beta}$ is \emph{smooth} if it can be extended to a smooth map $\tilde{f}:\RRN\to\RRN$, i.e.\ if it can be extended to a map $\tilde{f}$ that has continuous partial derivatives of all orders.
	
	The homeomorphisms $\varphi_\alpha\comp \varphi_\beta\inv:\varphi_\beta(U_\alpha\cap U_\beta)\to \varphi_\alpha(U_\alpha\cap U_\beta)$ are \emph{transition maps} of $\mathcal{A}$.
	We say that $(U_\alpha,\varphi_\alpha)$ and $(U_\beta,\varphi_\beta)$ are \emph{smoothly compatible} if either $U_\alpha\cap U_\beta$ is empty or the transition maps $\varphi_\alpha\comp \varphi_\beta\inv$ and $\varphi_\beta\comp \varphi_\alpha\inv$ are smooth as maps $\HH^{k_\beta}\times\R^{n-k_\beta}\to\HH^{k_\alpha}\times\R^{n-k_\alpha}$ and  $\HH^{k_\alpha}\times\R^{n-k_\alpha}\to\HH^{k_\beta}\times\R^{n-k_\beta}$ respectively.

	If every pair of charts in an atlas is smoothly compatible then that atlas is a \emph{smooth atlas}.
	Two smooth atlases are equivalent if their union is a smooth atlas.
\end{defn}

\begin{defn}[Manifold]
	\label{def:manifold}
	Let $M$ be a second-countable Hausdorff topological space, $\mathcal{A}$ an atlas for $M$, and $n$ the dimension of the codomain of each chart in $\mathcal{A}$.
	If $\mathcal{A}$ is a smooth atlas, then the pair $(M,\mathcal{A})$ is a \emph{smooth n--manifold},  \emph{n--manifold}, or just \emph{manifold}.
	If $\mathcal{A}$ is a maximal smooth atlas then we call it a \emph{smooth structure} on $M$.
\end{defn}

It eases the notational burden to assume that our manifolds are always equipped with a smooth structure.
This assumption allows us to omit writing an atlas when talking about a manifold.
We also assume that our manifolds are connected.

We classify the points of a manifold by the charts that exist around them.

\begin{defn}
	Let $M$ be a smooth $n$--manifold, $p\in M$, and $(U,\varphi)$ a chart about $p$.
	Consider $\varphi(p)=(x_0,\dots,x_{n-1})\in \HH^k\times\R^{n-k}$.
	Then $p$ is classified as a \emph{$j$--corner point}, where $j$ is the number of entries $x_0,\dots,x_{k-1}$ that are 0.
	For $p$ a $j$--corner, we define the \emph{depth} of $p$ by $\dep{p}=j$.
	By smooth compatibility of our charts, examining the image of $p$ through exactly one chart is sufficient this classification.
	
	To coincide with the topological notion of interior and boundary, we call a $0$--corner an \emph{interior point}, and a $j$--corner is a \emph{boundary point} for $j>0$.
	For $j>0$, we use $\pd_j M$ to denote the set of points $\{p\in M:p\textrm{ is a }j\textrm{--corner}\}$, and call $\pd_j M$ the \emph{$j$--boundary} of $M$.
	The union $$\bigcup_{j>1} \pd_j M$$ is denoted $\pd M$ and is called the \emph{boundary} of $M$.
\end{defn}	

This classification let's us define the subclass of smooth manifolds with corners that is most useful to us.

\begin{defn}
	Let $M$ be a smooth $n$--manifold, and $\pd_1(M)$ the 1--boundary of $M$.
	A \emph{face} of $M$ is the closure of a connected component of $\pd_1 M$, and we denote the set of faces of $M$ by $\pd_f M$.
	We say that $M$ is a \emph{manifold with faces} if, for every $j$--corner point $p\in M$, $p$ belongs to exactly $j$ different faces of $M$,
\end{defn}

There are several advantages to considering the subclass of manifolds with faces.
First, if $M$ is a manifold with faces then any connected face or disjoint union of connected faces of $M$ is, itself, a manifold with faces.
Second, if $M$ is a manifold with faces then so is $M\times \HH$.
	
\begin{defn}
	If $M$ is a compact smooth manifold and every point of $M$ is interior, then $M$ is \emph{closed} as a manifold.
\end{defn}

There is a potential conflict of this definition with the concept of topological closure.
In the rare case that we want to say that a manifold is topologically closed, we will specify that the type of closure is topological.
Otherwise, a ``closed manifold'' will refer to a manifold with empty boundary.

\begin{ex}
	The $n$--sphere, denoted $S^n$ and defined as a subset of $\R^{n+1}$, is
	\[
	S^n = \{x\in\R^{n+1}:\norm{x}=1\}.
	\]
	The $(n+1)$--dimensional topologically open ball, or $(n+1)$--ball, defined as a subset of $\R^{n+1}$, is
	\[
	B^{n+1} = \{x\in\R^{n+1}:\norm{x}< 1\}.
	\]
	We use $\norm{\,\cdot\,}$ to denote the Euclidean norm defined for $x\in\R^{n+1}$ by
	\[
	\norm{x} = \Big( \sum x_i^2 \Big)^{1/2}.
	\]
	The space $\D^{n+1}$, the topological closure of $B^{n+1}$, is the closed $(n+1)$--disc.
	Both $B^{n+1}$ and $\D^{n+1}$ are $(n+1)$--manifolds.
	The disc $\D^{n+1}$ is a manifold whose boundary is the $n$--sphere, a closed $n$--manifold.
	Note that the closed 1--disc is the unit interval, which we denote by $\I=[\,0,1\,]$.
	
	It is often more efficient, in terms of notation, to consider $S^1$, $\D^2$, $S^3$, and $D^4$ as submanifolds of the complex plane $\C$ or $\C^2$.
	We do exactly this in Theorem \ref{thm:mpgV} while defining maps $S^1\times\DD\to S^1\times\DD$.
\end{ex}

We use the following surfaces extensively in the decomposition of more complicated 2--manifolds.

\begin{defn}
	\label{def:annulus}
	An \emph{annulus} is any surface homeomorphic to $S^1\times\I$.
	An annulus's boundary consists of two components that are homeomorphic to $S^1$.
	Another description of an annulus is as a 2--disc minus an open ball.
\end{defn}

\begin{defn}
	\label{def:pants}
	A \emph{pair of pants} or \emph{pants surface} is a surface that is homeomorphic to a 2--disc minus two disjoint open balls.
	The boundary of a pair of pants consists of three components that are homeomorphic to $S^1$.
\end{defn}

\begin{ex}
	The unit square $\I\times \I=\I^2$ is a 2--manifold with faces that is homeomorphic to $\DD$.
	The 1--boundary is 
	\[\pd_1(\I\times\I) = \big((0,1)\times\{0\}\big)\cup\big((0,1)\times\{1\}\big)\cup\big(\{0\}\times(0,1)\big)\cup\big(\{1\}\times(0,1)\big),\]
	which yields connected faces of
	\[
		\begin{array}{cccc}
			\I\times\{0\}, & \I\times\{1\}, & \{0\}\times\I, \textmd{ and}&\{1\}\times\I.
		\end{array}
	\]
	The 2--corners of $\I\times\I$ are the points $(0,0)$, $(0,1)$, $(1,0)$, and $(1,1)$.

	The product $\I\times\I\times S^1$ is also a manifold with faces.
	It's faces are exactly the products of the faces of $\I\times\I$ with $S^1$.
	We also obtain the 2--corners of $\I\times\I\times S^1$ by taking the product of the 2--corners of $\I\times\I$ with $S^1$.
	
	These results are due to the fact that $S^1$ is a closed manifold.
	For contrast, consider $\I\times\I\times\I=\I^3$, the unit cube:
	\begin{enumerate}
		\item The set of 1--corners of $\I^3$ is the union of $\pd_0(\I\times\I)\times\pd_1\I$ with $\pd_1(\I\times\I)\times\pd_0\I$.
		\item The set of 2--corners of $\I^3$ is the union of $\pd_0(\I\times\I)\times\pd_2\I$ with $\pd_1(\I\times\I)\times\pd_1\I$ and $\pd_2(\I\times\I)\times\pd_0\I$.
		\item The set of 3--corners of $\I^3$ is the union of $\pd_1(\I\times\I)\times\pd_2\I$ with $\pd_2(\I\times\I)\times\pd_1\I$.
	\end{enumerate}	
	This pattern generalizes.
	If $X$ and $M$ are manifolds with faces then the $k$--corners of $X\times M$ are found as
	\[
		\pd_k(X\times M) = \bigcup_{\substack{i+j=k\\i,j\,\geq0}} \pd_i X\times \pd_j M,
	\]
	and the set of faces of $X\times M$ is the union $(\pd_0 X\times\pd_f M)\cup(\pd_f X\times\pd_0 M)$.
\end{ex}

At times it is necessary to form a \emph{partition of unity} on a manifold.
To do this, we first require a suitable atlas.

\begin{defn}
	\label{def:adequate}
	Let $M$ be a manifold and $\mathcal{A}=\{(U_\alpha,\varphi_\alpha)\}$ an atlas on $M$.
	Then $\mathcal{A}$ is \emph{adequate} if it is locally finite, the image of $U_\alpha$ through $\varphi_\alpha$ is the whole of $H^k\times\R^{n-k}$ for the appropriate value of $k$, and the union 
	\[
		\bigcup_\alpha \varphi_\alpha\inv(B^n)
	\]
	covers $M$.
\end{defn}

\begin{theorem}[Theorem I.2.2 in \cite{Kosi93}]
	Let $\mathcal{V}=\{V_\beta\}$ be a covering of the manifold $M$.
	Then there is an adequate atlas $\{U_\alpha,\varphi_\alpha\}$ such that $\{U_\alpha\}$ is a refinement of $\mathcal{V}$
\end{theorem}

\begin{defn}
	Let $M$ be an $n$--manifold, let $\{U_\alpha,\varphi_\alpha\}$ be an adequate atlas on $M$, and let $\lambda$ be a smooth nonnegative function on $\R^n$ that takes the value of $1$ on the disc of radius $1$ and is zero outside of the disc of radius $2$.
	Define the smooth functions $\lambda_\alpha$ on $M$ by $\lambda_\alpha=\lambda\comp\varphi_\alpha$ on $U_\alpha$ and $\lambda_\alpha=0$ outside of $U_\alpha$.
	Then $\{\lambda_\alpha\}$ is \emph{the family of bump functions} associated to $\{U_\alpha\}$, and the family $\{\mu_\alpha\}$ of functions defined by
	\[
		\mu_\alpha(p)=\frac{\lambda_\alpha(p)}{\sum_\alpha\lambda_\alpha(p)}
	\]
	is the \emph{partition of unity} associated to $\{U_\alpha,\varphi_\alpha\}$.
\end{defn}

We conclude this section by abbreviating our terminology.
An \emph{$n$--manifold} refers to a \emph{smooth $n$--dimensional manifold with faces}.
In other words, ``manifold'' refers to the broadest definition of the term.
We may get more specific by saying \emph{$M$ is a closed manifold}, and in this case every point of $M$ is a 0--corner, and $M$ is compact.


