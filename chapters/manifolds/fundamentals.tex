At its simplest, most colloquial definition, an $n$--dimensional manifold is a space that locally looks like an $n$--dimensional real, half, or cornered space.
We use \emph{charts} and \emph{atlases} to make explicit what is meant by ``looks like.''

%%%%%%%%%%%%%%%%%%%%%%%%%%%%%%%%%%%%%
%	we can easily rework the definition to a more standard one
%	by that, i just mean that we define a manifold without boundary
%	then a manifold with boundary then a manifold with corners.
%	it only made sense to do it as it is now when we wanted to combine
%	boundaryless and boundarywith manifolds into one definition.
%
%	we would define a "manifold" as anything matching the criteria
%	(with or without boundary) as our later chapters were written to allow
%	these things to be defined for manifolds with boundary
%
%	check on tangent space...
%%%%%%%%%%%%%%%%%%%%%%%%%%%%%%%%%%%%%

\begin{defn}[Real, Half, and Cornered Spaces]
	Let $n>0$ and $0\leq k\leq n$ be integers.
	The \emph{$n$--dimensional real space} is denoted $\R^n$ and is defined by
	\[
		\R^n=\{(x_0,x_1,\dots x_{n-1}):x_i\in\R\textrm{ for each }i\},
	\]
	and the \emph{closed half space}, denoted $\HH$, is defined by
	\[
		\HH=\{x\in\R:x\geq 0\},
	\]
	and is a topological subspace of $\R$ endowed with the subspace topology.
	We use the notations $\inter{\HH}$ and $\pd \HH$ to denote the topological interior and boundary of $\HH$ as subsets of $\R$ which are $\{0\}$ and $\{x\in\R:x>0\}$ respectively.
	
	We take products of $\HH^k$ with $\R^{n-k}$ to form \emph{$n$--dimensional real space with a $k$--corner}.
	Explicitly, this space is
	\[
		\HH^k\times\R^{n-k} = \{(x_0,\dots,x_{n-1})\in\RRN:x_i\geq\textrm{ for } i=0,\dots,k-1\},
	\]
	with the origin $\zz=(0,\dots,0)$ the \emph{model $k$-corner}.
	
	We use the notations $\inter{\HH^k\times\R^{n-k}}$ and $\pd (\HH^k\times\R^{n-k})$ to denote the topological interior and boundary of $\HH^k\times\R^{n-k}$ as subsets of $\RRN$ which, when $n>0$, are
	\[
	\begin{array}{rcl}
	\inter{\HH^k\times\R^{n-1}} & = & \{(x_0,\dots,x_{n-1})\in\RRN : x_i > 0\textrm{ for }i=0,\dots,k-1\},\\
	\pd (\HH^k\times\R^{n-1})	& = & \{(x_0,\dots,x_{n-1})\in\RRN : x_i = 0\textrm{ for some }i\in\{0,\dots,k-1\}\}.
	\end{array}
	\]
	When we allow $n=0$, $\HH^0 = \R^0 = \{0\}$, so $\inter{\HH^0}=\R^0$ and $\pd\HH^0=\emptyset$.
	The \emph{dimension} of $\HH^k\times\R^{n-k}$ is $n$.
\end{defn}

\begin{defn}[Coordinates]
	\label{def:coordinates}
	Let $M$ be a second-countable Hausdorff space.
	The pair $(U,\varphi)$ where $U$ is an open subset of $M$ and $\varphi$ is a homeomorphism from $U$ onto an open set of $\HH^k\times\R^{n-k}$ is called a \emph{chart}.
	If $p$ is a point in $M$ and $(U,\varphi)$ is a chart such that $p\in U$, then $(U,\varphi)$ is a \emph{chart about $p$}.
	The map $\varphi$ is a \emph{coordinate system} on $U$ and its inverse $\varphi\inv$ is a \emph{parameterization} of $U$.
	Writing $\varphi$ as
	\[
		\varphi(u) = (\xi_0(u),\dots,\xi_{n-1}(u)),
	\]
	the functions $\xi_i$ are \emph{coordinate functions}	
\end{defn}

\begin{defn}[Atlas]
	\label{def:atlas}
	Fix a nonnegative integer $n$.
	Let $\mathcal{A}=\{(U_\alpha,\varphi_\alpha):\alpha\in A\}$ be a collection of charts of $M$ such that the codomain for each chart is of the fixed dimension $n$.
	If $\bigcup_A U_\alpha$ contains $M$, then $\mathcal{A}$ is an \emph{atlas} for $M$.
	
	The homeomorphisms $\varphi_\alpha\comp \varphi_\beta\inv:\varphi_\beta(U_\alpha\cap U_\beta)\to \varphi_\alpha(U_\alpha\cap U_\beta)$ are \emph{transition maps} of $\mathcal{A}$.
	We say that $(U_\alpha,\varphi_\alpha)$ and $(U_\beta,\varphi_\beta)$ are \emph{smoothly compatible} if either $U_\alpha\cap U_\beta$ is empty or the transition maps $\varphi_\alpha\comp \varphi_\beta\inv$ and $\varphi_\beta\comp \varphi_\alpha\inv$ are smooth as maps $\HH^{k_\beta}\times\R^{n-k_\beta}\to\HH^{k_\alpha}\times\R^{n-k_\alpha}$ and  $\HH^{k_\alpha}\times\R^{n-k_\alpha}\to\HH^{k_\beta}\times\R^{n-k_\beta}$ respectively.
	
	A map $f:\HH^{k_\alpha}\times\R^{n-k_\alpha}\to\HH^{k_\beta}\times\R^{n-k_\beta}$ is \emph{smooth} if it can be extended to a smooth map $\tilde{f}:\RRN\to\RRN$, i.e.\ if it can be extended to a map $\tilde{f}$ that has continuous partial derivatives of all orders.
	
	If every pair of charts in an atlas is smoothly compatible then that atlas is a \emph{smooth atlas}.
	Two smooth atlases are equivalent if their union is a smooth atlas.
\end{defn}

\begin{defn}[Manifold]
	\label{def:manifold}
	Let $M$ be a second-countable Hausdorff topological space, $\mathcal{A}$ an atlas for $M$, and $n$ the dimension of the codomain of each chart in $\mathcal{A}$.
	If $\mathcal{A}$ is a smooth atlas, then the pair $(M,\mathcal{A})$ is a \emph{smooth n--manifold},  \emph{n--manifold}, or just \emph{manifold}.
	If $\mathcal{A}$ is a maximal smooth atlas then we call it a \emph{smooth structure} on $M$.
\end{defn}

It eases the notational burden to assume that our manifolds are always equipped with a smooth structure.
This assumption allows us to omit writing an atlas when talking about a manifold.

We classify the points of a manifold by the charts that exist around them.

\begin{defn}
	Let $M$ be a smooth $n$--manifold, $p\in M$, and $(U,\varphi)$ a chart about $p$.
	Consider $\varphi(p)=(x_0,\dots,x_{n-1})\in \HH^k\times\R^{n-k}$.
	We call $p$ a \emph{$j$--corner point}, where $j$ is the number of entries $x_0,\dots,x_{k-1}$ that are 0.
	By smooth compatibility of our charts, examining the image of $p$ through exactly one chart is sufficient for classification in this way.
	
	We call a $0$--corner an \emph{interior point}, and for $j>0$, a $j$--corner is a \emph{boundary point}.	
\end{defn}
	
\begin{defn}[Boundary]
	\label{def:boundary}
	Let $M$ be a smooth $n$--manifold.
	A chart $(U,\varphi)$ is called an \emph{interior chart} if $\varphi(U)$ is an open subset of $\RRN$ and is a \emph{boundary chart} if $\varphi(U)$ is an open subset of $\HH\times\R^{n-1}$ with $\varphi(U)$ nontrivially intersecting $\pd \HH\times\R^{n-1}$.
	Let $p\in M$.
	We say that $p$ is an \emph{interior point} if it is in the domain of an interior chart, and is a \emph{boundary point} if it is in the domain of a boundary chart $(U,\varphi)$ such that $\varphi(p)$ lies in $\pd\HH\times\R^{n-1}$.
	The set of all boundary points of $M$ is called the \emph{boundary of $M$} and is denoted by $\pd M$.
	Similarly, the set of all interior points of $M$ is called the \emph{interior} of $M$ and is denoted by $\inter{M}$.
	If $M$ is compact with empty boundary then $M$ is \emph{closed} as a manifold.
\end{defn}

There is a potential conflict of this definition with the concept of topological closure.
In the rare case that we want to say that a manifold is topologically closed, we will specify that the type of closure is topological.
Otherwise, a ``closed manifold'' will refer to a manifold with empty boundary.

\begin{defn}[Manifolds with Corners]
	Let $M$ be a second-countable Hausdorff topological space.
	An atlas that permits corner charts is called an \emph{atlas with corners}, and equipping $M$ with a maximal atlas with corners produces a \emph{manifold with corners}.
	If $(M,\mathcal{A})$ is an atlas with corners such that $\mathcal{A}$ is a smooth atlas on $\inter{M}$, then $M$ is a \emph{smooth manifold with corners}.
	
	A corner chart $(U,\varphi)$ is a \emph{full corner chart} if $\varphi(U)$ is an open subset of $\HH^2\times\R^{n-2}$ and $\varphi(U)$ intersects the subspace 
	\[
		(\HH^2\times\R^{n-2})_c = \{(x_0,x_1,\dots,x_{n-1})\in\HH\times\HH\times\R^{n-2}:x_0=0,x_1=0\}
	\]
	nontrivially.
	
	Let $p\in M$.
	We say that $p$ is a \emph{corner point} if it is in the domain of a full corner chart $(U,\varphi)$ so that $\varphi(p)\in(\HH^2\times\R^{n-2})_c$.
	The set of all corner points of $M$ is called the \emph{corner(s) of $M$} and is denoted by $M_c$.
	The interior and boundary of $M$ are defined as in Definition \ref{def:boundary}.
	Note that $\pd M\cup M_c=M\setminus\inter{M}$.
\end{defn}

\begin{ex}
	The $n$--sphere, denoted $S^n$ and defined as a subset of $\R^{n+1}$, is
	\[
	S^n = \{x\in\R^{n+1}:\norm{x}=1\}.
	\]
	The $(n+1)$--dimensional topologically open ball, or $(n+1)$--ball, defined as a subset of $\R^{n+1}$, is
	\[
	B^{n+1} = \{x\in\R^{n+1}:\norm{x}< 1\}.
	\]
	We use $\norm{\,\cdot\,}$ to denote the Euclidean norm defined for $x\in\R^{n+1}$ by
	\[
	\norm{x} = \Big( \sum x_i^2 \Big)^{1/2}.
	\]
	The space $\D^{n+1}$, the topological closure of $B^{n+1}$, is the closed $(n+1)$--disc.
	Both $B^{n+1}$ and $\D^{n+1}$ are $(n+1)$--manifolds.
	The disc $\D^{n+1}$ is a manifold whose boundary the $n$--sphere, a closed $n$--manifold.
	Note that the closed 1--disc is the unit interval, which we denote by $\I=[\,0,1\,]$.
	
	It is often more efficient, in terms of notation, to consider $S^1$, $\D^2$, $S^3$, and $D^4$ as submanifolds of the complex plane $\C$ or $\C^2$.
	We do exactly this in Theorem \ref{thm:mpgV} while defining maps $S^1\times\DD\to S^1\times\DD$.
\end{ex}

\begin{ex}
	A regular $n$--gon, $G^n$, in the plane is a 2--manifold with corners $G_c^n=\{z\in\C:z^n=1\}$ that is homeomorphic to $\DD$.
	Included in this example is the unit square $\I\times \I$.
	
	Taking the product $G^n\times S^1$ yields a manifold with corners that is homeomorphic to $D^2\times S^1$ and whose corners are exactly $G^n_c\times S^1$.	
\end{ex}


