Important tools in both constructing and describing manifolds are \emph{bundles}.
The ultimate construction in this section is the tubular neighbourhood of a submanifold, an object that is unique up to a fiber-preserving isotopy.

\begin{defn}[Bundle]
	A real \emph{vector bundle} is a tuple $\beta = (E,B,\pi:E\to B)$ where $B$ is called the \emph{base space}, $E$ the \emph{total space}, and $\pi$ the \emph{projection}.
	The projection is a continuous map for which the subspaces $\pi\inv(b)=V_b$ all have the structure of a $k$--dimensional vector space.
	The space $V_b$ is \emph{fiber} over $b$.
	
	A bundle is \emph{locally trivial}.
	That is, for every point $b\in B$ there exists an open neighbourhood $U\subset B$ containing $b$ and a homeomorphism
	\[
		\varphi: U\times \RRN\to \pi\inv(U)
	\]
	such that the map $v\mapsto\varphi(b,v)$ is a vector space isomorphism $\RRN\mapsto V_b$ for every $b\in U$.
	Such a homeomorphism is called a \emph{local trivialization}, and such a pair is a \emph{local coordinate system}.
	Any pair of local trivializations
	\[
		\begin{array}{cccc}
			\varphi_U: & U\times\RRN & \to & \pi\inv(U), \\
			\varphi_V: & V\times\RRN & \to & \pi\inv(V)
		\end{array}
	\]
	must be compatible in the sense that the composition 
	\[
		\varphi_V\inv\comp\varphi_U : (U\cap V)\times\RRN\to (U\cap V)\times\RRN,
	\]
	is well defined on $U\cap V$ and satisfies
	\[
		\varphi_V\inv\comp\varphi_U(b,v) = (b, [A_{UV}(b)](v))
	\]
	for every $b$ in $U\cap V$, where $A_{UV}$ is a function
	\[
		A_{UV}: U\cap V\to \gl{n}{\R}
	\]
	that assigns a linear transformation from $\gl{n}{\R}$ to every point $b$ of $U\cap V$.
	If it is possible to take $U$ to be all of $B$, then $\beta$ is a \emph{trivial bundle} and the map $U\times\RRN\to E$ is a \emph{trivialization} of the bundle.
	
	When $E$ and $B$ are smooth manifolds, $\pi$ is smooth, and the local trivializations are diffeomorphisms, $\beta$ is a \emph{smooth vector bundle}.
	We assume that the vector bundle structures we encounter over smooth manifolds are always smooth.
	
	A continuous map $s:B\into E$ that is a right inverse of $\pi$ is called a \emph{section}.
	The \emph{zero section} of $\beta$ is used to refer to both the map
	\[
		\begin{array}{crcl}
			z: & B & \to 	 & E \\
			   & b & \mapsto & (b,0),
		\end{array}
	\]
	and its image $z(B)$ in $E$.
	Note that the compatibility condition on local trivialization guarantees that the zero section is well defined.
	
	For $\beta$ a $k$--vector bundle, we can form new objects called \emph{$k$--disc bundles} by restricting the fibers of $\beta$ to vectors with length at most 1.
	The machinery we build for vector bundles restricts to this construction.
\end{defn}

Other spaces, such as spheres, may be taken as fibers to form the general \emph{fiber bundle}.
The definition follows that for vector bundles, except we lose the structure that guarantees the existence of a zero section.


\begin{defn}[Bundle Isomorphism]
	Let $\Phi:\beta_0\to \beta_1$ be a map between vector bundles.
	We call $\Phi$ a \emph{fiber map} if $\Phi:E_0\to E_1$ covers a map $\varphi:B_0\to B_1$.
	For $\Phi$ to cover $\varphi$, that means the following diagram commutes:
	\[
		\begin{tikzcd}
			E_0 \arrow{r}{\Phi} \arrow[swap]{d}{\pi_0} & E_1 \arrow{d}{\pi_1} \\
			B_0 \arrow{r}{\varphi} & B_1
		\end{tikzcd}
	\]
	This means that if $b\in B_0$ and $\varphi(b)=c$, then $\Phi$ maps $V_b$ to $V_c$ by a map we will denote $\Phi_b$.

	If $\Phi_b$ is a linear map for every $b\in B_0$, then we call $\Phi$ a \emph{bundle morphism}.
	If $\Phi$ is a bundle morphism, $B_0=B_1=B$, $\Phi_b$ is bijective for each $b\in B$, and $\varphi$ is $\ident{B}$, then $\Phi$ is a \emph{bundle isomorphism}.
	A bundle isomorphism $\Phi:E\to E$ is a \emph{bundle automorphism}.
\end{defn}

\begin{defn}[Tangent Bundle]
	Let $M$ be an $n$--manifold.
	We define the total space $TM$ of a smooth vector bundle with base space $M$ and fibers $\RRN\approx T_p M$ at any $p\in M$ by taking the disjoint union of all tangent spaces:
	\[
		TM = \bigsqcup_{p\in M} T_p M.
	\]
	The projection $\pi:TM\to M$ is defined by $\pi(q)=p$ for every $q\in T_p M$.
	We call $TM$ the \emph{tangent bundle} over $M$.
	A section $s:M\to TM$ is a \emph{vector field} on $M$.
\end{defn}

\begin{defn}[Normal Bundle]
	Let $X$ be a $k$--dimensional submanifold of the $(n+k)$--manifold $M$.
	At a point $p\in X\subset M$, the tangent space $T_p X$ is a subspace of the tangent space $T_p M$.
	Denote the orthogonal complement to $T_p X$ in $T_p M$ by $N_p X$ and call it the normal space at $p$ in $X$.
	That is, $T_p X\oplus N_p X = T_p M$.
	From linear algebra, $N_p X$ is an $n$--dimensional vector space.
	We define the total space $N_M X$ of a vector bundle with base space $X$ and fibers $\RRN\approx N_p X$ at any $p\in X$ by taking the disjoint union of all normal spaces:
	\[
		N_M X = \bigsqcup_{p\in X} N_p X.
	\]
	The projection $\pi:N_M X\to X$ is defined by $\pi(q)=p$ for every $q\in N_p X$.
	We call $N_M X$ the \emph{normal bundle} over $X$ in $M$.	
\end{defn}

\begin{defn}[Tubular Neighbourhood]
	\label{def:tubularneighbourhood}
	Let $X$ be a closed submanifold of the closed manifold $M$.
	A smooth embedding $f:N_M X\to M$ with $f(x,0)=x$ and $f(N_M X)$ an open neighbourhood of $X$ in $M$ is called a \emph{tubular neighbourhood} of $X$ in $M$.
	We often denote the pair $(X,f)$ by $\nu_M X$.
\end{defn}

Similar to the tubular neighbourhood is the \emph{closed tubular neighbourhood}, which has instead the structure of a normal disc bundle over $X$.
We use $\D_M X\subset N_M X$ to denote the disc bundle over $X$, and $\overline{\nu}_M X$ to denote a closed tubular neighbourhood.
This naming convention refers to topological closure.

\begin{theorem}[Existence of Tubular Neighbourhoods, Theorem 4.5.2 in \cite{Hirsch67}]
	\label{thm:tubularneighbourhood}
	Let $X$ be a closed submanifold of the closed manifold $M$.
	Then $X$ has a tubular neighbourhood in $M$.
\end{theorem}

\begin{defn}
	\label{defthm:collar}
	Let $M$ be a manifold and $X$ a codimension 1 submanifold of $M$ with $X\subset \pd M$.
	A \emph{collar} of $X$ is a diffeomorphism $f: X\times \HH\to X_U$ such that $U_X$ is an open neighbourhood of $X$ in $M$ and $f(x,0)=(x)$ for every $x\in X$.
	The set $U_X$ is a \emph{collar neighbourhood} of $X$.
	As with tubular neighbourhoods, we denote a collar by $(X,f)$.
	If a collar exists for $X$, we say that \emph{X admits a collar}.
\end{defn}

\begin{prop}
	\label{prop:collar1}
	Let $M$ be a manifold with nonempty $\pd_1 M$ such that $\pd_j M$ is empty for all $j>0$.
	Then any open subset of $\pd_1 M$ admits a collar.
\end{prop}

Collars are a standard object in the study of manifolds without corners, and Proposition~\ref{prop:collar1} is the standard result.
The proof found in \cite{Kosi93} relies on the existence of solutions to inward pointing differential vector fields defined on $\pd M$, and the vector field defined in \cite{Kosi93} can be adapted to manifolds with corners.

\begin{prop}
	Let $M$ be a manifold with faces for which $\{(U,\varphi)\}$ is an atlas
	There is a vector field $s$ on $M$
	
\end{prop}

\begin{prop}
	\label{prop:collar2}
	Let $M$ be a manifold with nonempty $\pd M$.
	Then $\pd M$ admits a collar.
\end{prop}

\begin{proof}
	The proof is nearly identical to that found in \cite{Kosi93} except with a differnt inward pointing differential vector field.
	
	
\end{proof}

\begin{prop}
	Let $M$ be a manifold and $X$ a codimension 1 submanifold of $M$ with $X$ a connected subset of $\pd M$.
	If $X$ is an open subset of a face of $M$, then $X$ admits a collar.
\end{prop}

A collar can be thought of as a half--tubular neighbourhood of a submanifold of the boundary.
Note that the collar is actually a fiber bundle with base $X\subset \pd M$, fiber $\HH$, and trivial structure.
This allows us to relate our future results on tubular neighbourhoods to include collars.
In particular, tubular neighbourhoods are unique up to a fiber-preserving isotopy, which we define now.

\begin{defn}[Isotopy]
	\label{def:isotopy}
	Let $X$ be a smoothly embedded submanifold of $M$.
	An \emph{isotopy} of $X$ in $M$ is a smooth homotopy
	\[
		\begin{array}{crcl}
			F: & X\times \I & \to & M \\
			   & F(x,t) & = & F_t(x)
		\end{array}
	\]
	such that the related map
	\[
		\begin{array}{cccc}
			\hat{F}: & X\times\I & \to & M\times\I \\
					 & (x,t) & \mapsto & (F_t(x),t)
		\end{array}
	\]
	is an embedding.
	The submanifolds $F_0(X)$ and $F_1(X)$ are \emph{isotopic}.
	When $X=M$ and $F_t$ is a diffeomorphism for each $t$, $F$ is a \emph{diffeotopy} of $M$.
	
	Let $X$ be a smoothly embedded closed submanifold of $M$ and consider a pair of tubular neighbourhoods $f,g:N_M X\to M$ of $X$ in $M$.
	An isotopy
	\[
		\begin{array}{crcl}
			F: & N_M X\times\I & \to & M \\
			   & F(x,t) & = & F_t(x)
		\end{array}
	\]
	satisfying the following properties:
	\begin{enumerate}
		\item $F_0=f$ and $F_1=g$,
		\item $F_0(N_M X)=F_1(N_M X)$,
		\item $F_1\inv\comp F_0$ is a vector bundle isomorphism $N_M X\to N_M X$,
	\end{enumerate}
	 is an \emph{isotopy of tubular neighbourhoods}, and the tubular neighbourhoods $(X,f)$ and $(X,g)$ are \emph{isotopic}.
\end{defn}

\begin{prop}
	For $M$ a manifold, isotopy and diffeotopy are equivalence relations on the space of submanifolds of $M$.
\end{prop}

With these definitions at hand, we can precisely state the uniqueness result for tubular neighbourhoods and the similar result for collars.

\begin{theorem}[Uniqueness of Tubular Neighbourhoods, Theorem 3.3.1 of \cite{Kosi93}]
	\label{prop:uniquenesstubularneighbourhood}
	Let $X$ be a closed submanifold of $M$.
	Any pair of tubular neighbourhoods of $X$ in $M$ are isotopic.
\end{theorem}

\begin{theorem}[Uniqueness of Collars]
	Let $M$ be a manifold and let $X$ be a codimension 1 submanifold of $M$ with $X\subset \pd M$.
	Let $(X,f)$ and $(X,g)$ be collars of $X$.
	Then $f(X\times\HH)$ and $g(X\times\HH)$ are isotopic through an isotopy $F:(X\times\HH)\times\I\to M$ with $F_0=f$, $F_1=g$, and $F_t(x,0)=x$ for every $x\in X$.	
\end{theorem}

There is a stronger uniqueness theorem for tubular neighbourhoods that uses a tighter definition.
It essentially says that two of these neighbourhoods are isotopic through a diffeotopy that is stationary outside of a small neighbourhood of the tube.

\begin{theorem}[Isotopy Extension]
	\label{thm:isotopyextension}
	Let $X$ be a smooth compact submanifold of the smooth closed manifold $M$, and let
	\[
		\begin{array}{crcl}
			F: & X\times\I & \to & M \\
			   & F(x,t) & = & F_t(x)
		\end{array}
	\]
	be an isotopy of $X$ in $M$.
	Let $L$ be the subset of $M$ equal to the union of the images of $F_t$ for each $t$.
	More precisely,
	\[
		L = \bigcup_{t\in[0,1]} F_t(X).
	\]
	Then there exists a diffeotopy 
	\[
		\begin{array}{crcl}
			G: & M\times\I & \to & M \\
			   & G(y,t) & = & G_t(y)				
		\end{array}
	\]
	with $G_0=\ident{M}$, $G_1$ equal to $F_1$ on $X\subset M$, and $F_t$ the identity on $M$ outside of an arbitrarily small neighbourhood of $L$ for all $t$.
\end{theorem}

\begin{defn}[Ambient Isotopy]
	\label{def:ambientisotopy}
	Let $X$, $M$, $F$ be as in Theorem \ref{thm:isotopyextension}.
	The isotopy $G:M\times\I\to M$ guaranteed by Theorem \ref{thm:isotopyextension} is called an \emph{ambient isotopy}.
	The images $F_0(X)$ and $F_1(X)$ are \emph{ambiently isotopic} as submanifolds of $M$.
\end{defn}

When the neighbourhood is closed, isotopy extension can strengthen the uniqueness of tubular neighbourhoods theorem to one that is unique through an ambient isotopy.
We can perform a similar strengthening on open tubular neighbourhoods as long as we restrict the definition slightly.

\begin{defn}
	A \emph{proper tubular neighbourhood} is a tubular neighbourhood that is the interior of a closed tubular neighbourhood.
\end{defn}

Examples of improper tubular neighbourhoods would be $\RRN$ as a neighbourhood of the origin in $\RRN$, or the strip $\{ (x,y)\in \RR : |x|< \frac{\pi}{2}\}$ as a neighbourhood of the line $x=0$ in $\RR$ whose fibers are the curves $y=\tan x + C$ for each $C\in\R$.
An example of a proper tubular neighbourhood would be the same strip, but with fibers $y=C$.
It should be clear that the interior of a closed tubular neighbourhood is proper, and every proper tubular neighbourhood is the interior of a closed tubular neighbourhood.

\begin{theorem}[Uniqueness of Proper Tubular Neighbourhoods, Theorem 3.3.5 of \cite{Kosi93}]
	Let $X$ be a closed submanifold of $M$.
	Any two proper tubular neighbourhoods of $X$ are isotopic through an isotopy that can be extended to an ambient isotopy.
\end{theorem}
