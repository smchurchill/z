A common tool in both the construction and description of a manifold is a \emph{bundle}.
The ultimate construction in this section is the tubular neighbourhood of a submanifold, which is unique up to a fibre-preserving isotopy.

\begin{defn}[Bundle]
	A bundle is most concisely represented by the composition
	\[
		F\into E\overset{\pi}{\onto} X,
	\]
	where $X$ is called the \emph{base space}, $E$ the \emph{total space}, $F$ the \emph{fibre}, and $\pi$ the \emph{projection}.
	There are some basic requirements for the tuple $\beta = (F,E,X,\pi)$ to be a bundle.
	First, we require that $\pi$ be continuous, that the subspaces $F_x = \pi\inv(x)$ are each homeomorphic to $F$, and that for every point $x\in X$ there exists an open neighbourhood $U\subset X$ containing $x$ and a homeomorphism
	\[
		\varphi: U\times F\to \pi\inv(U)
	\]
	so that $\pi\comp\varphi\,(u,f)=u$ for any pair $(u,f)$ in $U\times F$.
	Such a homeomorphism is called a \emph{local trivialization}.
	Any point $p\in E$ may be represented by a pair $p=(x,f)$ with $x\in X$, $f\in F$, and $\pi(x,f)=x$ using a local trivialization.
	
	A bundle that is simply $F\times X$ is called a \emph{trivial bundle}.
	When $F$ is a $k$--dimensional vector space, we call $\beta$ a \emph{$k$--vector bundle}.
	If $\beta$ is a $k$--vector bundle with base space an $n$--manifold, then $E$ is an $(n+k)$--manifold.
	
	A continuous map $s:X\into E$ so that $s(x)\in F_x$ for every $x\in X$ is called a \emph{section}.
	When $\beta$ is a vector bundle, we can define the \emph{zero section} of $\beta$ to be the map
	\[
		\begin{array}{cccc}
			z: & X & \to 	 & E \\
			   & x & \mapsto & (x,0).
		\end{array}
	\]
	We also call the subspace $z(X)\subset E$ the zero section of $E$.
\end{defn}

\begin{defn}[Bundle Isomorphism]
	Let $\Phi:\beta_0\to \beta_1$ be a map between fibre bundles.
	We call $\Phi$ a \emph{fibre map} if $\Phi:E_0\to E_1$ covers a map $\varphi:X_0\to X_1$.
	For $\Phi$ to cover $\varphi$, that means the following diagram commutes:
	\[
		\begin{tikzcd}
			E_0 \arrow{r}{\Phi} \arrow[swap]{d}{\pi_0} & E_1 \arrow{d}{\pi_1} \\
			X_0 \arrow{r}{\varphi} & X_1
		\end{tikzcd}
	\]
	This means that if $x\in X_0$ and $\varphi(x)=y$, then $\Phi$ maps $F_x$, the fibre over $x$, to $F_y$, the fibre over $y$, by a map we will denote $\Phi_x$.

	If $\beta_{0,1}$ are vector bundles and $\Phi_x$ is a linear map for every $x\in X_0$, then we call $\Phi$ a \emph{bundle morphism}.
	If $\Phi$ is a bundle morphism, $X_0=X_1=X$, $\Phi_x$ bijective for each $x\in X$, and $\varphi$ is $\ident{X}$, then $\Phi$ is a \emph{bundle isomorphism}.
	A bundle isomorphism $\Phi:E\to E$ is a \emph{bundle automorphism}.
\end{defn}

\begin{defn}[Tangent Bundle]
	Let $X$ be an $n$--manifold.
	We define the total space $TX$ of a vector bundle with base space $X$ and fibres $\RRN\approx T_p X$ at any $p\in X$ by taking the disjoint union of all tangent spaces:
	\[
		TX = \bigsqcup_{p\in X} T_p X.
	\]
	The projection $\pi:TX\to X$ is defined by $\pi(q)=p$ for every $q\in T_p X$.
	We call $TX$ the \emph{tangent bundle} over $X$.
\end{defn}

\begin{defn}[Normal Bundle]
	Let $X$ be a $k$--dimensional submanifold of the $(n+k)$--manifold $Y$.
	At a point $p\in X\subset Y$, the tangent space $T_p X$ is a subspace of the tangent space $T_p Y$.
	Denote the orthogonal complement to $T_p X$ in $T_p Y$ by $N_p X$ and call it the normal space at $p$ in $X$.
	That is, $T_p X\oplus N_p X = T_p Y$.
	From linear algebra, $N_p X$ is an $n$--dimensional vector space.
	We define the total space $N_Y X$ of a vector bundle with base space $X$ and fibres $\RRN\approx N_p X$ at any $p\in X$ by taking the disjoint union of all normal spaces:
	\[
		N_Y X = \bigsqcup_{p\in X} N_p X.
	\]
	The projection $\pi:N_Y X\to X$ is defined by $\pi(q)=p$ for every $q\in N_p X$.
	We call $N_Y X$ the \emph{normal bundle} over $X$ in $Y$.	
\end{defn}

\begin{defn}[Tubular Neighbourhood]
	\label{def:tubularneighbourhood}
	Let $X$ be a closed submanifold of the closed manifold $Y$.
	A smooth embedding $f:N_Y X\to Y$ with $f(x,0)=x$ and $f(N_Y X)$ an open neighbourhood of $X$ in $Y$ is called a \emph{tubular neighbourhood} of $X$ in $Y$.
	We often denote the pair $(X,f)$ by $\nu_Y X$.
\end{defn}

Similar to the tubular neighbourhood is the \emph{closed tubular neighbourhood}, which is defined by replacing the fibres $N_Y X$ by an appropriate closed $n$--disc.
This is a restriction of the normal bundle to vectors of length at most 1, so no new machinery is needed.
The notion of ``closed'' used in this naming convention is that of topological closure.

\begin{theorem}[Existence of Tubular Neighbourhoods, Theorem 4.5.2 in \cite{Hirsch67}]
	\label{thm:tubularneighbourhood}
	Let $X$ be a closed submanifold of the closed manifold $Y$.
	Then $X$ has a tubular neighbourhood in $Y$.
\end{theorem}

Similar to the tubular neighbourhood is the collar of a manifold's boundary.

\begin{prop}[Collar]
	\label{defthm:collar}
	Let $Y$ be a manifold with nonempty boundary $X = \pd Y$.
	There exists an open neighbourhood $U$ of $X$ in $Y$ and a diffeomorphism
	\[
	f: U \to X\times\HH^1
	\]
	where $f(X)=X\times \{0\}$.
	The pair $(X,f)$ is called a \emph{collar neighbourhood} or \emph{collar} of $X$.	
\end{prop}

Note that the collar is actually a fibre bundle with base $\pd Y$, fibre $\HH^1$, and trivial structure.
This allows us to relate our future results on tubular neighbourhoods to include collars.
In particular, tubular neighbourhoods are unique up to a fibre-preserving isotopy.
Let's define precisely what that means.

\begin{defn}[Homotopy]
	\label{def:homotopy}
	Let $f,g:X\to Y$ be smooth maps between smooth manifolds.
	Denote the closed unit interval $[0,1]$ by $\I$.
	A function 
	\[
		\begin{array}{cccc}
			H: & X\times \I & \to & Y \\
			   & (x,t)	& \mapsto & H_t(x)
		\end{array}
	\]
	with $H_0(x)=f(x)$ and $H_1(x)=g(x)$ is a \emph{homotopy} between $f$ and $g$.
	If a homotopy exists, then $f$ and $g$ are \emph{homotopic}.
	Less formally, $f$ and $g$ being homotopic means that one can be continuously deformed into the other.
	The topological spaces $X$ and $Y$ are \emph{homotopy equivalent} if there exist continuous maps $f:X\to Y$ and $g:Y\to X$ for which $g\comp f$ is homotopic to $\ident{X}$ and $f\comp g$ is homotopic to $\ident{Y}$.
	
	Because we are primarily interested in smooth functions to build our machinery, we extend our definition of homotopy to a smooth version.
	With the notation above, a smooth map $H:X\times[0,1]\to Y$ with $H_0(x)=f(x)$ and $H_1(x)=g(x)$ is a \emph{smooth homotopy} between $f$ and $g$.
	If a smooth homotopy exists, then $f$ and $g$ are \emph{smoothly homotopic}.
\end{defn}

A homotopy through embeddings is called an isotopy.
This is useful in general for comparing embedded submanifolds.

\begin{defn}[Isotopy]
	\label{def:isotopy}
	Let $X$ be a smoothly embedded submanifold of $Y$.
	An \emph{isotopy} of $X$ in $Y$ is a smooth homotopy
	\[
		\begin{array}{crcl}
			F: & X\times \I & \to & Y \\
			   & F(x,t) & = & F_t(x)
		\end{array}
	\]
	such that the related map
	\[
		\begin{array}{cccc}
			\hat{F}: & X\times\I & \to & Y\times\I \\
					 & (x,t) & \mapsto & (F_t(x),t)
		\end{array}
	\]
	is an embedding.
	The submanifolds $F_0(X)$ and $F_1(X)$ are \emph{isotopic}.
	When $X=Y$ and $F_t$ is a diffeomorphism for each $t$, $F$ is a \emph{diffeotopy} of $Y$.
	
	Let $X$ be a smoothly embedded closed submanifold of $Y$ and consider a pair of tubular neighbourhoods $f,g:N_Y X\to Y$ of $X$ in $Y$.
	An isotopy
	\[
		\begin{array}{crcl}
			F: & N_Y X\times\I & \to & Y \\
			   & F(x,t) & = & F_t(x)
		\end{array}
	\]
	satisfying the following properties:
	\begin{enumerate}
		\item $F_0=f$ and $F_1=g$,
		\item $F_0(N_Y X)=F_1(N_Y X)$,
		\item $F_1\inv\comp F_0$ is a vector bundle isomorphism $N_Y X\to N_Y X$,
	\end{enumerate}
	 is an \emph{isotopy of tubular neighbourhoods}, and the tubular neighbourhoods $(X,f)$ and $(X,g)$ are \emph{isotopic}.
\end{defn}

This leads into our uniqueness result for tubular neighbourhoods.

\begin{theorem}[Uniqueness of Tubular Neighbourhoods, Theorem 3.3.1 of \cite{Kosi93}]
	\label{prop:uniquenesstubularneighbourhood}
	Let $X$ be a closed submanifold of $Y$.
	Any pair of tubular neighbourhoods of $X$ in $Y$ are isotopic.
\end{theorem}

A similar theorem also applies to collar neighbourhoods.

\begin{theorem}
	Let $Y$ be a manifold with boundary $X = \pd Y$.
	Let $(X,f)$ and $(X,g)$ be collars of $X$.
	Then $f(X\times\HH^1)$ and $g(X\times\HH^1)$ are isotopic through an isotopy $F:(X\times\HH^1)\times\I\to Y$ with $F_0=f$, $F_1=g$, and $F_t(x)=x$ for every $x\in X$.	
\end{theorem}

There is a stronger uniqueness theorem for tubular neighbourhoods that uses a tighter definition.
It essentially says that two of these neighbourhoods are isotopic through a diffeotopy that is stationary outside of a small neighbourhood of the tube.

\begin{theorem}[Isotopy Extension]
	\label{thm:isotopyextension}
	Let $X$ be a smooth compact submanifold of the smooth closed manifold $Y$, and let
	\[
		\begin{array}{crcl}
			F: & X\times\I & \to & Y \\
			   & F(x,t) & = & F_t(x)
		\end{array}
	\]
	be an isotopy of $X$ in $Y$.
	Let $L$ be the subset of $Y$ equal to the union of the images of $F_t$ for each $t$.
	More precisely,
	\[
		L = \bigcup_{t\in[0,1]} F_t(X).
	\]
	There exists a diffeotopy 
	\[
		\begin{array}{crcl}
			G: & Y\times\I & \to & Y \\
			   & G(y,t) & = & G_t(y)				
		\end{array}
	\]
	with $G_0=\ident{Y}$, $G_1$ equal to $F_1$ on $X\subset Y$, and $F_t$ the identity on $Y$ outside of an arbitrarily small neighbourhood of $L$ for all $t$.
\end{theorem}

\begin{defn}[Ambient Isotopy]
	\label{def:ambientisotopy}
	Let $X$, $Y$, $F$ be as above.
	The isotopy $G:Y\times\I\to Y$ guaranteed by Theorem \ref{thm:isotopyextension} is called an \emph{ambient isotopy}.
	The images $F_0(X)$ and $F_1(X)$ are \emph{ambiently isotopic} as submanifolds of $Y$.
\end{defn}

Notice that, when the neighbourhood is a closed, isotopy extension can strengthen the uniqueness of tubular neighbourhoods theorem to one that is unique through an ambient isotopy.
We can perform a similar strengthening on open tubular neighbourhoods as long as we restrict the definition slightly.

\begin{defn}
	A tubular neighbourhood that is obtained by an arbitrarily small shrinking of another tubular neighbourhood is called \emph{proper}.
\end{defn}

Examples of improper tubular neighbourhoods would be $\RRN$ as a neighbourhood of the origin in $\RRN$, or the strip $\{ x\in \RR : \norm{x}< \frac{\pi}{2}\}$ as a neighbourhood of the line $x=0$ in $\RR$ whose fibres are the curves $y=\tan x + c$.
An example of a proper tubular neighbourhood would be the same strip, but with fibres $y=c$.
It should be clear that the interior of a closed tubular neighbourhood is proper, and every proper tubular neighbourhood is the interior of a closed tubular neighbourhood.

\begin{theorem}[Uniqueness of Proper Tubular Neighbourhoods, Theorem 3.3.5 of \cite{Kosi93}]
	Let $X$ be a closed submanifold of $Y$.
	Any two proper tubular neighbourhoods of $X$ are isotopic through an isotopy that can be extended to an ambient isotopy.
\end{theorem}


%The existence of tubular neighbourhoods was guaranteed only for closed submanifolds of closed manifolds.
%We can extend our results to manifolds with boundary as long as the boundaries satisfy some niceness conditions.

%\begin{defn}[Neat Submanifold]
%	We use here $\ttilde{\HH}^m$ to denote the subspace of $\HN$ for which the last $n-m$ coordinates are $0$. 
%	Let $X\subset Y$ be a $m$--submanifold of the $n$--manifold $Y$ satisfying the following conditions:
%	\begin{enumerate}
%		\item $X$ is a topologically closed subset of $Y$,
%		\item $X\cap \pd Y=\pd X$, and
%		\item for every $x\in\pd X$, there is a chart $(U,f)$ of $Y$ with $f:U\to\HN$ such that $f\inv(\tilde{\HH}^m)=U\cap X$.
%	\end{enumerate}
%	Then $X$ is a \emph{neat} submanifold of $Y$.
%	The last condition for neatness essentially guarantees that $\pd X$ meets $\pd Y$ the same way that %$\tilde{\HH}^m$ meets $\HN$.
%\end{defn}

%\begin{defn}[Neat Tubular Neighbourhood]
%	Let $X$ be a neat submanifold of $Y$, and $\nu X$ a tubular neighbourhood of $X$.
%	We say that $\nu X$ is a \emph{neat tubular neighbourhood} if $\nu X\cap \pd Y$ is a tubular neighbourhood of $\pd X$ in $\pd Y$.	
%\end{defn}

%\begin{theorem}[Theorem 3.4.2 of \cite{Kosi93}]
%	If $X$ is a neat submanifold of $Y$, then it has a neat tubular neighbourhood.
%\end{theorem}

For $X$ a closed submanifold of $\pd Y$, our definitions guarantee a tubular neighbourhood of $X$ in $\pd Y$, but not of $X$ in $Y$.
We now extend our definition of tubular neighbourhood to this instance, obtaining the last tools necessary to begin defining handles.

Letting $X=\pd Y$, notice that $\pd Y$ has a trivial normal bundle $\pd Y\times \R$ in $Y$, and a collar that is an embedding of $\pd Y\times\HH^1$ in $Y$.
It is clear that an appropriate analogue of the tubular neighbourhood to submanifolds of the boundary would be a generalization of the collar to a sort of ``half-tubular neighbourhood.''

\begin{defn}
	\label{def:halfneighbourhood}
	Let $Z$ be a manifold with boundary and let $X$ be a submanifold of $Y=\pd Z$.
	The normal bundle of $X$ in $Z$ is the fibre-wise direct sum of $N_Y X$ with $\restr{(N_Z Y)}{X}$.
	Because $N_Z Y$ is a trivial bundle, the normal bundle of $X$ in $Z$ is just $N_Y X\times \R$.
	The tubular neighbourhood of $X$ in $Z$ is then an embedding $\ttilde{f}:N_Y X\times \HH^1\to Z$ that extends an embedding of a tubular neighbourhood $(X,f:N_Y X\to Y)$ over a collar of $Y$.
\end{defn}

Our theorems regarding the existence and uniqueness of tubular neighbourhoods extend with the definition.
Proper tubular neighbourhoods are defined in exactly the same way, and uniqueness up to isotopy is now uniqueness up to composition of isotopies of $\nu_Y X$ and isotopies of the collar neighbourhood.