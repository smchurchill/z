Our first task is to build the machinery necessary to describe manifolds.
We begin with a quick, formal definition, and build some basic properties.
Next, we talk about a method of building and augmenting manifolds using handles.
We end by defining triangulations: a combinatorial description of a manifold that allows us to cleanly describe our algorithms.

This chapter is intended as a review of common tools in geometric topology so, in most places, full proof will be omitted.
A knowledge of the fundamental properties and definitions of smooth maps between real vector spaces is assumed.
For full details on the fundamentals of smooth manifold theory, refer to the texts \cite{GompStip}, \cite{Hirsch67}, \cite{Kosi93}, or \cite{Lee00}.
For details on manifolds with corners, refer to Chapter 3 of \cite{JSP}.