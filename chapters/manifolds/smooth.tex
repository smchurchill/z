\begin{defn}[Smooth Map]
	\label{def:smoothmap}
	Let $(X,\{U_\alpha,\varphi_\alpha\})$ and $(M,\{V_\beta,\psi_\beta\})$ be smooth $n$-- and $k$--manifolds, with $\{U_\alpha,\varphi_\alpha\}$ and $\{V_\beta,\psi_\beta\}$ maximal smooth atlases.
	Let $f:X\to M$ be a map.
	If, for any $\alpha$ and $\beta$, the composition $\psi_\beta\comp f\comp \varphi_\alpha\inv$ is smooth as a map $$\psi_\beta\comp f\comp \varphi_\alpha\inv:\R^n\supset\varphi_\alpha(U_\alpha)\to\varphi_\beta(V_\beta)\subset\R^k,$$ then we say $f$ is \emph{smooth} as a map between manifolds.
	The space of smooth maps $X\to M$ is denoted $C^\infty(X,M)$.
	
	If $f$ is smooth and a well-defined $f\inv$ exists and is smooth, then $f$ is called a \emph{diffeomorphism} between manifolds.
	We say that manifolds are \emph{diffeomorphic} if there exists a diffeomorphism between them.
\end{defn}

\begin{prop}
	\label{prop:diffeoequiv}
	Diffeomorphism is an equivalence relation on the space of smooth manifolds.
\end{prop}

Manifolds are defined by their local homogeneity, and a tangent space is a precise description of that homogeneity near a point.
There are many ways to define tangent spaces, and we do so using derivations as in \cite{Lee00}.
This approach is used for two reasons.
First, it causes the vector space structure of the tangent space to follow directly from the definition.
Second, the definition is applicable to manifolds with faces without any fiddling around with details, and the tangent spaces obtained are still vector spaces of an appropriate dimension.

\begin{defn}[Derivations on $\R^n$]
	Let $x$ be a point in $\R^n$, and denote the space of smooth functions $\R^n\to\R$ be $C^\infty(\R^n)$.
	A linear map $A:C^\infty(\R^n)\to\R$ is called a \emph{derivation at $x$} if it satisfies the product rule
	\[
		A(\varphi\,\psi)=\varphi(x)A\psi+\psi(x)A\varphi
	\]
	for any $\varphi,\psi\in C^\infty(\R^n)$.
	Denote the set of all derivations at $x$ of $C^\infty(\R^n)$ by $T_x \R^n$.
	The derivations $T_x \R^n$ form a vector space under the operations
	\[
		\begin{array}{rcl}
			(A+B)\varphi&=&A\varphi+B\varphi,\\
			(cA)\varphi&=&c(A\varphi).
		\end{array}
	\]
\end{defn}

\begin{prop}[Corollory 3.3 in \cite{Lee00}]
	\label{prop:cor33}
	For any $x$ in $\R^n$, the $n$ derivations
	\[
		\restr{\frac{\pd}{\pd x_1}}{x}, \dots, \restr{\frac{\pd}{\pd x_n}}{x}
	\]
	defined by
	\[
		\restr{\frac{\pd}{\pd x_i}}{x}\varphi=\frac{\pd \varphi}{\pd x_i}(x)
	\]
	form a basis for $T_x\R^n$, and $T_x\R^n$ is therefore an $n$--dimensional vector space. 
\end{prop}

Our definition of derivation extends to the spaces $\HH^k\times\R^{n-k}$, and Proposition~\ref{prop:cor33} extends as well after we establish the following lemma.

\begin{lem}[Extension Lemma, 2.7 in \cite{Lee00}]
	\label{lem:ext}
	Let $M$ be a smooth manifold, let $V\subset M$ be a closed subset, and let $f:V\to\R^d$ be a smooth function.
	Let $U$ be any open set containing $V$.
	Then there exists a smooth extension $\tilde{f}:M\to\R^d$ such that $\restr{\tilde{f}}{V}=f$ and $\supp \tilde{f}\subset U$.
\end{lem}

\begin{defn}[Derivations on $\HH^n$]
	Let $x$ be a point in $\HH^n$.
	A linear map $A:C^\infty(\HH^n)\to\R$ is called a \emph{derivation at $x$} if it satisfies the product rule
	\[
	A(\varphi\,\psi)=\varphi(x)A\psi+\psi(x)A\varphi
	\]
	for any $\varphi,\psi\in C^\infty(\HH^n)$.
	Denote the set of all derivations at $x$ of $C^\infty(\HH^n)$ by $T_x \HH^n$.
	The derivations $T_x \HH^n$ form a vector space under the operations
	\[
	\begin{array}{rcl}
	(A+B)\varphi&=&A\varphi+B\varphi,\\
	(cA)\varphi&=&c(A\varphi).
	\end{array}
	\]
\end{defn}

\begin{prop}[Lemma 3.10 in \cite{Lee00}]
	\label{prop:cor310}
	For any $x$ in $\HH^n$, the $n$ derivations
	\[
	\restr{\frac{\pd}{\pd x_1}}{x}, \dots, \restr{\frac{\pd}{\pd x_n}}{x}
	\]
	defined by
	\[
	\restr{\frac{\pd}{\pd x_i}}{x}\varphi=\frac{\pd \varphi}{\pd x_i}(x)
	\]
	form a basis for $T_x\HH^n$, and $T_x\HH^n$ is therefore an $n$--dimensional vector space. 
\end{prop}

\begin{proof}
	The case where $x$ is in the interior of $\HH^n$ is exactly the same as in Proposition~\ref{prop:cor33}.
	We concern ourselves only with the case when $x$ is a boundary point of $\HH^n$.
	
	Let $i:\HH^n\to\R^n$ be inclusion.
	We will show that $i_*:T_x\HH^n\to T_x\R^n$ is an isomorphism.
	In the case where $i_*A=0$, let $\varphi$ be a smooth, real--valued function defined on a neighbourhood of $x$ in $\HH^n$ and let $\tilde{\varphi}$ extend $\varphi$ to a smooth function on an open subset of $\R^n$.
	From this, we have $\tilde{\varphi}\comp i=\varphi$, so
	\[
		A\varphi = A(\tilde{\varphi}\comp i) = (i_* A)(\tilde{\varphi})=0.
	\]
	This means that $i_*A=0$ implies $A=0$, so $i_*$ is injective.
	For surjectivity, let $B\in T_x\R^n$ and let $\varphi\in C^\infty(\HH^n)$.
	Define $A\in T_x \HH^n$ by
	\[
		A\varphi = B\tilde{\varphi},
	\]
	where $\tilde{\varphi}$ is any extension of $\varphi$.
	Expressing $B$ in terms of the basis for $T_x \R^n$ on the right hand side of this expression gives us
	\[
		A\varphi = \sum_{i=1}^n B_i \frac{\pd \tilde{\varphi}}{\pd x_i}(x).
	\]
	The derivatives $\pd\tilde{\varphi}/\pd x_i$ are determined by the derivatives of $\varphi$ in $\HH^n$, and this definition of $A$ is shown to be a derivation by evaluating $A(\varphi\psi)$ as $B(\tilde{\varphi}\tilde{\psi})$.
	Confirming that $i_*A=B$ uses $\varphi=\tilde{\varphi}\comp i$ as above, and $i_*$ is therefore surjective.
	We conclude that $i_*$ is an isomorphism.
\end{proof}

Note that $\HH^n$ contains model corner points for every type of $j$--corner, $0\leq j\leq n$, along the intersections of its coordinate planes.
Investigating the tangent spaces in $\HH^k\times\R^{n-k}$ allows us to fully realize tangent spaces to manifolds through the use of charts.

\begin{defn}
	Let $M$ be an $n$--manifold and $p$ a point in $M$.
	A linear map $A:C^\infty(M)\to\R$ is a \emph{derivation at p} if it satisfies
	\[
		A(\varphi\,\psi)=\varphi(p)A\psi+\psi(p)A\varphi
	\]
	for any $\varphi,\psi\in C^\infty(M)$.
	Denote the set of all derivations at $p$ of $C^\infty(M)$ by $T_p M$.
	As in the case with $\R^n$, $T_p M$ is a vector space under
	\[
		\begin{array}{rcl}
			(A+B)\varphi&=&A\varphi+B\varphi,\\
			(cA)\varphi&=&c(A\varphi).
		\end{array}
	\]
	We call $T_p M$ the \emph{tangent space to $M$ at $p$} and call the elements of $T_p M$ \emph{tangent vectors}.
\end{defn}

To find a basis, and therefore dimension, of a tangent space we first define how smooth maps act on tangent vectors.

\begin{defn}[Pushforward]
	Let $f:X\to M$ be a smooth map between manifolds.
	For every $p\in X$, we define the map
	\[
		f_*:T_p X\to T_{f(p)} M
	\]
	by
	\[
		(f_*A)(\varphi) = A(\varphi\comp f).
	\]
	If $\varphi\in C^\infty(X)$ then $\varphi\comp f\in C^\infty(M)$, so $X(\varphi\comp f)$ is, at least, defined.
	The operator is clearly linear and it is a derivation at $f(p)$ by
	\[
		\begin{array}{rcl}
			(f_* A)(\varphi\,\psi)&=&A((\varphi\,\psi)\comp f)\\
			&=&X((\varphi\comp f)(\psi\comp f))\\
			&=&(\varphi\comp f(p))A(\psi\comp f) + (\psi\comp f(p))A(\varphi\comp f)\\
			&=&\varphi(f(p))(f_* A)(\psi) + \psi(f(p))(f_* A)(\varphi).
		\end{array}
	\]
	The map $f_*$ is called the \emph{pushforward} of $f$, and it relates how the tangent vectors in $T_p X$ are transformed into tangent vectors of $T_{f(p)}M$ when $p$ is passed through $f$.
\end{defn}

\begin{lem}[Lemma 3.5 in \cite{Lee00}]
	\label{lem:pushprop}
	Let $f:X\to M$ and $g:M\to N$ be smooth maps.  Let $p\in X$.
	\begin{enumerate}
		\item $f_*:T_p X\to T_{f(p)}M$ is linear.
		\item $(g\comp f)_*=g_*\comp f_*:T_p X\to T_{g\comp f(p)}N$
		\item $(\id_X)_*=\id_{T_p X}:T_p X\to T_p X$
		\item If $f$ is a diffeomorphism then $f_*:T_p X\to T_{f(p)}M$ is an isomorphism.
	\end{enumerate}
\end{lem}

\begin{prop}[Proposition 3.7 in \cite{Lee00}]
	\label{prop:incl}
	Let $M$ be a smooth manifold and $U\subset M$ an open submanifold.
	Let $i:U\to M$ be inclusion.
	For any $p\in U$, the pushforward $i_*:T_p U\to T_p M$ is an isomorphism.	
\end{prop}

We know that $T_x \HH^n$ is $n$--dimensional for any $x$ in $\HH^n$.
Combining Lemma~\ref{lem:pushprop} and Proposition~\ref{prop:incl} with a smooth chart proves the corresponding dimension theorem for tangent spaces to smooth manifolds.

\begin{lem}
	Let $M$ be a smooth $n$--manifold and $p$ a point in $M$.
	Then $T_p M$ is an $n$--dimensional vector space with basis given by the coordinate vectors
	\[
		\restr{\frac{\pd}{\pd x_1}}{p}, \dots,	\restr{\frac{\pd}{\pd x_n}}{p}
	\]	
	in any smooth chart.
\end{lem}

\begin{proof}
	Let $(U,\varphi)$ be a chart about $p$.
	Then $\varphi$ is a diffeomorphism from $U$ to an open subset $\tilde{U}$ of $\HH^k\times\R^{n-k}$ for some value of $k$. 
	By Lemma~\ref{lem:pushprop} and Proposition~\ref{prop:incl}, the pushforward $\varphi_*$ is an isomorphism $T_p M\to T_{\varphi(p)}\HH^k\times\R^{n-k}$.
	By Proposition~\ref{prop:cor310} $T_x\HH^k\times\R^{n-k}$ is an $n$--dimensional vector space with basis the derivations
	\[
		\restr{\frac{\pd}{\pd x_1}}{x}, \dots, \restr{\frac{\pd}{\pd x_n}}{x},
	\]
	so $T_p M$ is as well.
\end{proof}

We also classify smooth maps between manifolds.
This classification helps us define a submanifold and provides some of the groundwork for defining handles in the next section.

\begin{defn}[Embedding]
	Let $f:X\to M$ be a smooth map between manifolds.
	The \emph{rank} of $f$ at the point $p\in X$ is the rank of the pushforward $f_*$ as a linear map.
	This is computed as the rank of the matrix of partial derivatives of $f$ in any smooth chart or as the dimension of $f_*(T_p X)\subset T_{f(p)} M$.
	If $f$ has rank $k$ for every $p$ in $X$, then $f$ is of \emph{constant rank} and we say $\rk f=k$.
		
	If $f_*$ is injective at $p$, then $f$ is \emph{immersive} at $p$.
	If $f$ is an everywhere immersive, then it is an \emph{immersion}.
	This is equivalent to $\rk{f}=\dimn X$.
	If $f_*$ is surjective at $p$, then $f$ is \emph{submersive} at $p$, and $p$ is a \emph{regular point} of $f$.
	If $f$ is everywhere submersive, then it is a \emph{submersion}.
	This is equivalent to $\rk{f}=\dimn M$.
	If $\rk{f_*}$ is less than maximal, then $p$ is a \emph{critical point} of $f$.
	In this case, $f(p)=q$ is a \emph{critical value}.
	The set of critical points of $f$ in $M$ is called the \emph{singular set} of $M$, and is denoted $s_f$.
	If $p$ is a critical point and the Hessian of $f$, i.e.\ the matrix of second partial derivatives of $f$ in any smooth chart, is of less than full rank, then $p$ is a \emph{degenerate critical point}.
	For $q\in M$, the set of points $x$ in $X$ such that $f(x)=q$ is called the \emph{preimage} of $q$ through $f$ or the \emph{fiber} of $f$ over $q$ and is denoted $f\inv(q)$.
	A point $q \in M$ whose fiber consist entirely of regular points is a \emph{regular value} of $f$.
	
	If $f$ is an injective immersion that is a homeomorphism onto its image $f(X)\subset M$ under the subspace topology, then it is a \emph{smooth embedding} or just \emph{embedding}.
	
	Let $X\subset M$ where $M$ is a manifold, and let $i:X\into M$ be inclusion.
	If $i$ is an embedding, then we call $X$ an \emph{embedded submanifold} or \emph{submanifold} of $M$, and $X$ is, by itself, a manifold with smooth structure induced by the embedding.
\end{defn}

\begin{prop}
	\label{prop:boundariesaremanifolds}
	Let $M$ be an $(n+1)$--manifold.
	Any connected face in $\pd_f M$ is an $n$--dimensional submanifold of $M$.
\end{prop}

\begin{theorem}[Regular Value Theorem]
	Let $f:X\to M$ be a smooth map between manifolds.
	Let $q\in M$ be a regular value.
	The preimage $f\inv(q)$ is a $(\dim X-\dim M)$--submanifold of $X$.
\end{theorem}

Maps that are ``close enough'' to one another via perturbation tend to be considered equivalent.
This concept is made precise through homotopy.

\begin{defn}[Homotopy]
	\label{def:homotopy}
	Let $f,g:X\to M$ be smooth maps between smooth manifolds.
	Denote the closed unit interval $[0,1]$ by $\I$.
	Let $H$ be a continuous function 
	\[
		\begin{array}{cccc}
			H: & X\times \I & \to & M \\
			& (x,t)	& \mapsto & H_t(x)
		\end{array}
	\]
	with $H_t$ a continuous function $X\to M$ for all $t$, $H_0(x)=f(x)$, and $H_1(x)=g(x)$.
	Then $H$ is called a \emph{homotopy} between $f$ and $g$.
	If a homotopy exists, then $f$ and $g$ are \emph{homotopic}.
	Less formally, $f$ and $g$ being homotopic means that one can be continuously deformed into the other.
	The topological spaces $X$ and $M$ are \emph{homotopy equivalent} if there exist continuous maps $f:X\to M$ and $g:M\to X$ for which $g\comp f$ is homotopic to $\ident{X}$ and $f\comp g$ is homotopic to $\ident{M}$.
	
	Because we are primarily interested in smooth functions to build our machinery, we extend our definition of homotopy to a smooth version.
	With the notation above, a smooth map $H:X\times\I\to M$ with $H_0(x)=f(x)$ and $H_1(x)=g(x)$ is a \emph{smooth homotopy} between $f$ and $g$.
	If a smooth homotopy exists, then $f$ and $g$ are \emph{smoothly homotopic}.
	
	Let $M$ be a manifold and $X\subset M$ a subspace that is not necessarily a submanifold.
	A continuous map $F:M\times\I\to M$ such that, for every $m$ in $M$ and $x$ in $X$, $F(m,0)=m$, $F(m,1)\in X$, and $F(x,1)=x$, is called a \emph{deformation retraction}, and the subspace $X$ is called a \emph{deformation retract} of $M$.
	Because a deformation retraction is a homotopy equivalence between $M$ and $X$, a deformation retraction from $M$ onto $X$ exists if and only if $M$ and $X$ are homotopy equivalent.
\end{defn}

\begin{prop}
	For a fixed pair $(X,M)$ of spaces, homotopy and smooth homotopy are equivalence relations on the space of maps $X\to M$.
\end{prop}

\begin{defn}
	Let $X$, $M$ be smooth manifolds, let $P$ be a property that an element of $C^\infty(X,M)$ can have, and let $C_P^\infty(X,M)$ be the subset of elements of $C^\infty(X,M)$ that possess property $P$.
	A property $P$ is \emph{stable under perturbation} if, for any function $f$ possessing $P$ and any smooth homotopy $H:X\times \I\to M$ with $H_0=f$, there is an $\varepsilon>0$ such that $H_t$ possesses $P$ for any $t<\varepsilon$.
	A property $P$ is considered \emph{generic} if $C_P^\infty(X,M)$ is open and dense in $C^\infty(X,M)$.
\end{defn}

The property of having no degenerate critical points is our standard model for genericity.
Consider the map $f:\R\to\R$, $f(x)=x^3$.
The critical point at $x=0$ is degenerate, but it may be perturbed into a pair of nondegenerate critical points or into a map with no critical points, and for either of these maps that property is stable.
Because degenerate critical points may be easily ``fixed,'' a generic smooth map is one whose critical points are all nondegenerate.
Another generic property is transversality.

\begin{defn}
	Let $M$ be an $n$--manifold with submanifolds $X$ and $Y$.
	We say that $X$ and $Y$ are \emph{transverse} as submanifolds if, at every intersection point, the sum of their tangent spaces is the tangent space of $M$ at that point.
	Symbolically, transversality is denoted by $\pitchfork$ and the condition is stated by $X\pitchfork Y$ if and only if for every point $p\in X\cap Y$, $T_p X+ T_p Y=T_p M$.
	We may also say that an intersection point is, itself, \emph{transverse}, or that $X$ and $Y$ \emph{intersect transversally at a point}.	
\end{defn}

Transversality of a pair of submanifolds $X$, $Y$ is stable under small perturbations of the embeddings $X\into M$ and $Y\into M$.
A point of $X\cap Y$ that is not a transverse intersection may be perturbed into stable transversality or non-intersection, so we say that intersections of submanifolds with complementary codimension are generically transverse.

The main result of this work is applicable to orientable 3--manifolds, so we will review what is meant by a space being orientable.
Orientability is the guarantee that when you go for a walk through a manifold you don't return home to find that your right and left sides have switched places.

We first define orientations on artibrary real vector spaces, which allows us to define orientations on $\RRN$.
Orientations of $\RRN$ let us define orientations of manifolds.
Orientations of manifolds let us define a smooth map as orientation preserving or orientation reversing.


\begin{defn}
	\label{def:orientation}
	Let $V^n$ be an $n$--dimensional real vector space, $\mathcal{B}(V^n)$ the set of ordered bases for $V^n$, and $\gl{n}{\R}$ the general linear group, i.e.\ the space of $n\times n$ invertible matrices with entries in $\R$.
	Let $b_1$ and $b_2$ be any two ordered bases from $\mathcal{B}(V^n)$.
	It is a standard result of linear algebra that there is a unique element $A$ of $\gl{n}{\R}$ that transforms $b_1$ into $b_2$.
	If the determinant $\det A$ is positive, then $b_1$ and $b_2$ are \emph{positively oriented} with respect to each other.
	If $\det A$ is negative, then $b_1$ and $b_2$ are \emph{negatively oriented}.
	We can define an equivalence relation $\sim$ on $\mathcal{B}(V^n)$ by saying that $b_1\sim b_2$ if they are positively oriented.
	An \emph{orientation} of $V^n$ is a choice of one of the two equivalence classes of $\mathcal{B}(V^n)$.
	This also allows us to classify linear transformations of $\gl{n}{\R}$ by their action on the quotient space $\mathcal{B}(V^n)/\sim$.
	If $A\in\gl{n}{\R}$ has positive determinant then it fixes the equivalent classes, and $A$ is \emph{orientation preserving}.
	If $A$ has negative determinant then it swaps the equivalence classes, and $S$ is \emph{orientation reversing}.
\end{defn}



\begin{defn}
	Suppose we have fixed an orientation of $\RRN$.
	Let $M$ be an $n$--manifold.
	We say that an \emph{orientation} of $M$ is a consistent choice of orientation of the tangent space at every point of $M$.
	A consistent choice of orientation means that for every chart $(U,f)$ with $f:U\to\RRN$, the vector space isomorphism $df:T_p M\to \RRN$ is orientation preserving at every point in $U$. 
	If $M$ admits an orientation, then it is \emph{orientable}.
	If $M$ does not admit an orientation, then it is \emph{non-orientable}.	
\end{defn}



\begin{defn}
	Let $f:X\to M$ be a smooth map between oriented $n$--manifolds.
	If the differential $df_p:T_p X\to T_{f(p)} M$ is orientation preserving (resp. reversing) as a map between vector spaces for every $p$ in $X$, then $f$ is \emph{orientation preserving} (resp. \emph{reversing}) as a map.
\end{defn}
